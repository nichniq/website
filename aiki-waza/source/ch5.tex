\chapter[Somatic Elements in Social Conflict (2006)]{Somatic Elements in Social Conflict\footnote{Revised version of a paper presented at the 37th World Congress of the International Institute of Sociology, Stockholm, Sweden. July 6, 2005. For help in revising I thank Michael Bare, Daniel Kimmel, Paki Reid-Brossard, Dan Silver, and Mark Walsh.}}

\section*{Introduction}

Social conflict presents a topic where the wish to bring bodies into sociological analysis should meet no resistance. Although conflict theory can be dryly abstract, its close connections to the realities of physical combat, by metaphor when not literally, makes it easy to link representations of social conflict with the interaction of physical bodies. Think of conflict and you quickly bump up against bodies---yelling and screaming, pushing and shoving, punching and wrestling, stabbing and shooting. Even in purely verbal conflict the body swerves quickly into view: reddened faces, clenched jaws, tensed muscles, and quickened breath. Even when conflicts of interests or ideas are negotiated in a non-combative mode, differences in bodily posture and demeanor readily appear. And bodily changes manifest even when the parties to conflict are not in direct physical proximity.

Yet for the past century the literature on social conflict has ignored its corporeal substratum. Post-war classics---Coser (1956), Coleman (1957), Boulding (1962/1988), and Shelling (1960)---do not mention the body. Randall Collins's (1975) comprehensive \emph{Conflict Sociology} has nary a reference to the bodily dimensions of his subject, nor does Louis Kriesberg's (1998) compendious analysis of conflicts, destructive and constructive. Instead of bemoaning such neglect, suppose we turn the point around and view that neglect as understandable, if not warranted, given the paucity of theoretical resources on how to formulate such linkages. Suppose then that we address the problematic of social conflict and the body with an eye for openings through which we might insert fresh lines of substantive work.

As point of departure for such an effort I shall reference the contributions of Talcott Parsons. This will seem odd to those who consider Parsons irrelevant to such concerns on grounds that his stressing the normative dimension of action precluded engagement with the body in society. As with many other dismissive glosses on Parsons, this one is hard to square with a review of what he actually produced. In publications spanning more than thirty-five years, Parsons considered the organismic dimension of human action in a number of places. These encompass subjects related to age and sex, including seminal papers on the incest taboo and youth subcultures; contributions to the theory of socialization; analyses of the cultural framing of life and death; an influential discussion of the parameters of medical practice; classic papers on aggression and reactions to social strains; writings on the human body itself---with attention to such phenomena as proper clothing, treatment of bodily injuries, and norms regarding physicians' access to patients' bodies; and intermittent efforts to weave the corporeal dimension into the general theory of action, culminating in his testamentary `Paradigm of the Human Condition' in 1978.

In spite of these substantive contributions, in his general theory of action Parsons did not focus on the organismic dimension anywhere near to the extent that he did when analyzing the psychological, social, and cultural dimensions of action. To be sure, on occasion he signaled his awareness of the theoretical lacuna between the physiological body, as one of the external environments of action, and the orientations of actors. Given his commitment to the Weberian concept of action, which conceives action as subjectively meaningful conduct, Parsons had some sense of the difficulty involved in leaping from purely biological process to a process imbued with meanings. He attempted to address that problem with intermittent, almost perfunctory, glosses on what he called the \emph{behavioural organism}. This concept made it possible to represent aspects of action that involved the body but contrasted with the merely `vegetative' aspects of the organism's functioning. The conception of the behavioural organism came to include certain organ systems and physiological processes, especially those involved in mental functioning. Although the usual connotations of `behaviour' excluded meaning, Parsons used the term behaviour to mean that these processes represent capacities that manifested certain kinds of meaning. He located that dimension in what he was calling the adaptive subsystem of the general universe of action, and treated it in terms of the general quality of intelligence. Lidz and Lidz (1976) developed this notion further, emphasizing the separation from purely organismic processes by calling it the \emph{behavioural system} and incorporating thereunder work by Jean Piaget that analyzed complexes of intelligent operations as universal capacities.

Three decades later it appears that this piece of action theory has been left where Lidz and Lidz left it (Bare, 2006). In the meantime, explorations in other disciplines, related especially to education, have greatly extended the notion of behavioural capacity. The notion of intelligence has been expanded well beyond its earlier restricted sphere, to include a number of different functions including audiovisual powers, interpersonal skills, emotional capacities, and language abilities.\footnote{Howard Gardner (1983, 1993) has been a leading figure in this development. For its manifestation in liberal education programs at the undergraduate level, see Levine (2006).} Although the Lidzes' intervention threw new light on the topic, it rested on a questionable Cartesian split between body/mind and neglected the fact that humans possess, after all, only one nervous system. What is more, the body itself has come to be theorized as the seat of a number of powers of its own, involving kinesthetic perceptual abilities and movement skills, and has come to be understood as participating intimately in all of the other powers just enumerated. The latter field has been investigated and documented by work in the field known as somatics. In the words of one of the most brilliant pioneer somatic investigators, Moshe Feldenkrais, ``the most abstract thought has emotional- vegetative and sensory-motor components; the whole nervous system participates in every act'' (Feldenkrais, 1949, 26).

Following Piaget, Lidz and Lidz articulated the constituents of the behavioural system as `capacities to act which are intrinsic to human adaptation,' likening them to the notion of grammar in transformational linguistics; that is, grammar as denoting the ability of competent speakers to form sentences under any conditions (1976, 197). Adopting this notion provisionally, I propose to understand the behavioural system as signifying \emph{the repertoire of human capacities that consist of physical abilities and dispositions together with the somatic components of `non-physical' behaviours}. Accordingly, this would include physical capacities that are involved in the execution of conflict and the ability to control conflict. I shall return to the general issue of how to integrate the body conflict nexus into the general theory of action after I have reviewed afresh the general theory of conflict.

\section*{A Paradigm of Social Conflict}

To investigate conflictual phenomena thoroughly requires that we differentiate among types of conflict with respect to a variety of salient dimensions. These include the media of conflict (verbal/physical), intensity (violent/nonviolent), systemic location (internal/external), type of conflictual party (family, community, nation), and type of outcome (constructive/destructive). On the other hand, supposing that something is to be gained by considering conflict at a more abstract level, I shall outline a paradigm of generic conflict.

As a form of social interaction, conflict has properties that can be investigated without reference to the orientations of individual actors. Even so, its basic dynamics derive from actions of parties that can be represented as acts of individual subjects, as follows:

\begin{enumerate}
\item A makes a bid for conflict by aggressing against B (verbally or physically).
\item B responds through counter-attack of some sort. Thereafter,
\item A and B continue to engage in conflict, establishing
\begin{enumerate}
\item A static equilibrium in which conflict becomes a constitutive element of the relationship, or
\item A dynamic equilibrium in which both parties continue an escalating spiral, until one of them
\begin{enumerate}
\item Defeats the other, or
\item Tires or has a change of heart about the conflict, or
\item Responds to an outside force that dampens or resolves the conflict.
\end{enumerate}
\end{enumerate}
\end{enumerate}

From this paradigm, it follows that the elements involved in the generation of conflict will be the factors that dispose party A to aggress, party B to counterattack, and the two parties to continue waging their conflict.

What factors account for those dispositions toward aggressive action? From the literature on conflict I have culled six factors that abet the process (as well as two countervailing variables that dampen these dispositions).

\begin{enumerate}
\item \emph{Hostility level}. In his pioneering treatise on the subject, \emph{Conflict and Defense: A General Theory} (1962), Kenneth Boulding related the disposition to engage in conflict to an initial base of dispositions to aggress against others.
\item \emph{Reactivity}. James Coleman (1957) noted the tendency for conflict to escalate when a provoked party reacts in ways that antagonize the initiating party further, until the escalating process takes on a life of its own. Boulding designated the tendency for parties to react in this way as a reactivity coefficient.
\item \emph{Positional rigidity}. Boulding also viewed a factor that lessens the disposition for conflict to be a willingness to accept other satisfactory utilities as a substitute for one that another party craved equally. In a popular textbook on the subject, Roger Fisher and William Urry (1981) depict this as a capacity to alter `positions' regarding means to secure a particular `interest.'
\item \emph{Moral righteousness}. Hostile energy is intensified when conjoined with a sense of moral valorization. Georg Simmel analysed how conflict becomes intensified when objectified out of purely personal reactions into combat for a cause. Bettelheim and Janowitz (1950) identified a number of emotional dynamics in which out-groups were hated for qualities that the in-group members found unacceptable.
\item \emph{Weakness of conflict-aversive values}. a) Some cultures glorify combat and the virtues of the warrior. b) Conversely, Freud stressed the importance of internalized controls over the expression of social aggression: the superego process employs aggressive energy to inhibit or repress the activation of hostile impulses, the ego-ideal instantiates cultural ideals of harmony and peace. Durkheim similarly identified conscience and `effervescence' in groups as brakes on conflict.
\item \emph{Weakness of external dampening factors}. Parsons (1951) and Coleman (1957) among others delineated a range of social structural factors crucial to the existence of conflict. The absence or weakness of such factors facilitates the escalation of conflict. Conversely, the presence of such factors serves to dampen conflict. These factors include, for example, the activation of policing processes; the invocation of shared transcending values; the availability of mechanisms of cooptation, and customs that favor the resort to mediation.
\end{enumerate}

In what follows, I inquire into how these factors that generate or dampen conflictual actions relate to features of the physical body. Following Weber's authoritative definition of `action,' as behaviour to which some sort of meaning is attached, I ask: what kinds of conflict-relevant meaning might emanate from processes within the human body itself, and what supra-organismic variables imbue bodily conduct with meanings that relate to conflict?\footnote{Chris Shilling's recent discourse on the topic (2005), not to mention classic formulations like those of Max Scheler (1928/1961) and Talcott Parsons (1951/1964), iterates that streams of causality or influence flow in both directions.} I suggest renaming the site of these linkages as the \emph{actional organism}---the subsystem of action where the organism's input of energies and the inputs from sources of meanings meet and interpenetrate.

\section*{Somatic Elements That Promote Conflict}

\subsection*{1. The body and aggressive impulsivity}

When social science does appropriate knowledge about bodies into its discourse on conflict it often relies on assumptions about an inherent human disposition toward aggression. On the eve of World War I William James asserted that `our ancestors have bred pugnacity into our bone and marrow, and thousands of years of peace won't breed it out of us' ([1910] 1974, 314). In his landmark formulations on political realism, political scientist Hans Morgenthau argued that the social world results from forces inherent in human nature, which render it `inherently a world of opposing interests and of conflict among them' (1960, 4). Ethologist Konrad Lorenz (1966) depicted aggression as an essential part of the life-preserving organization of instincts, arguing that for numerous species conflict provides clear adaptive advantages: balancing the ecological distributions, selecting the fittest specimens through fights among rivals, mediating ranking orders needed for complex organizations, even instigating ceremonies that promote social bonding. Another ethologist, Nikolaas Tinbergen, likewise posits a universal instinctual proclivity to intraspecific conflict and finds human aggressiveness marked by a socially disruptive quality: `Man is the only species that is a mass murderer, the only misfit in his own society' (1968, 180).\footnote{This condition, Tinbergen explains, comes from a combination of instinctual, cultural, and technological factors. Whereas in other species and earlier human periods the impulse to fight was balanced by the fear response, humans have contrived cultural conditions that dampen the impulse to flee from battle, while the technology of fighting at a distance eliminates the taming effect of personal contact in face-to-face encounters.} More recently, Richard Wrangham and Dale Peterson (1996) summarize evidence from ethological studies to conclude that the human animal, and the male of the species preeminently, has inborn propensities to attack and kill others that exceed adaptive needs.

Despite vicissitudes of instinct theories, psychoanalytic psychology has tended to assume an innate reservoir of egoistic and aggressive impulses that, amplified by externalization and projection, flood into interpersonal conflicts. Freud thought violent conflict endemic to humans, both to resolve conflicts of interest and to express an `active instinct for hatred and destruction.' He bemoaned the destructiveness of modern warfare but held little hope that cultured aversions to war could overcome the aggressive dispositions so deeply rooted in man's biological makeup ([1932] 1939). Freud's theory posited a self-destructive `death instinct' which gets diverted from the self toward others, thereby producing a constant fund of conflictual energies. Most psychoanalysts rejected Freud's assumption of a death instinct and some substituted a destructive instinct for the polar opposite of the sexual instinct.

Freud and his followers view the human organism as a perpetually renewed source of instinctual energies that well up and produce inner discomfort until they get released. Psychic and somatic symptoms reflect failures in the personality's ability to release those instinctual tensions, which eventually find outlet through indirect channels. In one way or another human aggression represents a constantly flowing impulse that emanates from the human body such that humans can never escape the proclivity to harm if not destroy either themselves or others. Although Lorenz took a more positive view of conflict, he too espoused a mechanistic-hydraulic view of aggression. Lorenz likens aggression to a gas constantly being pumped into a container or to a liquid in a reservoir dischargeable through a spring-loaded valve at the bottom. In Lorenz's conception, energies specific for an instinctive act accumulate continuously in neural centers for that behavior, leading animals and humans to hunt for stimuli in order to trigger the release of those energies.

For those who view conflictual action in this perspective, the propensity to act out aggressive impulses is limited by one or both of two other basic drives. For Freudian psychology, the aggressive instinct is balanced by Eros, the drive to form harmonious relationships with others. For Tinbergen, it is limited by fear of the countervailing force of enemies. To some extent, Hobbes can be seen as combining both combative and pacific drives. The perpetual and restless desire of power after power to which all men are inclined would lead inexorably to constant civil strife were it not for the activation of an even stronger natural inclination: the wish to avoid violent death. Humans are also motivated by a wish to live comfortably by means of conveniences, which only a regime of peace can procure. So the impulse to aggress against others gets subordinated to a wish for peaceful coexistence, a condition procured by establishing a sovereign political authority.

The logic of Hobbes's argument can be modified to cover a variety of social arrangements designed to control conflict: the body is the home of divergent impulses including aggressiveness, but aggression can get inhibited by other propensities that support institutions designed to prevent conflict. This image of the body is not unlike what we find in writers like Nietzsche and Sorel. The latter visualize a natural human disposition to be fierce and combative, a disposition that (for them, unhappily) gets swamped by fear and desires for convenience, thereby deflecting martial impulses into innocuous channels.

What none of these theories offers, however, is a way of connecting those dispositions with the constitutive systems of bodily organisms, a way that the relatively new discipline of somatics may help to illumine. Such analyses would proceed, for example, from considering hormonal levels of aggressivity through neuronal responses that mobilize aggressive physical or verbal impulses. Acting out such impulses involves their translation into complex neuro-muscularskeletal responses. The behavioural capacity to enact those responses, and thereby direct aggressive energies toward some social object, brings hormonal levels into the orbit of human action. Hormonally grounded aggressivity is the portion of the actional organism that energizes a trained capacity to attack and injure others.

\subsection*{2. The body and conflictual reactivity}

In his analysis of the dynamics of social conflict, Boulding points to a second variable that figures in the equation regarding escalation of conflict. He refers to these as `reaction processes,' processes in which a movement by one party provokes a movement by the other which in turn changes the field of the first, and so on. He proposes to designate this variable as a reaction coefficient: `the amount by which the equilibrium level of hostility of the one increases per unit increase in the hostility of the other' (1962, 26). Whatever the degree of initiating aggressive impulses, the actuation of conflict depends essentially on some level of reactivity on the part of the attacked party. It depends further on the rate of change of the reaction coefficient as hostility from the other increases. As Boulding emphasizes, the reaction of a party depends on the images it holds, both of itself and of the other. The reaction coefficients are likely to be high if a party feels itself to be misunderstood.

With this variable, we enter the domain of the self and its vulnerabilities. The more a self is threatened, the more likely that party is to resort to ego-defensive measures. The more fragile or insecure the self, the more likely the party is to perceive itself as being misunderstood and to perceive slights where none exist or at least to exaggerate their import. It is here that a more recent school of thought within the psychoanalytic tradition makes an important contribution. This stems from the work of figures like Winnicott, Kohut, and Bowlby, who view the need for attachment to social objects as a more fundamental instinct than the disposition toward aggression. In this perspective, aggression is not a primary drive, but a response to threats to attachment. Its manifestation in physical violence is then viewed as a product of disintegration or fears of disintegration, in which counter-phobic responses reenact dissociated traumatic events that seem intolerable for individuals in groups (Smith, 1993; Scheff and Retzinger, 1991). Neurophysiological processes, in this view, bring bodily functions into the orbit of aggressivity through hard-wired anxiety.

\subsection*{3. The body and mental rigidity}

In discussing what he calls static models of conflict, Boulding analyses conflict in terms of interests rather than passions. In this context, he defines conflict as `a situation of competition in which the parties are \emph{aware} of the incompatibility of potential future positions and in which each party \emph{wishes} to occupy a position that is incompatible with the wishes of the other' (1962, 5). The extent to which parties are committed to gaining specific positions rather than exploring ways of satisfying their needs forms a disposition towards conflict. The ability to do otherwise---to focus on \emph{interests} rather than \emph{positions} (Fisher and Urry, 1981)---depends on how rigid the competing parties are in pursuing their objectives by specific means.

Here again, the repertoire of available actional responses depends on a bodily infrastructure. Many workers in the field of somatics have demonstrated that the tightness of sets of muscles is related to the inability to be open and flexible---cognitively, emotionally, and behaviourally. Whereas high reactivity to threats reflects how weak and vulnerable the self is, rigidity of habits indicates how strongly defended the self is.

John Dewey's teachings about human nature considered the matter of rigid habits a central issue in human experience. It was Dewey's lessons with renowned somatic teacher F. M. Alexander, who focused attention on the proper relaxed use of the body, that he said enabled him to hold a philosophical position calmly and to change it if new evidence came up warranting a change. Dewey contrasted this disposition with that of academic thinkers who adopt a position early on and then go on to use their intellects to defend it indefinitely (Jones, 1976, 97).

\subsection*{4. The body and moral righteousness}

Simmel early on identified the dynamic whereby conflicts become intensified the more they are separated from the personality of the parties to the conflict. His chief examples in that regard were conflicts carried out through legal procedures and conflicts on behalf of social causes. One can generalize Simmel's point by saying that conflicts become intensified whenever they become informed by normative directives. Although Simmel's cases were chosen to show how shifting the locus of conflict away from the personalities of the engaged parties works to heighten the intensity of a conflict, this may be seen just as well when applied intrapsychically. This is to say that once conflicts enlist the support of the superego, they will be driven by the same emotional energies that constitute the punitive forces of the ordinary superego. The statement by one presidential political campaigner---at first I just wanted to defeat my opponent, now I want to save the country from him---nicely illustrates the dynamic at work here. Both moral indignation and bigoted antagonisms of the sort analysed by Bettelheim and Janowitz (1950) represent striking exemplars of this syndrome.

Moral righteousness can be said to involve an infusion of bodily energies akin to the aggressive or counter-phobic instincts that drive the initiation of conflicts. One can almost visualize the way in which aggressive impulses intensify as the adrenaline flows and the blood boils on behalf of righteous indignation. This will escalate the conflict, although not necessarily make it more lethal. The infusion of so much agitation into the conflictual process might, as well-trained warriors and martial artists know, interfere with the optimal execution of a task and the actual execution of combat.

\subsection*{5a. The body and conflict-supportive values}

The activation of conflict depends not only on those elements that dispose parties to engage in conflict: high levels of aggressive impulsivity, reactivity, rigidity, and proclivity for moral indignation. It depends just as much on factors that work to inhibit the outbreak or continuance of conflictual interaction. These are primarily of two sorts, general values regarding conflict and the operation of social controls.

Cultural values can work either to instigate conflict or to suppress it. Conflict-supportive values appear in cultures where masculine aggressivity is particularly esteemed and promoted. This appears where the symbolism of warriorhood holds an esteemed place, as in archaic Greece and ancient Rome. In the cultures of Japan and Ethiopia, the values of warriorhood were so esteemed that they came to permeate the culture as a whole (Levine, 2002). The same is true of elements of Islamic tradition that idealize violence against those perceived as infidels or legitimate objects of external jihad. Masculine aggressivity is also valorized where considerations of proper recognition of the self are paramount, most famously in Mediterranean `honor and shame societies' (Giordano, 2005). Cultures that embrace masculine aggressivity provide socializing experiences that enhance combative bodily dispositions and abilities.

\section*{Somatic Elements That Reduce Conflict}

\subsection*{5b. The body and conflict-aversive values}

On the other hand, cultural values of harmony and peaceableness have been developed in most cultures. Where such values are dominant, as in particular institutional sectors like monasteries, or in societies reported to possess entirely pacific cultures, impulses to engage in conflict tend to be nipped in the bud if not entirely repressed.

Contemporary somatics supports the view that human bodies are actually designed to function in a loving, empowered way. Fear and anger weaken the body and therefore the whole self. Actions driven by feelings of fear and anger tend to create, escalate, and perpetuate conflict. As Paul Linden puts it, `Generally, disputes are carried out in a spirit of distrust, competitiveness, fear, and anger, which leads to escalation and the generation of new disputes. Conflict, as it is usually experienced, includes fear and anger. When people are afraid or angry, they lash out and try to hurt the people who make them feel afraid or angry' (Linden, 2003). To minimize such reactions, Linden prescribes a number of bodily practices, including a relaxed tongue and a soft belly, which he associates with the normative natural state for human bodies.

These views are supported by millennia of wisdom about the body cultivated in a number of Asian traditions. This hearkens back to ancient Hindu traditions starting with the \emph{Bhagavad Gita}, which described a state of human fulfillment brought about by a practice that calms the mind and the passions. This practice of unification---of `yoking,' or \emph{yoga}---of the body with the soul, the individual self with the universal spirit, involves a complex of methods that are physical as well as moral and mental. They include \emph{asana}, a discipline of holding postures, designed to exercise every muscle, nerve and gland in the body, and \emph{pranayama}, exercises in the rhythmic control of the breath. In similar ways the art of aikido, developed two millennia later in Japan, incorporates notions of unifying the entire bodily system through proper posture and of unifying the body with the mind by focusing one's attention on the bodily center of gravity. In the words of its founder, aikido `is the way of unifying the mind, body, and spirit' (Saotome, 1989, 33).

What does the image of the body conveyed by yoga and aikido imply about social conflict? When students of those disciplines stand or sit in the relaxed and centered postures cultivated in their practice, they experience calmness. From that experience they derive a conviction that there is no inherent, inexorable force driving human beings to aggress against one another. They also know that, compared to the state of calm enjoyment they experience, the act of committing aggression is unpleasant. When they sense an impulse to aggress proactively or reactively, they connect it with an immature or impure response, which can be overcome with training.

Yoga and aikido conceive the bodily harmony promoted by their teachings as a model of mature human functioning and accordingly view social conflict as a byproduct of inner discord. Yoga complements the state of inner harmony, which its physical and meditative disciplines aim at with various \emph{yama}, or ethical disciplines, by cultivating harmony with others. Closely related to this is the principle of \emph{abhaya}, freedom from fear: `Violence arises out of fear, weakness, ignorance or restlessness. To curb it most what is needed is freedom from fear' (Iyengar, 1973, 32). Similar ideas were articulated by the founder of aikido, Morihei Ueshiba. Although Ueshiba created his discipline as a \emph{budo}, a martial art, he came to insist that in his particular form of \emph{budo} `there are no enemies.' The only enemy consisted of the egoistic and aggressive strivings of the immature self, and the only victory worth pursuing was a victory over that immature self. For achieving this state, the powerful effects of a softened belly and an open heart have long been identified.

\subsection*{6. The body and social controls}

Sociologists have analysed a variety of mechanisms of social control that work to mute or dampen conflictual processes. These include binding arbitration; voluntary mediation; cooptation of antagonists; deflecting attention to symbols of higher allegiance; and dramatizing threats that transcend the partisan interests of the conflicting parties. Such mechanisms operate at the psychological and social levels, and would seem to admit little playroom for corporeal variables. Even so, one can ask: what psychosomatic processes inform the ways in which actors respond to intervening agencies?

That question in fact opens an enormous complex of possibilities. One process has to do with the degree of openness to arbitrating or mediating parties. This is the obverse of rigidity which, we saw above, demonstrably has a somatic basis. This openness is sometimes experienced as a relaxation of the visceral organs. Another process has to do with openings with new conflict-transcending social objects like larger communities or cultural objects such as values that enjoin conciliation and harmony. In the body, these are experienced as located in what have been called a mind-heart nexus.

\section*{Implications for General Theory}

The foregoing investigation opens up new lines for work in the general theory of conflict. The paradigm of generic conflict processes offers a framework with which to assemble contributions from various, normally disconnected, research traditions. Current advances in psychoneurophysiology and comparative ethology, for example, promise to enrich our understanding of anger, anxiety, and aggression a good deal. A generic conflict paradigm also enables us to develop a much more differentiated schema for analysing the onset, dynamics, and resolution of social conflict. It adds to the repertoire of existing conceptual tools such notions as hormonal levels, rigidity, reactivity, moral indignation---notions that come from different disciplines and that carry different sets of associations and supporting evidence. With that, it provides a framework with which to begin to consider more precisely somatic elements that pertain to conflict.\footnote{Future collaborative exploration by professionals in somatics and social psychology might well explore a hypothesis of organ specificity in this regard: the idea that even though all organismic responses are thought to involve the entire bodymind system, it may be possible to locate the physical seat of each in some part of the human body. Thus one might hypothesize that impulses of instinctual aggressivity are felt primarily in the visceral organs (`guts') and the shoulders; fear in the lungs and shoulders; rigidity in the throat, jaw, and neck (`stiff-necked'); moral anger in the head and the dorsal shoulders; receptiveness to masculine aggressive values in the upper chest.}

Returning to our point of departure, the material assembled above instantiates more general points that could provide a basis for revisiting the Parsonian legacy in a way that facilitates a more systematic analysis of the interfaces between the body and the other action systems of action. The conceptual link would be what I am calling the actional organism, defined, again, as `the subsystem of action where the organism's input of energies and the inputs from sources of meanings meet and interpenetrate.'

In a sense, this could be taken to mean something like returning to a modified version of the old instinct theories. Those were discarded because they were taken to represent hard-wired dispositions that propelled types of conduct no matter what. Incorporating this subsystem into the framework of action theory permits a clearer and more precise specification of interconnections.

Thus, within the cybernetic hierarchy, the actional organism is energized from below by the processes of the organic and inorganic systems. It energizes and receives direction, then, from the organized motives of the personality system; the organization systems of status-roles in the social system; and the organized symbolic complexes of the cultural system. Concrete action stands to be understood more completely by incorporating this set of abstractions that might now be more clearly identified and investigated.

\section*{References}

\begin{list}{}{\refstyle}
\item Bare, M. (2005) `Behavioural Organism, Behavioural System, and Body in the Social Theory of Talcott Parsons.' Unpublished M.A. thesis, Department of Sociology, University of Chicago.
\item Bettelheim, B. and Janowitz, M. (1950) \emph{Dynamics of Prejudice: A Psychological and Sociological Study of Veterans}. New York: Harper.
\item Boulding, K.E. (1962/1988) \emph{Conflict and Defense: A General Theory}. Lanham, Md.: Univ. Press of America.
\item Coleman, J.S. (1957) \emph{Community Conflict}. New York: The Free Press.
\item Collins, R. (1975) \emph{Conflict Sociology: Toward an Explanatory Science}. New York: Academic Press.
\item Coser, L. (1956) \emph{The Functions of Social Conflict}. Glencoe, Ill.: Free Press.
\item Feldenkrais, M. (1949) \emph{Body and Mature Behaviour}. Tel-Aviv: Alef.
\item Fisher, R. \& Urry, W. (1981) \emph{Getting to Yes: Negotiating Agreement without Giving In}. Boston: Houghton Mifflin.
\item Freud, S. (1932/1939) `Letter to Albert Einstein.' In \emph{Civilization, War and Death: Psycho-Analytical Epitomes}, No. 4, ed. John Rickman. London: Hogarth Press.
\item Fromm, E. (1973) \emph{The Anatomy of Human Destructiveness}. New York, NY: Holt, Rinehart and Winston.
\item Gardner, H. (1983/1993) \emph{Frames of Mind: The Theory of Multiple Intelligences}. New York: Basic.
\item Gilligan, J. (1996) \emph{Violence: Reflections on a National Epidemic}. New York: Vintage.
\item Giordano, C. (2005) `Mediterranean Honor and beyond. The social management of reputation in the public sphere.' \emph{Sociologija Mintis ir Veiksmas}. 01/2005.
\item Iyengar, B.K. (1973) \emph{Light on Yoga}. New York: Schocken Books.
\item James, W. (1910/1974) `The Moral Equivalent of War', in \emph{Essays on Faith and Morals}, ed. R.B. Perry, 311--28. New York: Longmans, Green.
\item Jones, F.P. (1979) \emph{Body Awareness in Action}. New York: Schocken Books.
\item Kriesberg, L. (1973) \emph{The Sociology of Social Conflicts}. Englewood Cliffs, N.J.: Prentice-Hall.
\item Levine, D.N. (1994) `Social Conflict, Aggression, and the Body in Euro-American and Asian Social Thought.' \emph{International Journal of Group Tensions} 24: 205--17.
\item \_\_\_\_\_\_\_\_\_\_\_ (2002) `The Masculinity Ethic and the Spirit of Warriorhood in Ethiopian and Japanese Cultures', Paper presented at the World Congress of Sociology, July.
\item \_\_\_\_\_\_\_\_\_\_\_ (2006) \emph{Powers of the Mind: The Reinvention of Liberal Learning}. Chicago: University of Chicago Press.
\item Linden, P. (2003) \emph{Reach Out: Body Awareness Training for Peacemaking---Five Easy Lessons} [online]. CCMS Publications. Available from World Wide Web: (www.being-in-movement.com).
\item Lidz, C.W. and Lidz, M.L. (1976) `Piaget's Psychology of Intelligence and the Theory of Action.' Ch.8 in \emph{Explorations in General Theory in Social Science}. New York: Macmillan.
\item Lorenz, K. (1966) \emph{On Aggression}. New York: Harcourt, Brace and World.
\item Mead, M. (1937) \emph{Cooperation and Conflict among Primitive Peoples}. New York, NY: McGraw-Hill.
\item Morgenthau, H. (1960) \emph{Politics Among Nations}, 3rd edition. New York: Knopf.
\item Parsons, T. (1951/1964) \emph{The Social System}. New York: Free Press.
\item Parsons, T. \& Shils, E. (eds) (1951) \emph{Toward a General Theory of Action}. Cambridge, MA: Harvard.
\item Saotome, M. (1989) \emph{The Principles of Aikido}. Boston: Shambhala Publications, Inc.
\item Scheff, T.J. \& Retzinger, S.M. (1991) \emph{Emotions and Violence: Shame and Rage in Destructive Conflicts}. Lexington, MA: Lexington Books.
\item Scheler, M. (1928/1961) \emph{Man's Place in Nature}. New York: Noonday Press.
\item Schelling, T.C. (1960) \emph{The Strategy of Conflict}. Cambridge, Mass: Harvard University Press.
\item Shilling, C. (2005) \emph{The Body in Culture, Technology, \& Society}. London: Sage.
\item Smith, T.S. (1993) `Violence as a Disintegration Product: Counterphobic Reenactments of Dissociated Traumatic Events in Individual and Group Life.' Institut International de Soziologie, Paris.
\item Tinbergen, N. (1968) `On War and Peace in Animals and Man: An Ethologist's Approach to the Biology of Aggression', \emph{Science} 160, 1411--18.
\item Wrangham, R. \& Peterson, D. (1996) \emph{Demonic Males: Apes and the Origins of Human Violence}. Boston: Houghton Mifflin.
\end{list}