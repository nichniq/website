\chapter[Social Conflict, Aggression, and the Body (1994)]{Social Conflict, Aggression, and the Body in Euro-American and Asian Social Thought\footnote{``Social Conflict, Aggression, and the Body in Euro-American and Asian Social Thought,'' \emph{International Journal of Group Tensions}, vol. 24, no. 3: 205-17.}}

\section*{Abstract}

\begin{quote}
\small
Philosophical perspectives on social conflict in Western social thought comprise four general positions, formulable by cross-classifying two variables: (1) is conflict viewed as inexorable or contingent, and (2) is conflict viewed primarily as a negative or a positive phenomenon? A ``pessimist'' views conflict as negative but inexorable. An ``optimist'' holds that conflict is inevitable but positive. A ``prudential'' position views conflict as contingent and entirely negative. Finally, a ``provocative'' view holds that conflict is a definite positive that needs to be promoted.

These positions can be linked with assumptions about the bodily bases of human aggression. Views of conflict as inexorable regard the body as a source of egoistic impulses that well up and initiate aggressive behaviors. Views of conflict as contingent regard the body as a source of flight or fear. A variant of the prudential position sees the body as a source of malleable plastic energies. In contrast, certain Asian traditions imagine a body that is neither at the mercy of aggressive instincts, nor a scene of conflicting drives, nor utterly lacking in natural structure. In particular, the traditions of yoga in India and of aikido in Japan depict the body as disposed to a state of calm and serenity through becoming unified with the mind and spirit. In the aikido view, conflict need not be the outcome of aggression, since the response to attacks can be neutralization rather than counterattack or submission. To reduce conflict, this prudential view relies, not on external social arrangements, but on internal practices that calm the mind and promote harmony within oneself and with others.
\end{quote}

The theory of social conflict includes a number of more or less consensually validated propositions about the causes, forms, levels, dynamics, resolution, and consequences of interpersonal and intergroup conflict. Regarding philosophical presuppositions about conflict, however, strong differences persist despite agreement on the more empirically ascertainable aspects of conflictual phenomena. I propose here to articulate some of these differences. I shall do so by constructing four ideal types, which I designate as pessimistic, optimistic, prudential, and provocative perspectives on conflict.\footnote{Calling these constructions ideal types signals my intent to present the perspectives in simplified form so as to clarify the issues. In particular, I note two egregious simplifications: the paper does not make stable distinctions between conflict and such overlapping terms as antagonism, competition, and combat; and in maintaining an opposition between views of conflict as mainly positive or negative, it runs the risk of appearing to support what Boulding rightly describes as ``the illusion \ldots{} that conflict in any amount is either bad or good in itself'' (1988, 305).} After discussing the defining features of each perspective and some of its eminent representatives, I shall analyze how these positions relate to assumptions about the natural human body. That will lead to an opening through which certain ideas developed in Asian thought could be included in the discourse about conflict, with the consequence of inviting us to take a look at the entire subject in fresh ways. 

\section*{Social Conflict As Inexorable}

What I am calling a \emph{pessimistic} perspective on social conflict has deep roots in Christian theology. Humans are essentially sinful creatures, disposed to aggress against their neighbors. The wages of this sinfulness are misery and suffering, which is the human lot on earth. Immanuel Kant presents a secular version of this view. Kant finds the disposition to engage in conflict ever-present and inherently immoral. From the day of birth human egoism advances unrestrained. Humans expect opposition on all sides because they know from within that they are inclined to oppose all others. In consequence, the tableau of human history is woven from childish vanity, malice, and destructiveness.\footnote{To be sure, Kant overlaid this pessimistic diagnosis of the human condition with a secular version of Providence that found in man's ``unsocial sociability'' the dynamic that leads to civil order and eventually a world state.}

The tenets of a Kantian philosophical anthropology have found their way into modern social science through research traditions in psychology, ethology, and political science. Psychoanalytic psychology, despite vicissitudes of thought regarding the instincts, has tended to assume both an inherent human disposition to aggression that leads to conflict, and inexhaustible reservoirs of intrapersonal conflicts that spill over, via externalization and projection, into interpersonal conflicts. Freud held that violent conflict was endemic to human experience, as a means to resolve conflicts of interest and as an expression of an instinctive craving----an ``active instinct for hatred and destruction.'' He bemoaned the destructiveness of modern warfare but held little hope that cultured aversions to war could overcome the aggressive dispositions so deeply rooted in man's biological makeup ([1932] 1939). Freud theorized about this by positing a self-destructive ``death instinct'' which gets turned away from the self toward others to produce a constant fund of conflictual energies. Although most psychoanalysts rejected Freud's assumption of a death instinct, they substituted a destructive instinct for the polar opposite of the sexual instinct, which let them incorporate Freud's pessimistic views on aggression without having to subscribe to what they considered a far-fetched metapsychological construct. 

The ethologist Nikolaas Tinbergen likewise posits a universal proclivity to intraspecific conflict based on genetically transmitted instincts. Comparing human aggression with aggression in other animals, however, he finds human aggressiveness distinguished by the fact that it is socially disruptive: ``Man is the only species that is a mass murderer, the only misfit in his own society'' (1968, 180). This condition comes from a combination of instinctual, cultural, and technological factors. Whereas in other species and earlier human periods the impulse to fight got balanced by the fear response, humans have contrived cultural conditions that dampen the impulse to flee from battle, while the technology of fighting at a distance eliminates the taming effect of personal contact in face-to-face encounters. Dismayed about these seemingly ineradicable dispositions which threaten to convulse modern society with destructive warfare, Tinbergen acknowledges the impact of increased population density on the impulse to fight and pessimistically admits that the internal urge to engage in combat will be difficult if not impossible to eliminate. A similar diagnosis was made a half-century earlier by William James. Despite the acknowledged horrors of modern warfare, James wrote on the eve of World War I, modern people have inherited a pugnacious disposition and a love of glory that inexorably feed combat: ``Our ancestors have bred pugnacity into our bone and marrow, and thousands of years of peace won't breed it out of us'' ([1910] 1939, 314). 

Political scientists who espouse a position of ``political realism'' express a comparably pessimistic position. Long an eminent spokesman for this position, Hans \mbox{Morgenthau} holds that the social world results from forces inherent in human nature which makes it ``inherently a world of opposing interests and of conflict among them'' (1960, 4). These conflicts are inexorable, and \mbox{Morgenthau} sees no need to glamorize them or consider them benign. Indeed, he cautions social scientists to take care not to mistake the policy prescriptions that follow from the perspective as \emph{moral}. \mbox{Morgenthau} thinks it important to uphold morality as a set of ideals, but urges social scientists and policy-makers to understand that reality consists of conflicts of interests that can neither be understood nor practically mediated from a moral point of view.

What I call an \emph{optimistic} position draws on a philosophic outlook in which conflict figures as an inexorable yet essential source of human well being. Its proponents hail the Heraclitean dictum that ``war is the father of all and king of all.'' Heraclitus chided those who dreamed of eliminating strife from among gods and men. Things exist only insofar as they embody a tension between opposites, and human goods come into being only through strife.

Among ethologists, Konrad Lorenz has been a prominent advocate of viewing conflict as inexorable but basically positive. Conflict has provided such adaptive advantages as balancing the ecological distribution of members of the same species, selection of the fittest specimens through fights among rivals, mediating the ranking orders need for complex organizations, and instigating ceremonies that promote social bonding. Aggression, he argues, ``far from being the diabolical, destructive principle that classical psychoanalysis makes it out to be, is really an essential part of the life-preserving organization of instincts'' (1966, 48). If not war, then at least conflict should be called the father of all things. Conflict between independent sources of impulse can produce tensions that lend firmness to systems, much as the stays of a mast give it stability by pulling in opposed directions (95).

The optimistic position was developed in classic sociology through the seminal work of Georg Simmel (1903/4; [1908] 1955). Simmel saw conflict not just as an inexorable feature of human social life but also as a process with essentially benign consequences. That is, Simmel conceptualized conflict as an essential constitutive feature of social structure. This is because antagonisms maintain distances essential to stable social structures. It is also because the expression of conflict preserves association among parties who might otherwise sever relations. Simmel suggested that mutual aversions are indispensable ingredients both of small intimate groups which involve numerous vital relations among their members and of large concentrations of people in modern metropolises. The capacity to accommodate conflict he considered to be a sign of the vitality of intimate relationships.

Simmel's classic analysis was recovered half a century later by Lewis Coser. In \emph{The Functions of Social Conflict} (1956) Coser refined Simmel's ideas by casting them in the form of discrete, clearly formulated propositions; comparing them with relevant materials from psychoanalysis, psychology, and social psychology; and showing how they could be qualified by the interposition of intervening variables. Although Coser argued that intragroup and intergroup conflicts promote social unification only under specified circumstances, he also identified ways in which the expression of conflictual sentiments enhances the effectiveness and long-term stability of groups.

\section*{Social Conflict As Contingent}

For all their differences, the pessimistic and the optimistic perspectives share the assumption that social conflict is universal and inexorable. A different perspective appears in authors who consider social conflict to be something that can be avoided or minimized. Among such authors, one group regards conflict as essentially negative in its nature or consequences. These authors therefore hold that social conflict can and should be kept under check or prevented through appropriate social interventions. I call this a \emph{prudential} perspective, with two main variants---one represented classically by Thomas Hobbes, the other by cultural psychologists like Margaret Mead and Erich Fromm. 

The Hobbesian perspective presumes that the pursuit of personal interests sooner or later disposes all human actors to engage in social conflict. This stems both from the promptings of pride and from the need to acquire power to defend one's goods against others. Unrestrained social conflict produces a condition he famously described as the ``war of every man against every man,'' in which people live in chronic fear and misery. To counter this ever-present possibility, fearful humans institute sovereign authorities. In exchange for the protection against anarchy and civil strife afforded by those authorities, citizens transfer their rights to self-defense. More generally, a Hobbesian perspective sees conflict as always latent but actually contingent. It can and should be forestalled through the establishment of appropriate governing authorities.

A variant of this perspective appears in authors who hold that the disposition for conflict does not inhere in the human condition, but results rather from how persons are brought up and how their relations are conventionally organized. Margaret Mead (1937) was perhaps the first cultural anthropologist to examine this variable across many cultures. She found that primitive societies range from highly competitive to highly cooperative ones, and that the main determinant of whether people behave in a competitive or a cooperative manner was the cultural conditioning which they experienced. Erich Fromm (1973) pursued the issue more intensively, examining thirty primitive societies from the standpoint of aggressiveness versus peacefulness. Fromm found several---like the Aztecs, the Dobu, and the Ganda---who evince a great deal of interpersonal aggression and violence, both within the tribe and against others. The atmosphere of life within those societies is truly Hobbesian, a condition of constant fear and tension. On the other hand, Fromm found a number of primitive societies where precisely the opposite qualities manifest themselves. Among the Zuni Pueblo Indians, the Mountain Arapesh, and the Mbutu, for example, he found little hostility and violence, virtually no warfare, hardly any crime, little envy and exploitation, and a generally cooperative and friendly attitude. Fromm goes on to analyze the specific social conditions that tend to generate aggressive responses, both of the biologically adaptive sort he calls defensive aggression and the nonadaptive, purely destructive sorts he calls malignant aggression. Psychologists from the behaviorist tradition, like Watson and Skinner, likewise view conflict as contingent. Since aggression represents a response to frustrating experiences and the reinforcement of aggressive behavior patterns, it can be curbed through the proper reinforcement of nonaggressive dispositions. Whatever the disciplinary orientation, this variant of the prudential perspective views much if not all conflict as eradicable through practices which dispose a largely if not entirely plastic human nature to live in accord with nonconflictual patterns. 

Quite the reverse of the prudential perspective is an outlook that advocates social interventions not to eliminate conflict but to stimulate it---what I am calling a \emph{provocative} perspective. Its most extreme versions appear in writers who extol the virtues of war and berate their contemporaries for not being sufficiently martial. Nietzsche's Zarathustra asks: ``You say that it is the good cause that hallows even war?'' and comments: ``I say unto you: it is the good war that hallows any cause''---yet Nietzsche viewed the general run of mankind in his time as objectionably timid. Among social thinkers this stance appears classically in George Sorel's \emph{R\'{e}flexions sur la violence}. Although Sorel proceeded from a radical socialist perspective, his arguments are generalizable and did in fact become utilized by spokesmen for a wide spectrum of ideological positions. Sorel advocates a view of combat that highlights its noble side in the way that poets have eulogized illustrious armies. The whole of classical history, he argues, was dominated by the idea of war conceived heroically. This idea celebrates the profession of arms as an elite vocation, reflecting the opportunities that great battles afford for submitting to tests of strength and for appealing to the sentiment of glory. Voluntary participation in war and the myths associated with such combat provide the inspiration for the loftiest moral convictions. 

Sorelian ideas found their way into 20th-century apologia both for colonial expansion and for anti-colonial violence. Benito Mussolini cited Sorel's forefather, Proudhon, to claim a ``divine origin'' for war. Everlasting peace would be depressing and destructive of man's basic virtues: pacifism represents cowardice before sacrifice. Fascism thus rejects all international structures designed to ensure peace, despite their having possibly been accepted temporarily for opportunistic reasons. War alone, Mussolini declaimed, ``carries to the maximum of tension all human energies and stamps with a seal of nobility the peoples which have the virtue of facing it. All other tests are substitutes which never put man in front of himself'' (Borgese 1938, 392, 346f.). 

Writing on the other side of the imperialist divide, psychiatrist Frantz Fanon invokes overtones of Sorelian combat against capitalist oppression to proclaim the ennobling effects of participation in violent struggle against colonial domination. Fanon sees liberation to be possible only after a ``murderous and decisive struggle between the two protagonists.'' He criticizes social forms that permit violence to be averted, either by channeling legitimate combative energies into outlets such as dance, spirit possession, or self-destructive symptoms where they are dissipated; or else by defusing them through anti-polemical ideational forms like religion, philosophies of human rights, ethics of non-violence, or a politics of compromise. Nonviolent forms of political opposition---work stoppages in a few industries, mass demonstrations, boycotting of buses or imported commodities---simply represent other forms of action that let people work off their energy and so constitute a kind of ``therapy by hibernation.'' Violent combat alone can liquidate colonialism, regionalism, and tribalism, and thereby introduce into common consciousness the ideas of a common cause, national destiny, and collective history. At the level of individual personality, ``violence is a cleansing force. It frees the native from his inferiority complex and from his despair and inaction; it makes him fearless and restores his self-respect'' (1968, 37, 66, 94).

However, the provocative perspective on conflict need not be tied to an espousal of physical violence. It can and has been expressed by those who advocate an increase in verbal forms of conflict as a means of promoting social change or as the preferred means of arriving at the truth. Herbert Marcuse helped persuade a generation of intellectuals to follow an ethic of negation on grounds that harmony of opinion was counter-emancipatory. Wayne Booth has described a polemicist position among literary critics that holds that ``the more vigorous the conflict, the healthier the body critical'' (1979, 4). Such a position appears among those who promote conflict as the best way to approach truth, an epistemological stance that Walter Watson (1985) designates as the agonistic method. Watson cites Machiavelli as one who applies the agonistic method to politics in arguing that the opposition of conflicting parties is needed to preserve liberty.

\section*{Bodily Bases of Aggression and Nonaggression}

Like much sociological discourse, conflict theory can become highly abstract. Yet its intimate connection with the realities of physical combat, by metaphor when not literally, makes it easy to relate the discussion of social conflict to the interaction of physical bodies. And the tendency to adduce biologically rooted dispositions for the presence or absence of conflict invites us to consider how differing perspectives on conflict might be related to differing assumptions about the human body.

The pessimistic perspective tends to view the human body as a continuously bubbling cauldron of egoistic and aggressive impulses that sooner or later spill over into combative action. The bodily imagery that underlies this view has been depicted most vividly in classical psychoanalysis. Freud saw the human organism as a perpetually renewed source of instinctual energies that well up and produce inner discomfort until they get released. Psychic and somatic symptoms reflect failures in the personality's ability to release those instinctual tensions, which eventually find release through indirect channels. In one way or another, directly or indirectly, human aggression represents a constantly flowing impulse that emanates from the human body such that humans can never escape the proclivity to destroy either themselves or others. 

Although Lorenz took a more positive view of conflict, he too espoused a mechanistic-hydraulic view of aggression. Lorenz likens aggression to a gas constantly being pumped into a container or to a liquid in a reservoir dischargeable through a spring-loaded valve at the bottom. In Lorenz's conception, energies specific for an instinctive act accumulate continuously in neural centers for that behavior, leading animals and humans to hunt for stimuli in order to trigger the release of those energies. Although Simmel downplayed the salience of instinctive aggressive energies as a source of conflict, he considered the mobilization of such energies useful for the prosecution of conflicts once they get started on the basis of conflicting interests. Even so, Simmel admits the existence of a pure hostility drive which manifests itself in the institution of combative games.

Insofar as they entertain considerations of the bodily sources of aggression and conflict, then, those who think of conflict as inexorable tend to see the body as a mechanism that regularly produces aggressive energies. Authors who regard conflict as contingent have a different set of images: either they see the body as producing other impulses that swamp the aggressive instincts, or they look at aggressive behavior altogether as not instinctually based.

Hobbes represents the former alternative. The perpetual and restless desire of power after power to which all men are inclined would lead inexorably to constant civil strife were it not for the activation of an even stronger natural inclination: the wish to avoid violent death. Humans are also motivated by a wish to live comfortably by means of conveniences which only a regime of peace can procure. So the impulse to aggress against others gets subordinated to a wish for peaceful coexistence, a condition procured by establishing a sovereign political authority. The logic of Hobbes's argument can be modified to cover a variety of social arrangements designed to prevent conflict, but his logic regarding the bodily bases of action can be left intact: the body is the home of divergent impulses including aggressiveness, but aggression can get inhibited by other propensities that support institutions designed to prevent conflict. This image of the body is not unlike what we find in writers like Nietzsche and Sorel. The latter visualize a natural human disposition to be fierce and combative, a disposition that (for them, unhappily) gets swamped by fear and desires for convenience, thereby deflecting martial impulses into innocuous channels. 

A third view of the body appears in authors who reject instinctual determinisms of any sort. The model here presents an organism whose genetic programming is so minimal that it extends only to general response capacities. Without cultural patterns to give some particular shape to human lives, ``man's behavior would be virtually ungovernable, a mere chaos of pointless acts \ldots{} his experience virtually shapeless'' (Geertz 1973, 46). Margaret Mead first applied this credo of the cultural anthropologist to the variable of conflict versus cooperation. Bodily dispositions to engage in combat reflect the internalization of symbols and the cultivation of habits promoted by combative cultures, but pacific cultures can just as successfully create nonaggressive dispositions.

\section*{Some Asian Views of the Body, Aggression, and Conflict}

Although disciplines concerned with bodily healing have recently started to examine what ``non-Western'' arts might contribute, it is rare that Euro-American social science has an opportunity to draw on the insights and understandings of other traditions. Yet it may be the case that certain Asian traditions afford ways of thinking about conflict that are hard to encompass within available Euro-American paradigms, and that the most direct entree into those traditions might come from looking at their distinctive views of the body and aggression. In what follows I shall discuss the traditions of yoga in India and aikido in Japan, although comparable ideas may also be found in certain aspects of the lore of Taoism in China and of the Korean tradition of hwarangdo. 

The general thesis I wish to advance is that these traditions imagine a body that is \emph{neither at the mercy of aggressive instincts, nor a scene of conflicting drives, nor utterly lacking in natural structure}. Rather, the state of being battered about by desires, whether shaped or chaotic, represents human nature only in an immature state. Mature humanity exhibits a body that is unified internally and unified with the mind, a being living in inner harmony and with little inclination to aggress against others. 

Two thousand years ago the Sanskrit classic \emph{Bhagavad Gita} represented a state of human joy and fulfillment brought about by a practice that calms the mind and the passions. This practice of unification---of ``yoking,'' or \emph{yoga}---the body with the soul, the individual self with the universal spirit---involves a complex of methods that are not only moral and meditative but physical as well. They include \emph{asana}, a discipline of holding carefully designed postures, and \emph{pranayama}, exercises in the rhythmic control of the breath. These are not extraordinary practices, the privilege of an exceptional elite or of superhuman creatures, but are available to anyone willing to work hard at them. Exercising every muscle, nerve and gland in the body, the asanas secure a fine physique, one that is energized, limber, and strong yet not muscle-bound. They are designed to produce a state of superb bodily health, understood as a state of complete equilibrium of body, mind, and spirit. 

A millennium-and-a-half after the principles of yoga were classically codified in a book of aphorisms by Patanjali, another Asian discipline was developed which holds a similar view of the human potential for living with a harmonious body-mind. The art of aikido, developed by the martial artist/religionist Morihei Ueshiba in the 1930s and 1940s, draws on a combination of Asian disciplines, including neo-Confucianism and Shinto as well as \emph{budo} (Japanese: martial ways). Foundational to this art are the notions of unifying the entire bodily system through proper posture and of unifying the body with the mind through focusing one's attention on the bodily center of gravity. The movements that adepts learn for responding to physical attacks require the body-mind system to be centered in this way, and certain exercises have been designed to enhance body-mind harmony. In the words of its founder, aikido ``is the way of unifying the mind, body, and spirit'' (Saotome 1989, 33).

What does the image of the body conveyed by yoga and aikido imply about social conflict? When students of those disciplines stand or sit in the relaxed and centered postures cultivated in their practice, they experience a state of calmness. From that experience they derive a conviction that there is no inherent, inexorable force driving all human beings to aggress against one another. They also know that, compared to the state of calm enjoyment they experience, the act of committing aggression is unpleasant---even when one commits aggressive acts in self-defense. When they sense an impulse to aggress proactively or reactively, they connect it with an immature response which can readily be overcome. So the bodily states experienced in yoga or aikido practice support a belief that conflict is neither inexorable nor desirable, which aligns them with proponents of what I have called the prudential perspective.

In contrast to the Hobbesian version of that perspective, however, they do not make refraining from aggression dependent on fear. The body in the relaxed and unified state experiences anxiety as little as it does aggression. Nor do they presume, as do cultural anthropologists, that only in a specially designed culture is it possible for an infinitely plastic human nature to be molded in nonaggressive directions. The body in the relaxed and unified state experiences itself as unaggressive, whatever cultural patterns may prescribe. 

Yoga and aikido conceive the bodily harmony promoted by their teachings as a model of mature human functioning and thus a model for right living. They also connect it with teachings about interpersonal conflict. They see such conflict as a byproduct of inner discord and thus neither inexorable nor necessary for the good human life. Yoga complements the state of inner harmony which its physical and meditative disciplines aim at with various \emph{yama}, or ethical disciplines, that cultivate harmony with others. These include the commandment of \emph{ahimsa} or non-violence. \emph{Ahimsa} is an injunction to show respect to all living creatures. Closely related to this is the principle of \emph{abhaya}, freedom from fear. As a distinguished contemporary yogi puts it, ``Violence arises out of fear, weakness, ignorance or restlessness. To curb it most what is needed is freedom from fear'' (Iyengar 1979, 32). Far from basing understanding of social life on a presumption of ineradicable instincts of aggressiveness and fear, this strand of classic Hindu thought evolved a conception of healthy human functioning in which both fear and combativeness could be avoided.

The preeminent application of yogic principles to contemporary social thought about conflict was the work of Mahatma Gandhi. Gandhi embraced certain well-known notions of the yogic tradition, including \emph{ahimsa} and \emph{satya} (truth),\footnote{Gandhi came to call the technique of political action he devised \emph{satyagraha}, the force that is born of truth. He defended its commitment to nonviolence on grounds that truth is absolute, equivalent to God, and ``man is not capable of knowing the absolute truth and therefore not competent to punish'' (Bondurant 1988, 16).} and reworked them into an approach to conflict based on refusal to respond to aggression with counter aggression. Following the yogic philosophy Gandhi insisted that it is possible---and finally more effective---to oppose the evil in the wrong-doer without opposing the wrong-doer. 

Similar ideas were articulated by the founder of aikido, Morihei Ueshiba. Although Ueshiba created his discipline as a \emph{budo}, a martial art, he came to insist that in his particular form of \emph{budo} ``there are no enemies.'' The only enemy consisted of the egoistic and aggressive strivings of the immature self, and the only victory worth pursuing was a victory over that immature self. Ueshiba described the goal of his \emph{budo} as a kind of \emph{ahimsa}, a spirit of loving protection of all living creatures. He dedicated his art to the ideal of promoting peace and harmony throughout the entire world community. 

This does not mean that aikido presumes a world wholly free of aggression. Aikido teachings do presume that from time to time some people will, wittingly or not, attack other persons or intrude into their space, physically or verbally. But aikido also assumes that the options for response are not restricted to those motivated by the impulses to fight back, to take to flight, or to submit obsequiously and so plant seeds for resentment and later conflict. The aikido position presents a fourth option, that of neutralizing the aggression of the attacker so that conflict can be avoided. The person or group attacked can respond in an aiki way by blending with the energy of the attacker, remaining centered, and redirecting that energy in a way that protects the victim but respects the attacker. 

Yoga, satyagraha, and aikido introduce a new position into the inventory of perspectives on conflict developed in Euro-American social thought. Like the other prudential perspectives, they argue that conflict is not good, because human life does not fulfill itself through discord: assaulting others bespeaks an expression of the immature self and disrespect for the truth that each person represents---not to mention the horrors brought about by warfare in this century. The virtues of courage, self-respect, and enlarged truth espoused by the supporters of conflict can be attained---indeed, attained more effectively---through modes of assertiveness that do not entail aggression against others. 

In contrast to the two other variants of the prudential position which I have sketched, the Asian approaches discussed here do not look to external institutions to curb conflict. To be sure, they would not repudiate formal political arrangements as espoused by Hobbes and others, or the effects of benign cultural conditioning as espoused by cultural anthropologists. Their primary emphasis, however, is on internal practices that calm the mind and unify body, mind, and spirit. Such practices promote a naturally-based harmony that energizes nonconflictual interactions and gets fortified by doctrines supportive of respectful relations with others. Perhaps contemporary discourse about social conflict might benefit from pondering the implications of this piece of Asian social thought.