\chapter[Extending the Way (2009)]{Extending the Way: Aikido for the 21\textsuperscript{st} Century}

``I did not invent aikido, I discovered it.'' So, we believe, said the Teacher who founded aikido, Morihei Ueshiba. He thought he had discovered a system of practices that was in deep accord with the fundamental energy processes of the universe---an aspect of his thought expounded by one of his most intimate, late deshis, Shihan Mitsugi Saotome in the brilliant book, \emph{Aikido and the Harmony of Nature} (1986). 

O'Sensei's sense that aikido is a form to be \emph{discovered} lends poignancy to the fact that his own understanding and practice of aikido continued to evolve throughout his life. It began as \emph{aikibujitsu}, his own polished version of the system of martial techniques developed since medieval times and transmitted through his own mentor Sokaku Takeda Sensei. It continued through his conversion of that system to \emph{aikibudo}, which he taught in the 1930s: a system of training in powerful techniques for vanquishing opponents but whose practice was geared to ennobling the character of the practitioner. His teaching of this system continued through 1941, the year that Japan's war against the United States began. It was in that year, writes Gozo Shioda in \emph{Aikido Shugyo} (1991), that O'Sensei turned toward a more spiritual path of development. Shioda Sensei notes that he did not follow O'Sensei's teachings further at that point, and that therefore, with perhaps some hyperbole, he claimed to be the last of O'Sensei's students to be trained as a martial artist: ``The concept of Aikido as a martial skill has ended with me'' (204). 

During the years of inner exile at Iwama, Ueshiba's system, which in 1941 he named aikido, continued to evolve. Its movements came to be inspired increasingly by the principle of attunement between partners. According to recollections of Saotome Sensei, Ueshiba's emphasis on interhuman harmony increased enormously due to two events that occurred in 1945, the atomic bombing of Hiroshima and Nagasaki, and the reports from Japanese acquaintances who had been present at the liberation of the Nazi concentration camps. Ueshiba became determined to develop a teaching whose emphasis was altogether contrasted with that of defeating an enemy---``in my aikido there are no enemies,'' he maintained. 

This change was experienced by Hikisutchi sensei when they reunited for the first time after the war in 1948, as he recounts in the video, ``The Birth of Aikido.'' ``Help me establish a wholly new approach to budo,'' O'Sensei pleaded to Hikisutchi at an emotional meeting, ``we must expound and promote a budo that is dedicated to the creation of peace.'' As Motomichi Anno Sensei, Hikisutchi's main deshi and successor at Shingu, recalled in an interview in 1999: ``According to O'Sensei, \emph{bu} is no longer a matter of fighting; \emph{budo} exists for the purpose of developing good relations among all people. \ldots{} As I listened to O'Sensei's teaching, it seemed that Aikido was something unprecedented, that O'Sensei had newly created out of his training in various classical Japanese martial arts [and] that Aikido has evolved beyond \emph{budo}.'' And Anno Sensei took the next logical step, affirming that the state of being that O'Sensei sought to cultivate through aikido could be achieved by devoted practice of a number of arts, including calligraphy, flower arranging, tea ceremony, and music. A pithy dictum of O'Sensei's makes this point: ``\emph{Aiki waza michi shirube}, Aikido training is but a signpost to the Way.''

Many of those who went on to teach aikido continued to teach it as a set of techniques for vanquishing the other, despite the account of aikido as a spiritual path set forth so eloquently in \emph{The Spirit of Aikido} (1984), by Ueshiba's son Kisshomaru Ueshiba, the late Doshu. Nevertheless, three of his deshis, in particular, began to explore pointedly the implications of seeing aikido essentially as a Way, designed to promote harmony in the world.

One of these was Koichi Tohei sensei, O'Sensei's preeminent student after World War II. Tohei's earlier studies with yoga teacher Tempu Nakamura, founder of a practice called Shin Shin Toitsu Do (Way of Mind and Body Coordination), equipped him to enrich aikido pedagogy with practices directly aimed at calming the mind and enhancing the flow of ki. While still the principal instructor at the Hombu dojo he set forth teachings on how to extend these practices beyond the mat in a book first printed in English in 1966, \emph{Aikido in Daily Life}. The organization Tohei later founded to promote these teachings directly, the Ki Society, took as its central motto: ``Let us have a universal spirit that loves and protects all creation and helps all things grow and develop.''

Tohei Sensei was the first person to introduce the teaching of aikido into the Untied States, where his influence was profound and extensive, such that the American reception of aikido proved from the outset resonant with the notion that aikido had some palpable connection with daily life. Other sources of ideas for extending aikido's teaching into daily life came from two talented Americans who studied with O'Sensei in the 1960s, Terry Dobson and Robert Nadeau. 

The only American to be an \emph{uchi deshi} student with the Founder, Terry Dobson told his junior colleague James Lee that ``O'Sensei's mission for him was to spread Aikido around the world and show people how it could be used to create peace in the world.'' Accordingly, he developed a range of materials for workshops on conflict management and personal growth. Dobson's first effort, \emph{Giving In to Get Your Way}, co-authored with Victor Miller, was published in 1978, and posthumously in 1993, with a new title: \emph{Aikido in Everyday Life}. The book encouraged people to engage in conflict and to respond to life's inexorable conflicts in ways that avoid fighting back, withdrawal, inaction, and deception in favor of confluent engagement. He continued to grapple with these issues, and prior to his untimely death had worked out the outline of a sequel, to be titled \emph{Soft Power: The Resolution of Interpersonal Conflict}. The book would have included centering exercises devised by Koichi Tohei and supplemented by several of Terry's own invention. He envisioned it as a unification of aikido ``with the academic discipline of interpersonal communication,'' wherein the verbal counterparts of aikido responses were realized through a number of ``verbal forms.'' Retrieved by James Lee, these verbal forms are explained in detail and examples given in \emph{Restoring Harmony: A Guide for Managing Conflict in Schools} (Lee, Pulvino, and Perrone, 1998). 

The other principal conduit for O'Sensei's idea of aikido as a vehicle for spiritual energy was Bob Nadeau. Nadeau's teachings ignited an enormous amount of creativity in the extension of aiki ways off the mat. At least five of his students went on to inspire countless others with fresh manifestations of extension work: George Leonard, who developed a systematic form of energy training he calls LET (Leonard Energy Training); Paul Linden, who created a healing modality known as Being in Movement\textregistered; Richard Moon, who focused on powers of empathy through his Listening Institute; Wendy Palmer, who created Conscious Embodiment, a system of practices designed to enhance inner awareness; and Richard Strozzi-Heckler, who fused somatic training with psychotherapy and then forged a somatically grounded approach to leadership training. All five epigones published considerably. In particular, one might mention Strozzi-Heckler's influential anthology, \emph{Aikido and the New Warrior} (1985), which assembled writings by aikidoka who applied the practice in various domains, including family therapy, sports, and playing with animals. A later book, \emph{In Search of the Warrior Spirit} (1990), documents his efforts to engage professional soldiers in aikido ways, and \emph{The Leadership Dojo} (2007) bases management strength on integral body awareness. 

Aware of these disparate efforts, and of other practitioners who on their own had attempted to use aikido movements and ideas in areas outside of conventional dojo settings, I thought it might be of value to organize a little network to create and enhance communication among them. During a semester teaching in Berkeley in the spring of 1998, I discussed the idea with longtime sempais Wendy Palmer and Philip Emminger. Later that year I clapped, expecting that at least a dozen or two would clap back. They did. In October 1999, after frustrating legal delays and the like, we incorporated formally in the State of Delaware as Aiki Extensions, Inc. An initial founding membership consisted of about twenty Americans, including all those named above (Lee, Leonard, Linden, Moon, Strozzi-Heckler as well as Emminger and Palmer).

During those months of gestation I was pleased to discover a publication by Peter Schettgen and invited him to join the network. The first aikidoka outside North America to join the group, Peter served on the AE Board of Director for several years, attended the first three Aiki Extensions conferences in the U.S., and organized a series of conferences in Germany. The first two of these resulted in published collections of articles, \emph{Heilen Statt Hauen} (\emph{Heal Don't Hack!}, 2002) and \emph{Kreativitat statt Kampf!} (\emph{Creativity Not Combat!}, 2003). 

The growth of Aiki Extensions work in Germany has been phenomenal. During the past year the same has been true in Great Britain, thanks largely to the efforts of AE project director Mark Walsh and Quentin Cooke. At this point AE is clearly an international effort, with members in some twenty-seven countries in six continents. Its pioneering activities include novel forms of youth outreach, including a center for favela youngsters in Sao Paolo, Brazil; the Bronx Peace Village in New York; weekend gasshukus for kids and a program at the Seven Tepees Youth Center in the Bay Area, California; and a Peace Dojo that forms part of the Awassa Youth Campus in Ethiopia. Its most ambitious project was a four-day international seminar at Nicosia, Cyprus, in April 2005, from which has sprung a variety of continuing efforts to build bridges among Arabs and Israelis.

With the passing of so many of the first generation of direct students of the Founder of aikido, the whole question of the future of this distinctive international movement comes into question. There are those who say that its social and spiritual dimensions represent the most enduring and valuable aspects of aikido practice. Indeed, AE Director Strozzi-Heckler writes that Aiki Extensions is ``the 21st-century iteration of how O'Sensei envisioned aikido's role in global peace. AE is in a direct lineage to his vision and it is thus playing out what his vision projected in a world marked by transforming technologies and new epidemics of strife.''

\section*{References}

\begin{list}{}{\refstyle}
\item Anno, Motomichi. 1999. Interview.
\item Dobson, Terry and Miller, Victor. 1978. Giving in to Get Your Way. New York, Delacorte Press.
\item Dobson, Terry and Miller, Victor. 1994. Aikido in Everyday Life. Berkeley: North Atlantic Books.
\item Heckler, Richard Strozzi. 1985. Aikido and the New Warrior. Berkeley: North Atlantic Books.
\item Hikitsuchi, Michio. Date. The Birth of Aikido. Video.
\item Lee, James and Pulvino, Charles and Perrone, Philip. 1997. Restoring Harmony: A Guide for Managing Conflict in Schools. CITY: Prentice Hall.
\item Saotome, Mitsugi. 1986. Aikido and the Harmony of Nature. Boston: Shambhala Publications, Inc.
\item Shioda, Gozo. [1991] 2002. Aikido Shugyo: Harmony in Confrontation. Trans. by Jacques Payet and Christopher Johnston. CITY: Shindokan Books
\item Strozzi-Heckler, Richard. 1990. In Search of the Warrior Spirit. Berkeley: North Atlantic Books.
\item \_\_\_\_\_\_\_\_\_\_\_\_\_\_\_\_\_\_\_\_\_\_. and Leider, Richard. 2007. The Leadership Dojo: Build Your Foundation as an Exemplary Leader. Berkeley: Frog Books.
\item \_\_\_\_\_\_\_\_\_\_\_\_\_\_\_\_\_\_\_\_\_\_. Date. Personal Interview.
\item Tohei, Koichi. [19060] 1966. Aikido in Daily Life. Tokyo: Rikugei Publishing House.
\item Ueshiba, Kisshomaru. 1984. The Spirit of Aikido. New York: Kodansha America.
\end{list}