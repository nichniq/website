\chapter[Extending the Mature Vision of Morihei Ueshiba Sensei (2013)]{Extending the Mature Vision of Morihei Ueshiba Sensei: Aikido as a Universal Practice of Peace and Self-Transcendence\footnote{Keynote address for the European Aiki Extensions Seminar, ``Aikido---an Embodied Art 
of Peace,'' Burg Rothenfels, Germany, June 7-9, 2013.}}

\begin{verse}
Moi drodzy przyjaciele i wojownik-towarzysze dla pokoju! \\
Agapitoi mou filoi kai sympolemist\'{e}s gia tin eirini! \\
Mes chers amis et guerriers-comrades pour la paix! \\
My dear friends and fellow warriors for peace! \\
Meine liebe Freunde und Mit-K\"{a}mpfer f\"{u}r die Friede!
\end{verse}

It is wonderful to be here with you this week. I am so very grateful to Bertram Wohak and the good people around him who have mounted this amazing gathering. It is a special delight to share with you some thoughts about what we do, why we do it, and how we might do it even better.

To begin with: a deep bow to the Founder of our practice, Morihei Ueshiba Sensei, whom we love to call O'Sensei. The bow to him that commonly precedes our training signifies appreciation for his creation of aikido. But today, in addition, I bow to him as a special exemplar in two respects. First off, I salute him as a courageous leader, one who promulgated an inspiring vision and remained steadfast in pursuing it for the rest of his life. In this quest, he showed moral courage in rejecting the violent policies of his country prior to and during the World War. His exemplary leadership resulted in a global movement which continues to flourish generation after generation.

O'Sensei also stands as an exemplar by virtue of being an innovator who did not hold fast to a particular form, but continued to develop throughout his life. And this means that the example of his life encourages us to experiment and move beyond his own attainments. ``Life is growth,'' said the Founder. ``If we stop growing, technically and spiritually, we are as good as dead'' (20).\footnote{The page references for all quotations by Morihei Ueshiba are taken from Ueshiba, The Art of Peace: Teachings of the Founder of Aikido (1992).} In other sayings, he hinted at a course of action for growth. ``I did not invent aikido,'' he said, ``I discovered it.'' That is, aikido represents a set of phenomena that exist in nature, and if we study nature intently, we should be discovering other things there as well. 

I often liken O'Sensei in this respect to another pioneer who worked to help us overcome our inner discords, Sigmund Freud. Both Freud and Ueshiba learned and experimented with new ideas and techniques continuously. Their example was not always followed by their students, some of whom treated as canonical only those lessons which they took pains to incorporate during the years when they were training with the master. On the other hand, just as some of Freud's followers followed his example and went on to fashion novel analytic concepts and therapeutic tools, so a number of Ueshiba's deshis---including Senseis Koichi Tohei, Terry Dobson, Seiseki Abe, Motomichi Anno, Mary Heiny, Seishiro Endo, Robert Nadeau, Mitsugi Saotome, and many others---created technical and philosophical innovations that sought to advance aikido in the spirit of his mature teachings. My remarks today aim to carry on in that spirit.

\section*{O'Sensei's Later Path}

One way, I think, that aikidoka cling to outdated visions of aikidO'S Founder is when they define it simply as a Japanese Martial Art of Self-Defense. This definition fits the earlier phases of Ueshiba Sensei's teaching, and continues to provide useful entree for novices. However, I find it misleading as a representation of the vision that came to fruition in the last two decades of his life. 

Let us consider in turn each term of this definition. `Martial,' to begin with. Although aikido originated as \emph{aiki-bujutsu}, as a new style of fighting in a panoply of combative arts traditions that climaxed with the samurai warriors of the 16th century, after World War II Ueshiba Sensei resolved explicitly to abandon the aims and the methods of those traditions. Already in 1941, when Japan's war against the United States began, Gozo Shioda reports that O'Sensei turned to a more spiritual path. Shioda Sensei notes that he did not follow O'Sensei's teachings further at that point and thus claimed, with perhaps some hyperbole, to be the last of O'Sensei's students to be trained as a martial artist: ``The concept of Aikido as a martial skill has ended with me'' (Shioda 1977, 204). 

According to O'Sensei's long-time live-in disciple, Mitsugi Saotome, two experiences accelerated that resolve: the catastrophes of Hiroshima and Nagasaki, and revelations from a Japanese soldier present at the liberation of Hitler's concentration camps. O'Sensei signaled this intent at a fateful meeting with Hikitsuchi Sensei in 1948, when he invited the latter to join in developing ``a new kind of budo,'' one devoted explicitly to promoting world peace. This new budo, he emphasized, would be taught through an entirely different curriculum. Its methods were not to rely on pain or physical force, but to welcome of the energy of an attack, neutralize its aggressive direction, and care for the attacker. The structure of combat was transformed into a harmonious exchange of gestures. O'Sensei would represent this shift with the famous saying: ``The secret of Aikido is not in how you move your feet, it is how you move your mind. I'm not teaching you martial techniques. I'm teaching you non-violence.'' And the goals of this curriculum changed radically---from defeating an opponent to gaining victory over oneself---\emph{agatsu}. In later statements, O'Sensei identified two concrete ends of aikido: to help realize each individual's personal life mission, and to promote social harmony and world peace. In that spirit, aikidoka often translate aikido as The Art of Peace. 

In so doing, however, they use a term that can also be seen as problematic: `Art.' 

But please remember: O'Sensei followed the precedent of Jigoro Kano, who reconfigured martial training by changing the term \emph{bu-jutsu} to \emph{bu-do}. What is \emph{jutsu}? It signifies an art, a technique for accomplishing something. This word parallels the Greek word \emph{techne}, from which English gets the word `technique.' The \emph{jutsu} or art of a carpenter is to make tables, of a painter to make pictures, of a doctor to make sick people well, and of a warrior to make enemies dead. By contrast, \emph{do} signifies a Way---a way of being, a way of acting. As a \emph{do}, aikido is not an art, but a way of living. Mindful of how classical Greek philosophers contrasted art (\emph{techne}) and action (\emph{praxis})---and regarded the practice of philosophy as a ``way of life''---let us call it a \emph{practice}. This notion has affinity with the neo-Confucian concept of ``cultivating practice'' (\emph{xiuxing}), and resonates well with the dictum of Buddhist monk Thich Nhat Hanh: ``There is no way to peace---peace is the way.''

What is the point of that practice? Not self-defense, surely. For one thing, aikido is not about being defensive. It is easy to demonstrate how a defensive response produces continuing fear and discord, not harmony. Rather, the point of aiki practice is how to make connections---\emph{musubi}, if you will. As Saotome Shihan has noted on the mat: ``When someone grabs your wrist, it does not signify the beginning of an attack; it means the beginning of a conversation.'' To grasp fully the somatic shifts involved in \emph{receiving} an attack, not defending against it, takes years of practice; but is that not the point of so much of our training?

Moreover, aikido is not about defending the ``self.'' Recall what the Founder described as the objectives of aikido training: to realize one's mission and to harmonize with others. With regard to the self and its boundaries, that implies an effort to transcend boundaries of the mundane self. This starts with simple etiquette, which O'Sensei once called the most important outcome of aikido training. It extends to care for our species and our planet. ``Those who practice aikido,'' he insisted, `` must protect the domain of Mother Nature \ldots{}and keep it lovely and fresh'' (24). Both etiquette and care for the earth, and everything else in between, involve moving above and beyond the ego. ``Return to the source [of all things],'' said O'Sensei, ``and leave behind all self-centered thoughts, petty desires, and anger'' (16). Elsewhere he adds, ``Forget about your little self, detach yourself from objects, and you will radiate light and warmth'' (116). His words resonate with the neo-Confucian contrast between the ``small self'' (\emph{xiao wo}) and the ``big self'' (\emph{da wo}), which involves a broadening of vision to connect with a wider community (Madsen 2012, 438); and bears a family relationship to the Hindu contrast between the individual, personal self (\emph{atman}), and the universal atman that is identical with \emph{brahman}, the ultimate ground of all being.

Putting all these notions together, we can describe aikido as \emph{a practice of peace and self-transcendence}. 

And what, finally, about the term `Japanese?' Here, too, some revision is in order. To be sure, aikido was created in Japan, imbued with Japanese language, and associated with the distinctively Japanese religion of Shinto. Nevertheless, aikido is not Japanese in the same way that kabuki theater, ikebana, and sushi are Japanese. For one thing, the cultural roots of aikido stretch unmistakably across Asia. Key features of Hinduism, Buddhism, Taoism, and Confucianism are conspicuously present in the ideas and rituals of aikido practice. Shintoism itself, although often associated with Japanese identity, draws heavily on those other traditions, as William Gleason (1995) has emphasized. In his youth, Ueshiba studied many aspects of those traditions. A publication of the Japanese Budo Association affirms that Confucianism, Taoist thought, and Buddhism, were an integral part of the culture that went into the formation of aikido---that they all take universal nature worship as their direct foundation, and ``generally speaking assert the concept of humankind as being at one with the universe and nature. Accordingly, to this extent \emph{we cannot say that these beliefs constitute an `indigenous philosophy' of Japan}'' (Sadami 2005, 39; emphasis mine). Although we need to be careful in generalizing about traditions of the ``East,'' aikido is justifiably regarded as a bearer of ``Eastern'' thinking in ways that other Japanese martial arts are not. 

As such, the widespread appreciation of aikido among Western practitioners can be seen as a yearning to incorporate the ``Wisdom of the East.'' It fulfills what Karl Jaspers imagined, in the clairvoyant work published just after World War II, \emph{The Origins and Goal of History} (\emph{Vom Ursprung und Ziel der Geschichte}), when he asked: ``What is it that, despite all the pre-eminence of Europe, has been lost to the West? It is in Asia that we find what we lack and what vitally concerns us! \ldots{} Asia is indispensable for our completion'' (1983, 95; ``Was ist bei allem Vorrang Europas doch dem Abendland verlorengegangen? Es gibt in Asien, was uns fehlt und was uns doch wesentlich angeht! \ldots{} Asien ist unsere unerl\"{a}\ss{}liche Erg\"{a}nzung.'' Translation mine.) Such an insight indicates that, if aikido is supra-Japanese in its origins, it is all the more so in its contemporary appeal. As Ueshiba Sensei would have affirmed happily, it is \emph{universal}, a gift for humanity. 

In sum: on reflection, why not define aikido as: \textbf{A Universal Practice of Peace and Self-Transcendence?}

\section*{Aikido as a Prophetic Teaching}

If you follow me thus far and accept, if only for the sake of dialogue, the definition I just proffered, it might be fruitful to step back and reflect on what is at stake in the practice of aikido so defined. \emph{Universalism}, \emph{self-transcendence}, and \emph{peace}: what do these ideas bring to mind? They remind me of what Karl Jaspers, in the book already referred to, described as the great human breakthrough: the complex of new values articulated in different geographical centers during middle centuries of the first millennium BCE: in China, with Confucius and Lao-tse; in Greece, with Plato and Thucydides; in India, with the Upanishads and Buddha; in Iran, with Zarathustra; and in Palestine, with Isaiah and Jeremiah. Jaspers described that historical period as an ``Axial Age''---axial in the sense of constituting a turning point in the history of mankind. This era was followed by comparable breakthroughs in later times, including those of Jesus; of Mohammed; and of the European Enlightenment. Our own time, I believe, has witnessed the emergence of two prophetic figures who produced new breakthrough visions regarding universalism, self-transcendence, and peace: Mahatma Gandhi and Morihei Ueshiba.

In recent decades an international collegium of scholars has turned to Jaspers's formulations as a point of departure for wide-ranging analyses of the sources, forms, contents, and consequences of the Axial Period. Among the many insights these analyses have produced, what comes to my mind is certain tensions that inhere in the working out of these prophetic teachings in the world of humans. One is the competition among followers of the prophetic figures for ownership of what can be affirmed as the true message of his teaching. Another tension is the assertion of particularisms that subvert the universalism of the message. 

The brief history of aikido has known both tensions. Although O'Sensei's dying wish, eye witnesses tell us, was to implore his disciples to hold the movement together, not long after he passed tensions arose among those who claimed to posses the true version of his teachings. Over the years those tensions became more pronounced, and they persist to the present day. In addition, certain particularistic tensions arose, especially among those who sought to maintain the notion of aikido as a distinctively Japanese practice, one owned by the Japanese themselves. Recall, for instance, it was only a few years ago that Japanese authorities agreed to certify non-Japanese with the highest honorific titles previously preserved for Japanese nationals only.

Such tensions express perfectly natural reactions to ethically ambitious aspirations. The wish to stand as the main or the sole legitimate representative of one of the Axial visions and the wish to hold fast to group identities manifest deep human needs. That is why, time and again, many are called to retrieve and re-assert the claims of universalism and of positions that transcend the contentions among epigones. 

From the outset, Aiki Extensions has stood as a prominent organization in the aikido world that holds fast to the universalistic ideal of the mature Ueshiba's prophetic vision. AE's Training Across Borders Seminar in Cyprus, 2005, gave dramatic expression to this ideal. In the video made about that event, Richard Strozzi-Heckler commented on how readily persons from antagonistic groups come together in aikido because they relate to one another through their simple humanity. Since that time, Shihan Hiroshi Ikeda has regularly brought together practitioners from different aikido organizations in North America. This remarkable seminar at Rothenfels, with participants listed from several countries and diverse aikido organizations, continues that effort. 

Beyond that, it engages directly the question of how the teaching of aikido can be brought more in accord with other aspects of O'Sensei's evolved vision. In that spirit, the remainder of my remarks will suggest ways to do so that challenge us to refresh our conception of the aikido curriculum all together.

\section*{Revisioning Aikido}

The Founder claimed that the teachings of aikido are intended to shape the whole of everyday human experience. As one of his memorable sayings goes, \emph{aiki waza michi shirube}, training in aikido is a signpost to the Way. This implies that mat practices should feed directly into ways we handle all situations in personal and public daily life. Many if not most aikido instructors transmit this claim. Yet one may ask: does what transpires in the normal course of aikido training accord with this ideal? Here are four ways in which we might do so more systematically, ways which I shall present in a simple typology of four dimensions of aikido: reflexive; receptive; projective; and mediative.

\begin{enumerate}
\item The conventional aikido curriculum consists of training on the mat in techniques to neutralize and redirect the aggressive energies of attackers. That very fact should give us pause. For one thing, O'Sensei's curriculum was made up of two parts, as Robert Nadeau Sensei likes to remind us. In addition to \emph{keiko}, or practice on the mat, it involved \emph{benkyo}, or study. This model invites us pursue inquiries that ponder the verbal teachings of O'Sensei and to explore current issues and experiments that relate to them. This could become a formal part of our work, and not be left to casual off-the-mat occasional chats over beer. I propose that we envision a category of training called ``reflexive aikido,'' something that we expect to focus on in AE's international Aiki Peace Week.
\item How we talk about redirecting attacks continues to employ words and techniques that remain combative. I refer in particular about term `\emph{nage}.' \emph{Nageru}, to throw, derives from samurai days, and connotes an aggressive response to an attacker. On the mat, this connotation surfaces when, following an initial harmonious `blend' or \emph{musubi} connection, the person playing the \emph{nage} role moves to hurl down the attacker. That response does not fit the meaning of aiki. In a recent conversation with Anno Sensei, when asked if it was indeed not time to give up the term `\emph{nage},' he thought for a moment and replied, ``Perhaps it is.'' (If the word `\emph{uke}' were not already used to signify the attacker, I would suggest \emph{ukeru}, to receive, as the proper response of the person being attacked.) 

We are all familiar with the sense of O'Sensei's dictum: ``When an opponent comes forward, move in and greet him'' (77). So let us have the courage of O'Sensei's wisdom and designate this mode of training as ``receptive aikido.'' This is the bread and butter of aikido keiko as we know it. There are two changes I would introduce. One is to move beyond the word and even more the attitude of \emph{nageru}, of tossing our attacker down, and instead to conclude the aiki transaction with the notion of just letting the ki flow through. The other is the idea, which I learned from Mary Heiny Sensei, of actually moving our bodies to make room for the attacker. That promotes a more welcoming attitude. 
\item What is more, although attacks offer frequent challenges in our lives, unless we are soldiers in battle, or politicians on the stump, dealing with attacks does not comprise the major activity in our lives. Rather, our primary attention goes to renewing our daily energies and pursuing our respective missions. Should we not then shape training with a focus on initiating and carrying through our \emph{projects}? Let us call this work ``projective aikido,'' a term to designate practices that embody the initiation and execution of projects.

For this, the range of exercises invented by the late Koichi Tohei forms a fruitful point of departure. These focus on finding ways to extend ki and to avoid breaking ki. It is important to bear in mind: Tohei Sensei's advice to ``extend'' ki, \emph{ki o dasu}, does not mean to try to \emph{do something} with one's energy. Rather, the point is to relax into a position of good posture and feeling centered, remaining aware of the larger context of one's action, and then to maintain a clear focus on an end. That done, the flow of ki follows naturally in the direction of one's attention. This can be a protocol for all instances of projective aikido.

Somatically grounded guidelines for leadership represent one significant area in which projective aikido has been developed. Expanding Tohei Sensei's emphasis on focused attention, Richard Strozzi-Heckler advises: ``To fulfill on our pledge as leaders, it's essential to know what to attend to and how to extend our attention toward that end. Energy follows attention'' (2007, 148). Adapting the warrior idiom he encourages leaders to fight for a stand, which involves ``the ability to decline what's inconsequential, insist on what's right, require others to pay attention, demand justice, quit those who pull you away from your stand, and, if necessary, put your identity and body at risk for what you say is important'' (114). In a kindred vein, Wendy Palmer has reconfigured aiki-inspired work in Conscious Embodiment into a training program for leadership, which teach ways to alter reaction patterns to stress, be more inclusive, and speak up clearly without combativeness or collapsing. She has extended this program to work with leaders in and around Capetown, South Africa, to support a positive future for an integrated country.

Other areas in which projective aikido has been implemented include applications of aiki principles to work in the arts. A preeminent exemplar of this mode was the late Seiseki Abe Sensei, for whom the principles of aikido were essentially the same as those for \emph{shodo}, calligraphy. In the performing arts, we have a number of models to inspire us. These include Dance Improv, which was inspired directly from aikido; the work of Paul Linden and Pamela Ricard with dramatic actors; and in music, the examples of Craig Naylor's aiki conducting, Bill Levine's aiki playing on the keyboard, Masumi per Rostad in viola-do, and Jack Wada in aiki flowing on the trumpet.

When I started to experiment with this perspective on the mat, I used the term ``uke-centered'' aikido. But again, \emph{ukeru}, to receive, was just as inappropriate for the activity on initiating projects as \emph{nageru} was for receiving the energy of an attacker. In this case, I found a perfectly fine Japanese term, \emph{hajimi}, which signifies one who starts something. And so, whether it be an attack on the mat, the draft of a charter, a brush stroke on parchment, or the stroke of a bow on a viola string, the point of training is to enhance the capacity to focus attention, to center oneself prior to initiating the move, to proceed freely and responsively, and to deal with obstacles in a caring and protective way. Creative aikido surely represents a dimension of action to which the notion of \emph{takemusu aiki} is exceptionally relevant.
\item Finally: what in our usual training embodies the goal of turning social antagonisms into harmony? Many of us speak of social conflict resolution as an important contribution of aikido, yet how often to we turn to aikido for ways to prevent violence or resolve the stopping of fights between others. The field is open for us to comb the literature and practice of aikido to codify exercises that enable conflicts to take a constructive turn. Here we would do well to collaborate with other disciplines that deal with conflict resolution. The practice of mediation by lawyers and former judges has been developed a great deal in recent years. Non-Violent Communication likewise gained an international following. 

For professional mediation, we already have a number of readily usable ideas. These include ways to affect the conduct of the disputing parties and their lawyers; to enhance the mediator's effect on the interactional context of the mediation efforts; and to guide the personal conduct of the mediators themselves (Levine 2013). Might we devise new exercises on the mat that work to break up fights and move combatants toward harmonious resolutions? This whole complex could form a challenging frontier area: Mediation-centered Aikido.
\end{enumerate}

To sum up: to refresh the aikido curriculum to bring it into greater harmony with the full teachings of the mature Ueshiba Sensei, I propose an approach to our practice that organizes it in terms of four different dimensions: 1) \emph{reflexive} aikido, to ponder the meaning of our practice; 2) \emph{receptive} aikido, to deal with attacks from others; 3) \emph{projective} aikido, to promote the initiation and execution of projects; and 4) \emph{mediative} aikido, to help resolve conflicts among others. 

My remarks this evening have been in the mode of reflexive aikido, or \emph{benkyo}. Tomorrow morning in \emph{keiko} I shall suggest some techniques for so doing. For now, I close with an expression of enormous gratitude for your attention and for sharing your time with me this evening. \emph{Domo arigato gozai mashita}.

\section*{References}

\begin{list}{}{\refstyle}
\item Gleason , William. 1995. Spiritual Foundations of Aikido. Location: Destiny Books. 
\item Jaspers, Karl. (1949) 1983. Vom Ursprung und Ziel der Geschichte. Munich: R. Piper.
\item Levine, Donald. 2013. ``Aikido and the Art of Mediation''. 
\item Madsen, Richard. 2007. Democracy's Dharma: Religious Renaissance and Political Development in Taiwan. Berkeley: University of California Press.
\item Sadami, Suzuki. 2005. ``Twentieth Century Budo and Mystic Experience.'' In Budo Perspectives, ed. Alexander Bennett. Auckland, NZ: Kendo World.
\item Shioda, Gozo. 1977. Dynamic Aikido: The Way of the Warrior. NY: Kodansha International.
\item Strozzi-Heckler, Richard. 2007. The Leadership Dojo: Build Your Foundation as an Exemplary Leader. Frog Books. 
\item Ueshiba, Morihei. 1992.The Art of Peace: Teachings of the Founder of Aikido. Boston: Shambhala.
\end{list}