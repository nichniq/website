\chapter{From Fear Politics to Harmony in Action}
%\addcontentsline{toc}{chapter}{\numberline{\bfseries{Preface}}\hspace{3em}From Fear Politics to Harmony in Action}

When invited to take part in an event celebrating forty years of Linda Holiday Sensei in aikido and the 41st anniversary of the dojo she has guided so lustrously, I asked myself what sentiments inform my feeling of a special connection with her. What sprang to mind was the dictum that forms the title of this collection, words attributed to the Founder of this discipline, Morihei Ueshiba Sensei. In my network of aikido instructors, Linda subscribed to that dictum prominently, cherishing its kanjis and embodying the notion that the point of aikido practice is not mastery of a repertoire of techniques but guidance toward The Way they illuminate.

That thought in turn prompted me to reflect on what I might bring to the event from my own journey on behalf of the Aiki Way. In a moment I had it: how about ransack my sprawling bibliography for a selection of articles and talks that evolved over the years as I pursued the meaning of that dictum?---not for the sake of a coherent book but rather to collect a random set of stepping-stones along the Way. Now that the collection of pieces lies before me, I am astonished to find how much coherence they evince, created as they were in rather different formats for even more different kinds of occasions. So it may be of interest for some readers to view some threads that appear now as I think about weaving them into a single bound volume.

\vspace{1.5em}

{\uppercase{Although none}} of this could have been foreseen at the time, the entire sequence was sparked by an invitation to present a public lecture to an audience of undergraduate students at The University of Chicago in 1983. This event transpired in the framework of a locally celebrated lecture series---organized by a Residence Hall Master, the beloved late Professor of Mathematics Isaac Wirszup, and his warm wife, Vera. In the midst of serving a term as dean of The College, it made sense for me to say something related to education; but then I seized the opportunity to speculate about what might be educational about the new disciplines which, at the age of 48, had recently evoked so strong a passion in my being.  Thus was born the first of these chapters, ``The Liberal Arts and the Martial Arts,'' a title then so absurd that a neighbor quipped, when he heard of it: ``Why that's like giving a talk on the subject, `Lincoln: Man and Car.'''

My neighbor's quip in fact motivated me to dig more deeply than I might have, to search for historical and philosophical justifications for combining the two notions. And what I found was in fact more of a revelation than I had expected, so striking that it eventuated in a searching book two decades later, \emph{Powers of the Mind: The Reinvention of Liberal Learning in America} (2005)---a book that brazenly asserts the value of including something like aikido practice in the curriculum of every undergraduate program of liberal education.

As dean, however, I had more on my plate than public spectacles and routine administration. I made it my principal business to reconsider the whole curriculum of The College, to launch a discursive enterprise of a sort that distinguished the University of Chicago since its founding (as narrated in Chapter Three of \emph{Powers}). In so doing, I designed a curricular project, which engaged close to a hundred faculty members and advisers in a yearlong effort to rethink the whole four or more years of undergraduate experience. This project took shape through teams of colleagues grouped around a dozen themes, such as the Task Forces on Musical and Visual Arts, on Civilizations, on the Senior Year, on Writing, and the like. One task force---bizarre for the notoriously cerebral University of Chicago community---was assigned to examine afresh the role of physical education. This group, headed by John MacAloon, author of the first cultural- historical analysis of the Olympic Games, produced the report that exerted the greatest impact on my pedagogic development. It broached the novel idea of linking the academic side of undergraduate work with experience in the gymnasium, through courses in which some sort of physical activity could be linked with an academic subject.

The notion gripped me, much as I had been gripped shortly before by the idea of conjoining the liberal arts and the martial arts. By the final year of my decanal term, I found myself offering a course entitled Conflict Theory and Aikido. I offered the course more or less regularly from Autumn 1986 to Autumn 2010; its most recent syllabus appears as Appendix A below. The course proved so successful and engaging that a few years later I found myself reporting on it at a Japanese-American Conference on athletics and undergraduate education. The published version of that talk appears here as Chapter 2, and an updated report on student comments as Appendix B. 

As I reflected on the materials and issues of that course, moreover, I came to ponder some intellectual issues incorporated in them. This led to the first strictly academic paper in this collection, Chapter 3: ``Social Conflict, Aggression and the Body in Euro-American and Asian Social Thought.'' Presented at an international sociological conference in Paris in 1993 and published soon after in the \emph{International Journal of Group Tensions}, the paper took aikido practices as a kind of text that could be compared with other texts about conflict by figures such as Hobbes, Freud, Morgenthau, Lorenz, and Gandhi. It was a sort of crib sheet for my course.

By the mid-1990s, much else was going on in my life: completion of a major work on the sociological tradition that was perhaps my most visible accomplishment in the sociological community; renewed interest in Ethiopia (on which I had already published two books) thanks to the fall of the hated Derg regime in 1991; an evolving interest in the history and culture of Japan, which I visited with my son Bill in 1992 and my wife Ruth in 1997; first shoots of the work that would constitute my major statement in the field of liberal learning; and a more prominent role in the area of social theory, including an array of fresh papers and an active term as chair of the Theory Section of the American Sociological Association in 1996. Even so, the trajectory laid out the first three pieces of this collection could not be stopped. I began to conjure the idea of a book-length work on what I wanted to call The Aiki Way. I even sketched an outline for a number of chapters, to be focused on such diverse areas of aikido applications as conflict resolution, psychotherapy, administration, character development, and even philosophy. My mature version of conceptualizing aikido in this manner appears as A Paradigm of the Aiki Way, here Appendix C, which offers a schema with which to list concepts that embody aikido practices and their practical applications all at once. 

Rather than pursue the idea of that book, however, I decided instead to form an association, one that would bring together the small number of aikido practitioners and instructors I'd met who were also committed to using the ideas and movements of aikido to effect changes in everyday life. During a semester at the University of California, Berkeley, in 1998, I met with a number of kindred souls and began to plot the outlines of the nongovernmental organization that came to be called Aiki Extensions, Inc. We began to hold international conferences: in Tucson, AZ (hosted by Bill Leicht), Columbus, OH (Paul Linden), Mill Valley, CA (Wendy Palmer), and Chicago, IL (myself). The story of the first five years of this NGO and its work is told in Chapter 4---``The Many Directions of Aiki Extensions''---a talk given at Augsburg, Germany at the fifth of the International Conferences. 

As the work of Aiki Extensions grew, so did the range of intellectual issues I wished to associate with explorations of the Aiki Way. The ensuing publications---here chapters 5 through 8---appeared in response to a sequence of occasions where aiki-relevant themes came to my attention.

The invitation to contribute to a session on the Sociology of the Body at meetings of the International Institute of Sociology in Stockholm, 2005, offered an apt venue for developing the ideas broached in chapter 3. The result, here chapter 5, appeared in 2006 as ``Somatic Elements in Social Conflict,'' in \emph{Embodying Sociology: Retrospect, Progress and Prospects}, ed. Chris Shilling. Oxford: Blackwell.

Shortly before that, I was invited a session of the Research Committee on Armed Forces and Conflict Resolution at the World Congress of Sociology in Brisbane. That gave me an occasion to develop ideas from work on a long paper, ``Ethiopia in Japan in Comparative Civilizational Perspective.'' I had become more convinced than ever that certain self-destructive aspects of Ethiopia's political life reflected the persistence of age old features of their warrior culture, a concern that came to the fore in a widely read paper from the Fourth International Conference on Ethiopian Development Studies at Western Michigan University.

A key point of the comparison held that Ethiopia had missed the kind of transition represented in the felicitous title of Eiko Ikegami's book, \emph{The Taming of the Samurai}. In Japan, martial traditions of killing techniques, \emph{bu-jutsu}, had been transformed into character-building ways of life, \emph{budo}, a transformation of which I believe aikido is the highest exemplar. The resulting 2006 paper---here Chapter 6---was titled ``The Masculinity Ethic and the Spirit of Warriorhood in Ethiopian and Japanese Cultures.'' \emph{International Journal of Ethiopian Studies} Vol.2, Nos.1\&2.

Locating the cultural context of aikido in comparative perspective proved too enticing not to extend further. An invitation to honor the doyen of comparative historical sociologists, Professor Shmuel N. Eisenstadt of Jerusalem, provided a wonderful occasion in which to do so. That 2003 presentation, ``Civilizational Resources for Dialogic Engagement?,'' was published in \emph{Comparing Modern Civilizations: Pluralism versus Homogeneity}, ed. Eliezer Ben-Rafael. Boston: Brill. For the Journal of Classical Sociology, producing an issue to commemorate the now late Professor Eisenstadt, I have substantially revised and updated the paper, a text that forms the basis for chapter 7, ``The Dialogue of Civilizations.''

Chapter 8 returns to a context shaped by Aiki Extensions: a conference on ``Living Aikido: Art of Movement, Art of Life,'' May 18, 2007 at the Aiki Institut in Schweinfurt, Germany. The lecture presented there, ``The Aiki Way to Therapeutic and Creative Intersubjectivity'' connects aiki work with earlier interests in psychoanalytic sociology and Parsonian theory.

The last chapter brings the journey up to date. Chapter 9 expands the narrative of Aiki Extensions further back in time and forward into future challenges and promises. The story was retold later in a video produced by AE Board member David Lukoff.

The foregoing narrative projects the gist of the story and rounds out my tale. Except for one obvious question: how did an academic intellectual ever get into this whole business in the first place???

\vspace{1.5em}

{\uppercase{Finishing high school in 1948}}, I felt buoyed by America's upbeat political atmosphere. Despite forebodings, hopeful and confident voices ruled. It was after all the time of the Marshall Plan, of Point Four, of the critical turn among progressive forces by those willing to take a strong stand against totalitarianism Left as well as Right. My own postwar idealism found nourishment in the world government movement, subserving an impeccable logic that found a ready analogy between policemen on the corner who spelled local security and a prospective world federal authority that spelled collective security. 

Korea's War smashed the hopes of those of us who assumed the road to world federal union might be forward and continuous. Voices and forces of U.S. belligerence forged a bipolar world. Public life suffered a remorseless escalation of fear. In 1950 I signed a plea for the United States not to be the first party to use a nuclear strike---the ``Stockholm Peace Petition''---and nearly went to jail; news I had done so treasonous a thing flashed from the Pittsburgh Post-Gazette front page. Building on the anti-communist hysteria fed by ambitious politicians, our State Department got cleansed of patriotic public servants knowledgeable about Russia and China. 

One piece of this upsetting development related to the self-image of American males. As Talcott Parsons precociously surmised (1954), overly mothered males turned fear of sissiness (and, Norman Mailer would add later, homophobia) into protest masculinity and externalized aggression. Free-floating anxiety permeated the Eisenhower-Dulles decade, starting with the CIA's miserable decimation of Iran's democratic regime under Mossadegh. Few understood and none acted on Eisenhower's testamentary warning of the military-industrial complex. The young senator who downed Ike's would-be successor trumpeted a spurious claim that the U.S. had an inadequate military arsenal.

After 1950 I searched for plausible countervailing forces, and found only the sterling pacifism of the American Friends Service Committee. I loved what they did but my mind could not accept absolute pacifism as a life doctrine. Like many of my activist colleagues, I turned toward what we thought we had named, in quiet protest against the bipolar structure of the Cold War, \emph{le tiers monde}, the Third World. My friend and role model Harris Wofford---later architect of the Peace Corps and a U.S. Senator---went with his wife Claire to Israel and then India; Manny Wallerstein (now a distinguished senior professor of sociology) went to the Gold Coast (later Ghana); Larry Fuchs went to the Philippines; Myron Weiner went to India; others went to Turkey, Indonesia, and elsewhere. Many forsook politics for philosophy.

I almost went the latter way, immersing myself in abstract social theory instead of social realities. Georg Simmel and Talcott Parsons became my homies. I became drawn wholeheartedly to the philosophical outlook of Richard McKeon, whose contribution to calming Cold War tensions consisted of elaborating a way of embracing philosophic \emph{differences} without having to resort to ideological \emph{combat}. Eventually, though, I found my way back into la vie \emph{engag\'{e}e}. I discovered and embraced the land of Ethiopia.

\vspace{1.5em}

{\uppercase{Ethiopia appealed}} for several reasons. Ethiopians I met impressed me with their self-confidence, ego strength, and cheer under adversity---quite unlike the many fearful and frenzied Americans I encountered. Symbolically, Ethiopia's bitter invasion by Italian Fascists---who opted to pursue a \emph{missione civilatrice} by dropping poisoned gas on barefoot farmers---represented par excellence the failure of the international community to manifest collective security and develop a reliable system of world governance. To a budding sociologist, moreover, Ethiopia offered a challenge to use the resources of social science to mediate a traditional society's lurch into modernization (the topic of my first book on Ethiopia, \emph{Wax and Gold: Tradition and Innovation in Ethiopian Culture} [1965]). It also offered the intriguing puzzle of how a ``backward'' society could possibly have defeated a European colonial power. This was Ethiopia's stunning victory under Emperor Menilek II against Italian invaders in (the point of departure of my second book about the country, \emph{Greater Ethiopia: The Evolution of a Multiethnic Society} [1974].)

What I had not expected to find in Ethiopia was the prominence of the cult of warriorhood. The provincial region where I did my main fieldwork, Menz, was noted for the hardiness of its denizens. The people of Menz cheered their boys on when they had temper tantrums. They gave them names like Ahide (Thrasher), Belaymeta (Hit Him on the Head), Chafchefe (Hacker), Nadaw (Wipe Him Out), and Tasaw (Smash Him). These and related findings were reported in \emph{Wax and Gold}. The association of masculinity with warriorhood became a theme I would explore further in Ethiopia. I found that in all the 70-some ethnicities, aggressive hardiness and virtues of the soldier were highly prized. This proved to be one factor behind the amazing defeat of Italian forces at the Battle of Adwa in 1896. And, half a century later, this cultural trait resurfaced on the international stage when Ethiopians earned a golden reputation as reliable and effective soldiery on behalf of missions for the United Nations, in Korea in the early 1950s, and in the Congo in 1960---the same year in which an Ethiopian named Abebe Bikila, running barefoot and without the benefit of much formal training, surprised the world by winning a gold medal in the marathon race of the Olympic games at Rome.

As it turned out, then, my three years of experience in Ethiopia, 1958-60, led me to internalize some of the Ethopians' warrior ethos and to feel myself more of a man by virtue of being more disposed to combat. That of course conflicted not only with my earlier inclinations toward pacifism, but also with what appeared to be the new rash of mindless escalation of US militarism in the Vietnam War. Before long I became active in protests against that War, even as I refused to give up my high regard for the virtues of warriorhood---and wishing that I could manifest more of those virtues. 

By the late 1970s, in my upper forties then, I decided finally to begin training in the martial arts. One day I went to a martial arts shop and purchased two books, one on karate and one on aikido. I knew nothing about the latter but thought I would look it over even as I kept looking for a place to learn karate. Then I chanced upon a notice of a campus aikido class, of a club founded by Jon Eley Sensei, and thought there would be no harm in checking it over. I did, and fell forward for it; it was love at first sight.

\vspace{1.5em}

{\uppercase{Aikido appealed}} to me initially as a martial art that seemed to offer a person in my age group an entr\'{e}e into a martial discipline that I might learn to excel in. That of course was flattering to my ego. Above all, its rhetoric of combining warriorhood with nonviolence offered just what I had been searching for. I plunged right in, never missing a class. Before long I was ready for my 6th kyu test. Slowly I began to walk through the dark streets of Hyde Park with greater confidence and to ease my way into what André Protin termed perfectly as ``\emph{un art martial, une autre mani\`{e}re d'etre}'': a martial art that embodies a whole ``other way of Being.''

More slowly but no less surely, aikido promised to offer the path I had sought for decades, wherein one could conjoin elements of what might be called an ethic of warriorhood with an ethic of nonviolence. Awareness of this potential emerged during the second year of my aikido training, which took place during a sabbatical year at the Center for Advanced Study in the Behavioral Sciences in Palo Alto. Studying that year with Senseis Frank Doran and Bob Nadeau, training with so many community-minded partners in Northern California---including many present at the 2010 Santa Cruz celebration---during that Golden Era of American aikido, and then having the good fortune of become a student of Shihan Mitsugi Saotome nourished my receptiveness to the idea of aikido as a Way in the spiritual sense. It was in that rich soil that seeds were planted, which not long after grew into the foundations of my teaching and research founded on the principle that Aikido Practice is a Signpost to The Way.