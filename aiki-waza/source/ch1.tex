\chapter[The Liberal Arts and the Martial Arts (1984)]{The Liberal Arts and the Martial Arts\footnote{Donald N Levine is a professor of sociology and the dean of the College at the University of Chicago. Larry Basem and Susan Henking assisted in its preparation for publication. \copyright{}1984 Association of American Colleges, \emph{Liberal Education}, 1984, Vol. 70, No. 3. Reprinted in \emph{The Overlook Martial Arts Reader: An Anthology of Historical and Philosophical Writings}, Overlook Press, 1989.}}

A complete rhetoric for liberal education must address the following six questions: 

\begin{enumerate}
\item What is ``liberal'' about liberal education?
\item What kinds of cultural forms are most suitable for the constitution of a liberal program?
\item What kinds of individual capacities should liberal training foster?
\item What are the characteristics of training programs designed to cultivate those capacities?
\item What is the relationship between liberal and utilitarian learning?
\item What is the ethical justification of liberal learning?
\end{enumerate}

In what follows I propose to clarify these questions by asking what we might gain by comparing the liberal arts with the martial arts---those forms of physical training and expression epitomized in the cultures of East Asia by kung fu, tai chi chuan, judo, karate-do, kendo, and aikido. My point is not to argue that some form of athletic training ought to be an integral part of the liberal curriculum, though on that question I find myself in accord with the views expressed by William Rainey Harper, who said: ``The athletic work of the students is a vital part of student life \ldots{} The athletic field, like the gymnasium, is one of the University laboratories and by no means the least important one.''\footnote{William Rainey Harper, Convocation, 1 July 1896. Cited in W.M. Murphy and D.J.R. Bruckner, eds., \emph{The Idea of the University of Chicago} (Chicago: University of Chicago Press, 1976), p. 212.} My argument, rather, is that courses of training in the martial arts often constitute exemplary educational programs, and that we might learn something of value for the liberal arts by examining them closely. Just to propose this will perhaps seem to some an act of buffoonery. To suggest that the martial arts are worthy of consideration on the same plane as that usually reserved for the liberal arts--surely that is nothing more than a bad pun. So I must begin by justifying my brazenness in coupling the arts, liberal and martial.

Before proceeding to justify my topic, however, I must confess that one thing about it is indeed gauche. Its two contrasting terms, ``liberal'' and ``martial,'' are not logically comparable. For ``martial'' refers to a kind of content---physical training for self-defense---while ``liberal'' refers to a quality of approach in training. A logical contrast to the martial arts would be either some other kind of physical training, or else some kind of non-physical training---which, of course, is what we have in mind, what might be called mental or intellectual arts. The logical contrast to liberal would be \ldots{} illiberal. If we provisionally define liberal arts as signifying pursuits undertaken for the sake of personal growth and self-development, then it is clearly the case that both the martial arts and the intellectual arts have both liberal and illiberal forms. So the comparison I want to make here is between the liberal (intellectual) arts and the (liberal) martial arts. 

So rephrased, my topic will be justified by arguing that the very culture that originated and legitimated the basic conception of liberal arts we follow in the West supported, at the same time, a conception of martial training as an integral part of the ideal educational program; and that, moreover, the tradition that provided the matrix for the martial arts in the East saw them as part of what can be called an Oriental program of liberal education as well.  Once I have defended those propositions, I shall turn to the comparison that is the heart of this exercise.

\section*{I}

To talk about liberal training is to talk about a form of education that emerged historically only in two very special cultures, those of classical Greece and China. In ancient Greece, this kind of educational aspiration was linked to the ideal of \emph{paideia}, the notion of using culture as a means to create a higher type of human being. According to Werner Jaeger, who wrote a celebrated book on the subject, the Greeks believed that education in this sense ``embodied the purpose of all human effort. It was, they held, the ultimate justification for the existence of both the individual and the community.''\footnote{Werner Jaeger, \emph{Paideia: The Ideals of Greek Culture}, Vol. I, trans. from the second German edition by Gilbert Highet (Oxford: Basil Blackwell, 1939), p. xvii.} That ennobling education took two major forms that were equally praised by the writers of ancient Greece, albeit with different emphases at different times---the cultivation of combative skills, on the one hand, and the contemplative intellect, on the other. 

To see the affinity between the martial arts and the arts of contemplation in ancient Greece let us look at two notions central to Greek thought: the concept of \emph{ar\^{e}te} and the understanding of the divine. 

\emph{Ar\^{e}te}, often translated by the word ``virtue,'' was the Greek term that conveyed the notion of qualitative excellence. \emph{Ar\^{e}te} signified a special power, an ability to do something; its possession was the hallmark of the man of nobility. The same term \emph{ar\^{e}te} was used to designate both the special powers of the body, such as strength and vigor, and the powers of the mind, such as sharpness and insight. In the Homeric epics, martial prowess was the kind of \emph{ar\^{e}te} that was preeminently extolled, but with Xenophanes and other writers of the sixth century B.C., the attainment of \emph{sophia}, or intellectual culture, was hailed as the path to \emph{ar\^{e}te}. Although Xenophanes wrote in a rather polemical vein against the older ideals of martial \emph{ar\^{e}te}, most classical Greek writers embraced them both. Thus, the poet Simonides could write: ``How hard it is to become a man of true \emph{ar\^{e}te}, four-square and faultless in hand and foot and mind.''\footnote{Ibid., p. 212.} For Plato and Aristotle, the list of preeminent virtues begins with courage, and ends with philosophic wisdom (with prudence and justice in the middle). 

Although the Greeks are best known to us as the progenitors of secular science and philosophy, they are known to classical scholars as a God-intoxicated people as well. And, so far as I can tell, there are preeminently two human activities that are repeatedly described as divine in Greek thought---the achievements of victors in athletic contests, and the activities of philosophic speculation. Since earliest known history Greek gymnastic activity was connected with the festivals of the gods. The four great pan-Hellenic games, of which the Olympics were the most famous, were cloaked in religious symbolism; thus, both the Olympian and the Nemean games were held in honor of Zeus. As Norman Gardiner has written of the former, the Games were ``much more than a mere athletic meeting. It was the national religious festival of the whole Greek race.''\footnote{E. Norman Gardiner, \emph{Athletics of the Ancient World} (Oxford: Clarendon Press, 1930), p. 222.} The poetry of Pindar celebrated this linkage with \ldots{} Pindaric rapture. In his triumphal hymns for victors of the athletic contests, 
\newline{}
Pindar expressed the religious significance of the spectacle of men struggling to bring their humanity to perfection in victorious combat.

One finds the pursuit of metaphysical speculation described with tones no less transcendent. Greek natural philosophers of the sixth century created a conception of a cosmos under the rule of law that offered a focus for their religious ideals; and Pindar's contemporary, Heraclitus, developed a doctrine that located man in that cosmos, one that held that ``through its kinship with the `everlasting fire' of the cosmos the philosophical soul is capable of knowing divine wisdom and harbouring it in itself.''\footnote{Cited in Jaeger, p. 183.} A century later, Plato and Aristotle in different ways depicted the activity of philosophic contemplation of pure Being as the most godlike of human activities. 

In the classic Greek synthesis, then, the arts of combat and the arts of intellect were conjointly eulogized. They were the vehicles of that supreme educational effort, the cultivation of the virtues, and of the journey to transcendence. In both, the Greeks found a supreme expression of their aesthetic quest, the beauty of the bodily form perfected, and the beauty of the universe refracted in the contemplation of pure cosmic forms. 

By the end of the fifth century, however, the unity of body and spirit that Simonides and others idealized became fractured. Due to the heightened importance of prizes and spectators, the athletic games became much more competitive. Athletes became professionalized; physical training no longer sought all-round development but aimed to produce strength at the expense of vitality, health, and beauty. Moreover, once the Greeks began to feel that the spirit was separate from or even hostile to the body, Jaeger tells us, ``the old athletic ideal was degraded beyond hope of salvation, and at once lost its important position in Greek life.''\footnote{Ibid., p. 206.}

During the Hellenistic period, the liberal program underwent changes that were fateful for the subsequent evolution of education in the West. Although athletic sports continued as a popular public spectacle, their formative role as part of liberal training declined markedly, and disappeared altogether by the time of the Christian period. There was a similarly progressive decline and eventual disappearance of artistic, especially musical, education, which had also been a major component of education in the classical period. What emerged as the sole respectable form of liberal education was literary studies. 

During the Roman period the literary curriculum was further elaborated, particularly the study of grammar and rhetoric. Although early Christian fathers were suspicious of these pagan subjects, by the fourth century A.D. Christian leaders like Augustine embraced major elements of the classical curriculum. Consequently, when the barbarian invasions had swept aside the traditional Roman schools, the Christian church, needing a literary culture for the education of its clergy, kept alive many of the educational traditions that Rome had adapted from the Hellenistic world. 

By the sixth century A.D. the clergy had rationalized the literary curriculum into the trivium---the arts of logic, grammar, and rhetoric---and a few centuries later institutionalized the quadrivium---the ancient Pythagorean program of mathematics consisting of arithmetic, geometry, astronomy, and music.

In the ninth century, Charlemagne restored some semblance of higher studies, drawing on traditions that had been maintained in Italian and Irish monasteries. The Carolingian Renaissance, reinforced by the rise of scholasticism, the beginnings of law and medicine as professions, and the recovery of classical knowledge nourished the liberal arts curriculum until it was securely established in the medieval university. During the Renaissance this curriculum was enriched by an emphasis on the humanistic significance of the classic texts. The Reformation brought a renewed effort to subordinate the trivium and quadrivium to religious materials and purposes. 

The liberal arts tradition (in its English manifestation) came to America with the Puritan divines in Massachusetts. Liberal education came to be instituted in the American college in a framework that combined Protestant piety and mental discipline. The mental discipline approach, justified in English and Scottish moral philosophy, held that mental faculties were best developed through their exercise. In the course of recitations in the areas of Latin, Greek, and mathematics, the student disciplined mental and moral faculties such as will, emotion, and intellect. As William F. Allen wrote: ``The student who has acquired the habit of never letting go a puzzling problem---say a rare Greek verb---until he has analyzed its every element, and under-stands every point in its etymology, has the habit of mind which will enable him to follow out a legal subtlety with the same accuracy.''\footnote{Laurence Veysey, \emph{The Emergence of the American University} (Chicago: University of Chicago Press, 1965), p. 24.}

The rapid modernization of American society after the Civil War gave rise to new perspectives on the role of higher education. Laurence Veysey has identified three rationales of academic reform, which came to compete with that of ``mental discipline'' in the late nineteenth century. He calls these the programs of utility, research, and liberal culture. The advocates of utility argued that the American university should prepare students to serve the needs of American society for skilled leadership in modern industry, business, and government. Inspired by the model of the German university, the advocates of research insisted that the sole mission of the American university should be the furthering of the frontiers of knowledge. The advocates of liberal culture, however, condemned utility for its crass philistinism, and research for its encouragement of what they considered sterile specialization. In their emphasis on a refined sense of value, through the study of language and literature, the advocates of liberal culture in late nineteenth century America hearkened back to the humanists of the Renaissance. The discovery of an essential and irreducible humanity, which they called ``character,'' was made possible by breadth of learning. This, together with the aim of self-realization, was the appropriate rationale for higher education according to such advocates of liberal culture as Barrett Wendell, Charles Eliot Norton, Andrew F. West, and Woodrow Wilson.\footnote{Some sense of the ideals of this movement may be gleaned from the following quotations from Andrew F. West: ``In the rush of American life \ldots{} [the college] \ldots{} [is] the quiet and convincing teacher of higher things. It has been preparing young men for a better career in the world by withdrawing them for a while from the world to cultivate their minds and hearts by contact with things intellectual and spiritual.'' \ldots{} and from Woodrow Wilson: ``If the chief end of man is to make a living, why, make a living any way you can. But if ever it has been shown to him in some quiet place where he has been withdrawn from the interests of the world, that the chief end of man is to keep his soul untouched from the corrupt influences and to see to it that his fellow-men hear the truth from his lips, he will never get that out of conscious-ness again.'' (Cited in Veysey, p. 216.)} Such was the intellectual background behind those well-known experiments in the liberal curriculum following World War I associated with the general education program at Columbia, with Alexander Meiklejohn at Amherst and Wisconsin, and with the Hutchins College at the University of Chicago. 

\section*{II}

Contemporary with the archaic and classical periods of ancient Greece, in China during the Chou dynasty we find an educational program that bears significant resemblance to that of the Greeks. The goal of education was to produce a broadly cultivated person, and this included training both in literary and martial subjects. The curriculum codified during the Chou period consisted of six subjects, often referred to as the liberal arts of classical Chinese education: rituals, music, archery, charioteering, writing, and mathematics. According to the historian Ping Wen Kuo: ``A liberal education included five kinds of ritual, five kinds of music, five ways of archery, five ways of directing a chariot, six kinds of writing, and nine operations of mathematics. \ldots{} The training was moral, physical, and intellectual in character. \ldots{} The ideal of education of the time of the Chou seems to have been the harmonious and symmetrical development of the body and mind, and may be said to represent a combination of Spartan and Athenian ideals of education, which called for a training at once intellectual and moral, as well as physical and military.''\footnote{Ping Wen Kuo, \emph{The Chinese System of Public Education} (New York: Teacher's College Press, 1914), p. 18.}

During the latter sixth century B.C., Confucius articulated the conception of the ideal person to be produced by this Chinese version of \emph{paideia}. He defined that ideal as one who possesses wisdom and courage, who is also magnanimous and accomplished in courtesy, ceremonial, and music. He heavily stressed the virtue of sincerity and held that education was a means to gain an enlightened mind, enlightened in the sense of coming to grasp the remarkable harmonies of nature. 

In later centuries this ideal of liberal learning was eroded as the study of Confucian texts became viewed in a more utilitarian vein, simply as preparation for the requirements of bureaucratic office. The martial subjects were dropped from the standard curriculum. However, new forms of martial training were incorporated in disciplines followed in Chinese monasteries. To understand that development, we must digress for a moment to ancient India. 

When the Hindus rationalized a program of muscular and breathing training in the discipline of Yoga, they created a system directed toward the perfection of the body with the intent of making it a fit instrument for spiritual perfection-a perfection consisting of beauty, grace, strength, and adamantine hardness. At an early stage in the development of Buddhism, systematic physical training became a central component of religious discipline. It is said that Gautama was so impressed with Indian fist fighting as an effective method of unifying mind and body that fist art was incorporated into the framework of Buddhism. This can be seen in the images of certain gods of the Buddhist pantheon-the two Guardian deities, the Devas, and the twelve Divine Generals---who appear in ancient fist-fighting stances. 

The movement of Buddhism to China was not only a fateful episode to the history of Buddhism but in the evolution of the martial arts as well. The agent of that migration was the Buddhist monk Boddhidharma, considered the 28th patriarch in a direct line from Gautama Buddha. In the sixth century A.D., Boddhidharma journeyed from India to China, where he introduced the form of Buddhism known as Dhyana (in Sanskrit), Ch'an (in Chinese), and Zen (in Japanese). While in China, Boddhidharma lived at the Shaolin Monastery in Honan Province. He found the monks there solely concerned with achieving spiritual enlightenment and negligent of their physical health. In fact, they were sickly and fell asleep during \emph{zazen} (seated meditation). As a member of the \emph{kshatriya} (warrior class) as well as a monk, Boddhidharma was very well versed in the fighting arts and understood the interdependence of mental, physical, and spiritual health. He introduced a series of eighteen exercises (the ``eighteen hands of the Lo-han'') to the monks for the improvement of their health and for their protection against dangerous forces. These exercises became the basis of Shaolin Temple boxing, which, along with other varieties of Chinese boxing, later influenced the development of the fighting arts in Japan, Korea, and Okinawa.

A second line of development in the liberal martial arts of Asia derives from another Chinese religious tradition, that of Taoism. Tai chi chuan (Grand Ultimate Boxing) was evolved to combine certain forms of Shaolin boxing with an emphasis on breathing and inner control based on Taoist breathing practices and medical lore. According to the most prevalent account of the origins of tai chi, a Taoist monk of the late Sung Dynasty (twelfth or thirteenth century A.D.), Chang San-feng, created the thirteen basic postures of tai chi as bodily expressions of the eight trigrams of the ancient text \emph{I Ching}, and the five basic elements of ancient Chinese cosmology. Somewhat later, a schoolteacher named Wang Chang-yueh is believed to have linked those postures in a continuous sequence of movement that formed the disciplinary core of the tai chi training program. 

Yet another set of innovations in the martial arts took place in Japan following the rise of the samurai class after the tenth century and the introduction of Zen Buddhism there in the twelfth century. From this time the culture of \emph{bushido}, the ``way of the warrior,'' developed gradually from ideas drawn from Buddhism, Confucianism, and Shintoism. Samurai training included unarmed combat, the use of weapons, literary subjects, and training in Zen Buddhism, which provided the courage to face possible death every day. Following the unification and pacification of Japan during the Tokugawa Shogunate, many samurai adapted that Buddhist strain to transform the martial arts from illiberal to liberal uses, vehicles for training that emphasized the spiritual development of participants. 

After the suppression of the samurai under the Meiji regime in the latter part of the nineteenth century, new martial arts were specifically created as forms of liberal training. This was the same period, incidentally, when Yang Lu-Chan for the first time taught tai chi publicly, in Beijing; until then it had been a secret heritage carefully guarded by certain elite Chinese families. In Japan a number of masters sought to revive the old \emph{bushido}-Zen ethic by creating new forms that were non-lethal in intent and designed to provide personal growth and spiritual uplift. In 1882, Jigoro Kano, an educator proficient in ju-jitsu, founded the first Judo Institute in Tokyo. The change from ju-jitsu to ju-do exemplifies, in terminology and practice, the self-conscious transformation of the martial arts from lethal weapons to means of self-development. The suffix ``jitsu'' means technique; ju-jitsu was, thus, a technique for inflicting serious damage on an opponent. The suffix ``do'' means ``way.'' It derives from the Chinese Tao, and in Japanese has connotations related to the outlook of Taoism. More fully, ``do'' means the way to enlightenment, self-realization, and understanding. As conceived by Jigoro Kano, judo---literally, the gentle way---adapted the best techniques from jujitsu, eliminated the harmful ones, and modified others so they could be practiced safely. As practiced by Kano and his followers, the aim of judo is to perfect oneself by systematic training of the mind and body so that each person works in harmony with others. 

Comparable developments took place a little later with other arts. Around 1905, when karate was introduced from Okinawa into mainland Japan, the symbol \emph{kara} (signifying ``Tang,'' or ``Chinese,'') was reinterpreted by invoking another meaning of the word \emph{kara}: ``empty.'' This was to allude not only to the idea of fighting with empty hands-without weapons-but also to the notion of ``emptiness'' in Zen, that is to say, emptiness of mind, mind like a mirror or water that reflects without distortion, and thus to connote the ideals of selflessness, austerity, and humbleness. Later, this philosophic component was stressed by adding the suffix ``do,'' and some of the preeminent schools now refer to themselves as teaching karatedo---that is, the way of life centering on the ``empty hand.''

In the early 1920s, when experiments to revive liberal learning began to flourish in the United States, a gifted master experienced in all the traditional Japanese martial arts, Morihei Ueshiba, evolved a new system which he called aikido. In this art, he created a program for the cultivation of \emph{ki}, the cosmic energy that flows through one's body and is thought to produce health and spiritual uplift, and the capacity for \emph{ai}, harmonious blending, a blending of the forces within oneself, with other people, and with the natural universe. 

A major institutional locus of the martial arts in the Far East today is the educational system. They have come out of the secrecy of monasteries and esoteric cults into the curricula of school systems and the clubs of universities. Although divided into hundreds of specialized forms, which vary considerably in styles, techniques, attitudes, and objectives, what can arguably be called their most rationalized forms---those that involve a coherent approach to dealing with aggressive attacks, a systematic approach to training, and a nontrivial grounding in philosophic beliefs---all pursue the goals of developing a harmonious blending of mental and physical powers, a sensitivity to the responses of others, the virtues of calmness and courage under stress, and some form of an experience of transcendence. 

This survey of the paidetic curriculum in two great traditions suggests, then, that the coupling of the intellectual and the martial arts is no mere trick of the tongue. Indeed, my sketch suggests that developments within the two traditions where each was perfected exhibit some instructive evolutionary parallels. 1) By the sixth century B.C., both in Greece and China, an ideal and a program of liberal training had evolved, which included both intellectual and martial components. 2) In both cases, this ideal became corrupted in later centuries, as combative arts became commercialized in the Hellenistic period, and as Confucian training became bureaucratized. 3) During the sixth century A.D., a liberal component of the older curriculum became codified and institutionalized in those havens of ideal pursuits, the monasteries. 4) In the medieval period, these paedetic curricula became enriched and extended, with the firm establishment of the trivium and quadrivium in medieval universities, and of the arts of kung fu and tai chi chuan in Chinese monasteries. 5) In the late nineteenth century, mainly in the United States and Japan, the ideals of those curricula were revived and propagated in the form of new secular programs of liberal training. 

\section*{III}

Let us proceed now to draw on these suggestive parallels between the intellectual arts and the martial arts to address the set of questions I posed at the outset. To begin with, what is liberal about liberal education? 

The terms in which Westerners are inclined to think about the distinction between education that is liberal and education that is not---or illiberal, or banausic---were classically formulated by Aristotle.\footnote{A more complete response to this question would, of course, have to attend to post-classical formulations of liberality and, indeed, include reference to some of the complexities associated with the idea of freedom.} Aristotle's emphasis was not so much on different kinds of subjects as on the spirit in which a subject is pursued. One may pursue a subject out of necessity, as, for example, learning a trade is necessary to make a living. One may pursue a subject out of utility, as reading is useful because it enables one to find numbers in a telephone directory. Or one may pursue a subject because, as we would say, of peer pressure: It is the fashionable ``thing to do.'' But by definition, to act from necessity is not the mark of being free; to seek for utility everywhere is not suited for men who are great-souled and free; and to follow some pursuit because of the opinion of other people, says Aristotle, would appear to be acting in a menial and servile manner. In contrast to these kinds of motives, Aristotle describes motives for the sort of learning that befits a free person: learning that is undertaken for its own sake, learning that is appropriate for promoting happiness and a good life. And, although Aristotle certainly does not deny the need to study the useful arts, he insists that they should not constitute the whole point of learning: people should study drawing, he urges, not merely to avoid being cheated when buying and selling furniture, but for the liberal reason that this study makes one observant of bodily beauty. 

Now one does not need to turn to the martial arts to catch the import of Aristotle's distinction, although it may be useful to see how readily it can be exemplified in that domain. Illiberal training in the martial arts, then, would be undertaken out of necessity-learning to fight to prevent your community from being enslaved or slaughtered by an invader; or, for utility-to know how to defend yourself in case you happen to get mugged on the street. And there are other kinds of reasons for studying the martial arts that would render the pursuit illiberal-as when one trains because it is the glamorous thing to do, or to impress one's friends. By contrast, when the martial arts are taught and practiced in a liberal manner, it is for the sake of perfecting oneself as a human being and for acquiring a kind of culture that is intrinsically valuable. 

At this juncture, I'd like to share an observation from my own experience with the martial arts that suggests an instructive elaboration on the Aristotelian notion of liberality in education. When I ask persons who have progressed rather deeply into the study of the martial arts why they are doing it, I get an answer that is typically different from what brings people to training in the first place. The reasons why people begin martial arts training are frequently illiberal: for self-defense, or to cure an ailment, or as an outlet for aggression, or because of social inducements. Once they have been training for a while, their motivations usually undergo some subtle change. By the time one has been actively training for a year or two, the reasons tend to converge on a single rationale: I'm training to perfect my masters of the art. What emerges is the sense of a lifelong quest for perfection, wherein each moment is intrinsically satisfying, but the experience is framed as a part of an unlimited pursuit of growth and improved expression. One is reminded of what John Dewey wrote concerning the fine arts: that the works of the fine arts are not merely ends in themselves which give satisfaction, but their creation and contemplation whet the appetite for new effort and achievement and thus bring a continuously expanding satisfaction.\footnote{John Dewey, ``Experience, Nature and Art,'' in \emph{John Dewey on Education: Selected Writings}, ed. by Reginald D. Archambault (Chicago: University of Chicago Press, 1974), pp. 157-65.} What this suggests is a criterion for liberal learning that amends the familiar classical definitions: that education is free and liberating insofar as it involves the quest for mastery of some domain of autonomous forms, forms that are in themselves the free creation of the human spirit. And because that world of form is in principle limitless, this entails a connection with transcendence that is part of the attraction toward liberal learning. 

So I would add, as another component of the generic definition of liberal education, martial and intellectual, that it is an enterprise devoted to \emph{the acquisition of cultural forms for their own sake}. Having said this, my next question is then: what types of cultural forms are most suitable for a liberal program? Once we have distinguished liberal education from the various illiberal forms of training-training for occupations, for solving particular social problems, for transmitting a certain tradition, and the like-there remains the more complicated problem of defining the best content for a liberal curriculum. Different philosophies of liberal education tend to take one of three positions. One position holds that the liberal curriculum should consist of a set of fundamental questions and plausible answers, e.g., those contained in a list of Great Books, or those simply having to do with the nature of the world and man's place in it. A second position holds that the liberal curriculum should consist of the most important structures of organized knowledge, e.g., a basic acquaintance with the principal disciplines of the humanities, social sciences, and natural sciences. A third position holds that the liberal curriculum should represent primarily those basic modes of inquiry and expression exemplified in the disciplines, e.g. how a scientist conducts experiments, or how a poet constructs a sonnet. 

A strong case could be made for viewing each of these as the central principle for a liberal curriculum, and perhaps an even stronger case for a perspective that attempted to represent them all in some balanced way. But what all of them have in common is a stress on what Georg Simmel called objective culture: the external representations of reality and the externalized expressions of meaning that have been created in human history. The true cultivation of individuals, by contrast, takes place in what Simmel called subjective culture: the personal growth that comes about through the internal appropriation of cultural forms. 

The advantage of looking at the martial arts in this context is that such training is almost exclusively concerned with the development of subjective culture-in this case, the competences of bodily movement that enable one to defend oneself in certain stylized ways. There is simply no way to think about the martial arts curriculum without dealing with the ways in which personal capacities of various sorts-perceiving, moving, responding-are nurtured and shaped and perfected. Thus, the martial arts curriculum provides a model for a kind of liberal training in which the principle of the learner's capacities is unmistakably and unavoidably at the center of attention. Although this principle was prominent in early nineteenth-century American notions of liberal intellectual learning, which focused on the goal of mental discipline, it has fallen by the way in contemporary discussions. The principle deserves, I believe, to be revived and viewed afresh as an important basis for organizing the modern liberal curriculum. 

Once we have set the cultivation of subjective capacities as a primary goal of liberal education, however, we must deal with what is perhaps the most complicated of all the questions in the theory and practice of liberal education: What competences should be cultivated? And the obvious answer to that question is another question: What competences are there? Open ten books about competences, and you will find seventeen lists. How does one compose an inventory of competences that can be ordered and ranked so as to provide a set of priorities for liberal education? 

Because I do not think this is a matter that can be resolved definitively for all time, or even that there is a single best way to resolve it at any given moment, I would not look to the martial arts for a model of how to solve it. The problem of identifying a basic list of competences is nearly as intractable in the martial and in the intellectual arts. But martial arts can be helpful on the question, because they illustrate so transparently what the issues are and how one might grapple with them.

Complications here stem from the fact that disciplines emerge historically as concrete traditions, while technical competences can be generalized and used across a variety of disciplines. For example, aikido is a tradition that uses diffused energy, circular body movements, and wrist and elbow throws, while karate relies on concentrated energy, direct body movements, and punches, blocks, and kicks. Yet in both of them a basic movement is the straightforward punch. Moreover, both have a variety of defenses against said punch. So one could imagine a type of competence called punching and responding to punching, the first learnable within either of the two arts but usable beyond, the other requiring some new curricular effort to bring together a wide variety of defenses against punches into a single training program. Just in the last few years, in fact, some martial arts programs have come out with eclectic training approaches not unlike this. 

There is, moreover, a set of generalized competences involved in various ways in all the martial arts that may be formulated as follows: Know oneself; know the other; and observe the right timing in one's response to the other. The idea of self-knowledge in the martial arts is tied to a concern for being centered. One must be in touch with the true center of one's being. One must be unified, the hands with the arms, the limbs with the torso, the body with the feelings and the mind. One must be poised in a state between relaxation and readiness to move-at all times. In the words of the seventeenth-century martial artist, Miyamoto Musashi, ``Do not become tense and do not let yourself go. Keep your mind on the center and do not waver. Calm your mind, and do not cease the firmness for even a second. Always maintain a fluid and flexible, free and open mind.''\footnote{Miyamoto Musashi, \emph{The Book of Five Rings}, trans. by Bradford J. Brown, Yuko Kashiwagi, William H. Barrett, and Eisuke Sasagawa (New York: Bantam Books, 1982), p. 34. In much theorizing about the martial arts, especially in Japan, this principle of subjective centralization, or centeredness, is viewed as a process of concentrating one's attention on the lower abdominal center-the ``\emph{hara}.'' Maintaining this center is viewed as an essential condition of maintaining some mental distance between yourself and events as they unfold around you.}

And yet preoccupation with oneself and one's readiness to act, by itself, would be foolhardy. One must be alert to the dispositions and responses of others no less. One must be aware of the other's balance points, the ``four corners'' of his position in which he is vulnerable. One must sense the precise direction and intensity of an attack from the other. In aikido, the term \emph{ai}, or harmony, refers in an important sense to the idea of blending effectively with the energy of one's attacker. 

Finally, the relational field between self and other must be viewed in dynamic terms, such that the \emph{timing} of one's response to the other is all-important. It does no good to be centered in oneself, and aware of the flow of the other's energy, if one responds too soon, or too late, to the other's attack. So a great deal of emphasis in training focuses on these three areas: how to maintain one's own center; how to perceive and blend in with the energy of the other; and how to time one's responses with pinpoint precision. What this suggests for the intellectual arts is that we might well start looking for basic forms of intellectual competence that are not tied to concrete traditions. In my judgment, this constitutes one of the most exciting challenges facing the academic profession today. Those who are honest about the matter acknowledge that a concrete tradition-sociology, say, or biochemistry-is rarely coterminous with a particular set of competences. I know, for example, that the distinctive skills needed to analyze social phenomena in the economistic terms of rational exchange, or the culturological terms of symbolic codes, are practiced across all of the social science disciplines, including cultural anthropology and economics. The challenge today is to take stock of the enormous changes in all the intellectual disciplines over the last few decades and, for purposes of liberal training, attempt to translate them into competence fields that can be truly defensible components of a future liberal curriculum.

Closely connected to the question of what subjective capacities are to be cultivated in the liberal curriculum is that of the kind of training program best suited to develop those capacities. On this question, I believe, training programs in the martial arts offer much that might be relevant to the design of training programs in the intellectual arts. Of many possible suggestions, let me mention two. 

The first is the stress on \emph{practice}---regular, systematic, unremitting practice. The components of each art must be identified and laid out in such a way as to admit increasing mastery through incessant practice. As Miyamoto Musashi has written: ``Practicing a thousand days is said to be a discipline, and practicing ten thousand days is said to be refining.''\footnote{Ibid., p. 53.} One must practice continuously, and make a lot of mistakes, so that one can be corrected, and be ever on the lookout for ways to refine one's art. 

Second, there is a sequence of phases in developing the practice of one's art. Gradations of rank, marked by a succession of tests that examine clearly defined levels of competence, form a crucial part of the training. Beyond that, there is a kind of progression, common to all arts, that I would call the road to the transcendence of mere technique. One begins by self-consciously practicing a certain technique. One proceeds slowly, deliberately, and reflectively; but one keeps on practicing until the technique becomes internalized and one is no longer self-conscious when executing it. After a set of techniques has been thoroughly internalized, one begins to grasp the principles behind them. And finally, when one has understood and internalized the basic principles, one no longer responds mechanically to a given attack, but begins to use the art creatively and in a manner whereby one's individual style and insights can find expression.\footnote{A parallel formulation of this progression appears in the classic treatise on tai chi chuan by Wang Chung-Yeh: ``From the stage of familiarity with the techniques comes the stage of a gradual understanding of the inner strength, and from the stage of understanding of the inner strength comes the state of spiritual illumination. However, without going through prolonged and serious practice, it is impossible to reach ultimate enlightenment.'' Cited in Tem Horwitz and Susan Kimmelman, \emph{Tai Chi Chu'an: The Technique of Power} (Chicago: Chicago Review Press, 1976), p. 78. }

Notions like these seem to me enormously suggestive for training programs in the intellectual arts. As one of their possible implications, I would stress the importance of some specialization as an essential component of a truly liberal education. There is simply no way to acquire any art to the point where it becomes truly effective as a means of advanced personal growth without the intensity of involvement that requires years of work and progressive mastery. Whether the capacity in question is knowing how to interpret an ancient text, or how to perform chemical experiments in the lab, or to formulate and analyze a problem of public policy, an enormous amount of practice is required in order to be able to progress in some field from techniques to principles to expression (and, indeed, if you will, to develop a sense of personal groundedness \emph{and} sensitivity to the objects \emph{and} knowledge of how and when to time interventions). That is the rationale, I believe, for including concentration programs as an integral component of a full curriculum in liberal education. 

\section*{IV}

I want now to discuss the question of the relationship between liberal and utilitarian learning. The rhetoric of liberal educators vacillates between two apparently contradictory positions. On the one hand, we say that liberal training is a good in itself, superior in worth to those illiberal pursuits that are merely practical. On the other hand, we often say that a liberal education is really the most practical of all. Is this just double-talk, somewhat like saying: I never borrowed your book, and besides, I returned it to you last week?  

Perhaps; but let us look at the martial arts once more to see if some clarification of this matter can be found. In the martial arts, the question of practical utility is always right at hand. In training dojos one often hears an instructor make some offhand reference to what might happen in real situations---''on the street,'' as they say. Yet nothing could be more clear-cut than the difference between an applied training program in self-defense and a liberal curriculum in the martial arts. If you want to acquire some immediate skills for the street, I would say: Don't take up one of the martial arts, but take a crash eight-week course in self-defense; just as I would say, if all you want is a job as a lab technician or an interviewer in a survey research organization, take a crash vocational course in those areas. Yet there is, I believe, a higher practical value in the liberal form of self-defense training. By proceeding to the point where one has mastered the basic principles of the art of self-defense, one has acquired resources for responding to a much wider range of threatening situations and a readiness to respond that flows from basic qualities of self-control, calmness, and courage that one has internalized as a result of years of dedicated training. It certainly would be advantageous to combine some techniques of practical self-defense with a liberal martial training---remember that Aristotle, after all, advocated that training in useful arts be combined with liberal training---but then the former are enhanced by being grounded in a broader conception of the principles of direct combat. The argument may proceed similarly in regard to the liberal intellectual arts: by learning, not merely the specific facts and techniques of a particular subject-matter but its most basic principles and methods, and by understanding these as exemplified in a range of fields, one has gained capacities that enable one to respond intelligently and independently, critically and creatively, to the conditions of a complex and rapidly changing environment, the kind of environment in which all of us are now fated to spend our lives. This is like the ideal that Pericles attributed to the free citizens of Athens: ``To be able to meet even variety of circumstance with the greatest versatility---and with grace.''\footnote{Cited, interestingly enough, in A. Westbrook and O. Ratti, \emph{Aikido and the Dynamic Sphere} (Rutland, Vermont and Tokyo, Japan: Tuttle, 1970), p. 87.}

The last question I want to raise in this comparative exercise may be put as follows: Isn't there something basically immoral in this program for liberal training? Doesn't it focus too much on the individual at the expense of the community? What's worse, couldn't it simply set people up---by training them in the arts---to carry out amoral or even vicious purposes? No matter how much the arts are glamorized, do they not only amount to sets of technical skills that can be put to evil purposes? And if my argument that liberal training produces a higher form of utilitarian competence is sound, then does it not follow that the person with an advanced liberal education has the capacity to be more evil than others?

Certainly this is a question that can never be far from the mind of those training in the martial arts. Indeed, the old masters in Asia were often very selective about whom they allowed to train with them, for they feared the consequences of putting their lore into the hands of those who might use these very potent powers for destructive purposes. In Japanese culture there is in fact a social type associated with that negative possibility-the \emph{ninja}. The \emph{ninja} is precisely one who has mastered martial techniques but puts them to selfish or destructive purposes. And I must say, before we liberal educators take too much pride in offering a wholly blameless product, that we must come to terms with the possibility of creating intellectual \emph{ninjas}---people who are very adept indeed in the manipulation of linguistic and mathematical symbols, and other intellectual capacities, and use them in the service of the basest opportunistic motives and even for destructive purposes. 

To say this is to raise the most fundamental issue of all about the liberal arts: the need for an ideological framework in which they find some ethical grounding. Precisely because the immoral potentialities of martial arts are so transparent, this question is harder to dodge. It is answered forthrightly by ethical formulations associated with the educational programs of all those martial arts I would call liberal today. In a manual of tai chi chuan, for example, one reads: 

The technique of self-defense \ldots{} implies a coherent vision of life that includes self-protection. The world is viewed as an ever-changing interplay of forces. Each creature seeks to realize its own nature, to find its place in the universe. Not to conquer, but to endure. The assumption is that there are hostile forces. One can be attacked by animals, by angry or arrogant people, or just by the forces of Nature, within and without. In the human world, attack is verbal and emotional as often as it is physical. The most subtle and manipulative struggles are the ones of which we are the least conscious. But the prescription for survival is always the same-integrity. [In the martial arts] this is more than a moral adage, it is a physical actuality.\footnote{Horwitz and Kimmelman, pp. 64-65.}

The practice of aikido is suffused by the kind of ethical vision embodied in these words by its founder, Morihei Ueshiba: 
\begin{itemize}
\item Understand Aikido first as \emph{budo} and then as a way of service to construct the World Family. 
\item True \emph{budo} is the loving protection of all beings with a spirit of reconciliation. Reconciliation means to allow the completion of everyone's mission. 
\item True \emph{budo} is a work of love. It is a work of giving life to all beings, and not killing or struggling with each other \ldots{} Aikido is the realization of love.\footnote{Kisshomaru Ueshiba, \emph{Aikido} (Tokyo: Hozansha, 1974), pp. 179-180.}
\end{itemize}

\section*{V}

As college educators face the need to develop a fresh rhetoric for liberal education, a rhetoric responsive to the enormous changes undergone in recent decades by the academic world and the global environment, we may do well to seek the insights and suggestions that can come from stepping outside our customary universe of discourse on the subject. This is a process we are familiar with from the numerous instances of cross-fertilization among the intellectual arts and disciplines. The foregoing essay at comparison has explored one such channel of cross-fertilization, with the following results: 

\begin{enumerate}
\item We have raised the question of the difference between liberal and illiberal learning. The experience of the martial arts suggests that one principle of the liberal program might be formulated as the cultivation of free cultural forms for their own sake.
\item We have asked about the kinds of cultural forms appropriate to a liberal program. The martial arts exemplify for us a neglected type of culture, that which concerns the perfection of the capacities of human subjects.
\item We have asked about the types of subjective cultivation that constitute a plausible inventory. The martial arts clarify for us the problem of distinguishing between concrete traditions and general technical capacities. 
\item We have asked about the character of training programs appropriate to develop such capacities. The martial arts exemplify for us the significance of practice; of a phased program of development, from techniques to principles to expression; and of the need for specialized work to develop any capacity through that curriculum. 
\item We have asked about the relation of liberality to utility. The martial arts exemplify the way in which liberally acquired powers are of especial utilitarian value in a complex and changing environment. 
\item We have asked about the moral justification of liberal training. The martial arts provide models in which those questions are resolved through being linked to an ethical worldview. 
\end{enumerate}