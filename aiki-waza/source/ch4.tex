\chapter[The Many Dimensions of Aiki Extensions (2003)]{The Many Dimensions of Aiki Extensions}

Standing before a sculpture of the Greek god Apollo---a torso only, without head, arms, or feet---Rainer Maria Rilke was overcome with emotion. This fragment, of a god who represented order, harmony, and civilization, radiated a luminous energy that held him in thrall. Beholding the archaic torso, the poet tells us:

\begin{verse}
\begin{flushright}
\emph{\ldots{} da ist keine Stelle \\
die dich nicht sieht. Du musst dein Leben \"{a}ndern.} \\
\ldots{} there is no place at all \\
that isn't looking at you. You must change your life.
\end{flushright}
\end{verse}

Entering the aikido dojo, I see the head of an old man with a white beard. There is something in his look, and in the attitude of the people who practice there in his name, that holds me in thrall. Wherever I go in the dojo, I feel: there is no place this man is not looking at me. And I imagine I hear him say: \emph{Du musst dein Leben \"{a}ndern}. You must change your life.

If I shall have entered the dojo for the first time, I will not have this experience. More likely, I have begun to practice this Japanese martial art of self-defense for a particular personal reason: to gain streetwise confidence, improve my health, impress old friends, meet new friends, who knows what. It is only after I have practiced for a while that the spirit of O'Sensei takes me in, and that my reasons for going there begin to change.

And slowly, I come to realize: what we are working on is not an art, not a set of techniques to accomplish something, but a \emph{practice}, a way of being and acting. Nor is this practice concerned with war, but about promoting \emph{Peace}. Strictly speaking it is not Japanese: its roots are in ancient India and China; it is cultivated and refined in dozens of countries \emph{all over the world}. Nor is it about self-defense, really. Aikido helps one to \emph{transcend the self}, not to firm up the ego. And it is not about \emph{being defensive}, but about connecting with and neutralizing aggression. O'Sensei was a prophet who sought to deal with the chaos and strife of the modern world by promoting order, harmony, and civilized conduct.

This view of aikido could not have been expressed more directly than by the title of a book by Andr\'{e} Protin published in Paris in 1977, \emph{Aikido: une art martiale, une autre mani\`{e}re d'\^{e}tre} (\emph{Aikido: A Martial Art, an Alternative Way of Being}). If aikido does indeed represent \emph{an alternative way of being}, then once we come under its spell, we become mindful of an injunction implicit in every moment of our practice: \emph{du musst dein \textbf{Leben} \"{a}ndern}. We begin to understand what the Founder meant when he said, ``Aikido is not about moving your feet, it's about moving your mind.'' And how, when he saw advanced students teaching it like some sort of athletic activity he said sadly, like a forsaken prophet, ``What they are doing is okay, but that is not what I do.''

O'Sensei reportedly was serious when he claimed that he wanted aikido to function as a medium for bringing peace to the world community; he wanted us to experience the world with compassion and equanimity, and to extend our energy outward in all we do. If that is his message, then are we who follow the practice he created not obliged to consider what we can do to change our lives in that direction?---in everything we do, including our work and social lives?

Several years ago, I became aware of several aikidoka who were struggling to do just that, by taking aikido out of the conventional dojo setting. Most of them were doing so in isolation, unaware that anyone else was following that path. I thought there might be as many as two dozen aikidoka so engaged---using aikido ideas and movements to alter the ways they would practice therapy, or teach, or run a business, or resolve conflicts---and clapped to see if they wanted to connect with one another. Before long, some two hundred aikidoka in seventeen countries had clapped back. The result is Aiki Extensions, and it is now my pleasant task to tell our story.

\section*{Aiki Extensions: The Three Modalities}

One way in which we extend aikido practice outside the conventional dojo setting is to provide such practice in settings that are closer to where the participants actually live. This can take place in high schools, churches, hospitals, detention centers, recreation centers, or anywhere else that such practice is approved and safe conditions are present. For example, Steve Ives of San Anselmo, CA, has offered regular aikido classes in the San Rafael Youth Center; members of Aikido Harmonia teach aikido to 7-to-14-year-old in a center for children from the \emph{favelas} (slums) of S\~{a}o Paolo, Brazil; S\o{}ren Beaulieu has worked with teen-agers in central city high schools in Philadelphia; and Martha Levenson teaches in middle schools in Seattle. 

A second modality of extension work is the use of selected exercise and movements to convey certain ideas. Practitioners might ask novices to experiment with different physical responses to attacks to experience how the \emph{attacker} feels when the response is counter-attack, or acquiescence, or a neutral off-the-line response. Or they might have students experiment with tight vision and soft vision, to observe the bodily sensations associated with each, and to experience the difference that relaxing the face and eyes makes in the scope of one's visual awareness. Or they might have executives feel the difference by moving a tight restraint with tight versus relaxed muscles.

The third modality is to use aikido ideas purely in non-physical forms. This modality has sometimes been called ``verbal aikido.'' The late Terry Dobson pioneered this sort of work. Among AE members, Aviv Goldsmith has implemented what prove to be powerful exercises of ``verbal aikido.'' For example, he has the group form a standing circle, facing inward. Each person takes a turn being in the center ('uke'). Facing each person around the perimeter in turn, uke receives an insult/negative statement of some sort, acknowledging it with a simple ``thank you.'' In a second, integral round of the practice, the same format is followed by having each participant receive compliments/positive statements.

In a variant of this technique, I emphasize the notion of reframing. First we practice, on the mat, experiencing the difference, when responding to an attack, between perceiving it as threatening, in a defensive state of mind, or as energizing, in a welcoming state of mind. Then I ask them to carry out this exercise non-verbally---with a room-mate, an acquaintance, a work partner, etc.---and write a short report of the reframing experience. Students often report major changes in the quality of the relationship.

\section*{Aiki Extensions in Work with Individuals}

Creative aikidoka have devised a rich repertoire of techniques for conveying insights about centering and how to relate harmoniously with others.

\subsection*{Body Work and Somatic Education}

Much discourse in the teaching of aikido concerns the process of staying centered and re-centering oneself. This theme was verbalized in the teachings of Koichi Tohei sensei, who talked about keeping ``one point.'' Relatedly, he also tied this process to bodily relaxation and correct posture. These aiki teachings converge with some major developments in Western somatic education, including F. T. Alexander's work on correcting posture, Ida Rolfs work on structural integration, and Moshe Feldenkrais's work on functional integration.

One of the earliest aikidoka to sense the affinity between aikido practice and Feldenkrais's work was Paul Linden, who developed a modality of somatic education he calls Being in Movement\textregistered. One point of departure for this work was the awareness of what a difference it makes in one's stability when grabbed if one bends one's head or not. Linden's work utilizes a number of directives to improve posture, breathing, and related somatic functioning. The set of practices Linden evolved have been used effectively in treating cases of paralysis, stress disorders following physical or sexual abuse, and severe backaches, and for promoting pain-free computer work and athletic functioning. 

Through a system of aikido-inspired practices she calls Conscious Embodiment, Wendy Palmer has developed a series of bodily practices that enable students to enhance intuitive capacity and to identify different modes of experiencing mental attention. Thus, they gain awareness of distinct attentional states (dropped, open, and blended), which serve specific purposes, while they become aware of other attentional states (contracted, ellipted, and split) which are inherently dysfunctional. Palmer employs awareness of one's responses to being led by the hand in different ways to elicit understandings about separation and connection. Her repertoire includes practices that expand understanding of the dynamics of fear, empower the self through becoming more centered, and engage inquiry about ethical choices. 

The line between bodywork and psychotherapy is thin to nonexistent. Assignment to one or the other category is often arbitrary, if not counterproductive. Asperger's syndrome (AS) offers one challenge that conspicuously involves both dimensions. Martha Levenson offers aikido practice as therapy to children who suffer from the debilitating social and physical disorder. She has found that through aikido, AS children find creative ways to develop social skills and integrate sensory input, while becoming successful in physical activity. 

\subsection*{Psychotherapy}

Numerous aikidoka are professional psychotherapists---more than three dozen in our list of members. Charlie Badenhop has created a practice he calls Seishindo\textregistered, which integrates with aikido various modalities of psychological growth, including NLP and Ericksonian psychotherapy. Hanna and G\"{u}nther Buck have had success in utilizing aikido-based techniques in clinical work with children, adolescents, and adults who suffer from Attention-Deficit-Hyperactivity-Disorder, and in helping managers in leading positions who often suffer from shadow symptoms of emotional regulation and self-monitoring problems. Scott Evans has taught aikido to groups of disturbed adolescents in a psychological treatment center, in the course of which the participants improved noticeably in their management of anger, control of anxiety, and relief of depression.

Tim Warneka adapts aiki techniques in clinical work with physically and/or sexually aggressive children and adolescents at an outpatient treatment center. Coming to believe that the degree of awareness concerning affect is directly correlated to the degree of awareness concerning somatic states, he has drawn on Paul Linden's work to create somatic, verbal, and combined exercises for this population. Thus, with adolescents who are ``up against'' the legal system for their offenses, he might have them stand up and push against a brick wall as a way to demonstrate the level of force they were trying to push against. This would lead to talk about ways to get around the brick wall instead of trying to GO THROUGH IT and then help the teen identify ways to blend and enter and do \emph{tenkan} with their present situation. 

In work with substance abusers, Steve Schuh has used an ``aiki-focused'' counseling approach to help recovering people face their addictions. In-group and individual therapy sessions, Steve has used simple aikido techniques to demonstrate recovery principles including how to ``blend'' with obstacles on the path of recovery. Learning how to center and breath properly is paramount in reducing stress, a key component in many relapses to substance use. Steve helps patients to physically experience handling anger and other difficult emotional states by having them pair up and do blending exercises. By learning how to connect physically with a partner who represents a negative emotional state, the patient regains a locus of control over the ``roller-coaster'' ride of emotions that surface in recovery. Steve has also designed and implemented a wellness component in a substance abuse residential treatment center that features aikido exercises and partner practices.

In \emph{Dynamic Counseling} (1994), Jim Lee compiled exercises with themes from Morita Therapy, Naikan Therapy, and other mind-body modalities as well as aikido. Jim draws on ki development ideas to train therapists to ``join,'' ``lead,'' ``connect,'' and ``maneuver'' clients for more harmonious outcomes. His exercises include: Being Centered in Relationships, Feedback and Centering, Aligning and Moving with Gravity, and Mind and Body Are One. 

\subsection*{Education}

Aikido affords a number of techniques that benefit academic work, including the ways students read and write, how they and the instructor relate to each other, and how they relate to one another in the classroom. Jim Lee has applied aikido methods to test students on the final exam of a counseling skills course: in randori style, students were ``attacked'' randomly with orders to perform particular counseling techniques called out by group peers. Jim Lee has applied aikido methods to test students on the final exam of a counseling skills course: in randori style in groups of 8. Students took turns being in the middle and were ``attacked'' randomly by reading client statements with orders to perform particular counseling techniques called out by Jim.

Aiki ideas assist the learning process in extra-academic settings as well. Fiona Kelty uses aiki techniques to assist blind people in Dublin, Ireland, to deal confidently and effectively with help---and hindrance---from strangers. 

When teaching my class on Conflict Theory and Aikido (the syllabus is included here as Appendix A) I treat the academic classroom itself as a dojo. We consider the difference between collaborative and competitive learning, and explore what it means to read a text, write a paper, converse with others, and take exams in an aiki manner. I ask students to consider their internal sensations from time to time, and use movements in the class to illustrate or explore certain concepts. On the mat, we use more expansive techniques to illustrate concepts dealt with in the classroom such as social distance, dynamics of escalation, and reciprocal priority.

\section*{Aiki Extensions in Work Within and Between Groups}

For a long time, Aiki Extensions' work has invigorated the area of organizational and human resources development, bringing fresh resources to questions of leadership and coaching, conflict management, team development, and personal mastery. Pioneers like Terry Dobson and Victor Miller led workshops on conflict management for business executives during the 1980s. Chris Thorsen and Richard Moon created Quantum Edge, an aiki-inspired consulting enterprise that focuses on leadership development and change management in corporate settings. Tom Crum founded Aiki Works, and teaches aiki extensions ideas in management seminars on leadership skills, personal vision, development and change. His popular book, \emph{The Magic of Conflict}, emphasizes the creative ``push'' conflict management gets by trained aiki responses: centering, connecting, and openness to change (and has been extended to work with children in \emph{Your New Conflict Cookbook}, with Judy Warner). Richard Strozzi-Heckler's Institute offers seminars and in-house projects on how to apply aiki principles to organizations and human resources management. He has recently anthologized pieces by twenty-two authors which explore ways that somatics and aiki practices can enhance creativity in the workplace, \emph{Being Human At Work: Bringing a Somatic Intelligence to Your Professional Life}. A number of AE members in Germany and Poland provide aiki-based consulting groups. 

A small library of books and models has emerged in this area, including \emph{Leadership Aikido} (O'Neil 1997), \emph{Corporate Aikido} (Pino 1999), and \emph{The Randori Principles -- The Path of Effortless Leadership} (Baum \& Hassinger 2002). These provide materials for courses in schools of business that present the systematic transfer of aiki principles to organizational settings. At the University of Augsburg Peter Schettgen teaches such courses using ``Aikicom,'' i.e., aiki communication for solving verbal disputes through centering, grounding, reframing, and using verbal analogies to the physical irimi-tenkan movement (see his \emph{Der allt\"{a}gliche Kampf in Organisationen} [\emph{Everyday Conflicts in Organizations}], 2000), while at Georgia State University in Atlanta, George Kennedy teaches graduate students in business aikido-based techniques of managing conflict. 

This modality of Aiki Extensions work was exemplified by AE founding member Philip Emminger, whose business enterprise reaped great benefits and profitability from adapting aiki methods into his managerial approach, which included holding center with the presence and awareness of a martial artist, yet blending compassionately---and seeing the fulfillment of the needs of others as a benefit to the whole. When a management consultant once approached Philip to hire the consulting firm that he worked for, to adopt their conventional, competitive approach, the aiki-based alternative so impressed the agent that the consultant left his job and came to work for Phil!

\subsection*{Mediation}

Almost by definition, the field of mediation is a natural for aiki practitioners. Donald Saposnek broke fresh ground in this area with his paper on using aikido in family therapy. His book, \emph{Mediating Child Custody Disputes}, which has become the classic text in its field, includes a chapter in which aikido diagrams represent ways of reducing conflict in disputes over child custody. Rod Windle has devised imaginative aiki techniques, including the use of jo, to mediate a wide range of civil and domestic disputes, and conflicts with schools.

In the international theatre, Chris Thorsen and Richard Moon have used aiki principles to aid peace processes. In Bosnia, Moon led peace-building work with a group of young people from the various factions in the conflict, while Thorsen carried out similar assignments in Cyprus. By teaching mediators and organizational leaders how to operate with the power of openness and listening, Thorsen and Moon have helped restructure systems so that they will operate more harmoniously and experience less conflict both internally and externally.

Dual American-Israeli citizen Jamie Zimron works with Israelis and Palestinians in Israel and in the US, teaching aiki principles of ``Peaceful Power'' as part of the Mideast peace process. In 1997 she helped found the Israel Women's Martial Arts Federation, which brings Palestinian girls and women into Jerusalem for training conferences. Despite the ongoing war and media emphasis on violence, Jamie reports that many people engage in non-violent conflict resolution efforts and co-operative educational and business projects, and that aikido is practiced all over Israel, as well as in Egypt, Jordan and other Arab countries. Her dream is to work with aikidoka throughout the Middle East to create an international peace dojo, Dojo Salaam Shalom

\subsection*{Law Enforcement and Public Safety}

Aikido has long been used in the training of policemen. Yoshinkan aikido has been taught to Tokyo riot police since 1955. Aikido of diverse schools has been taught to Law Enforcement officers in several countries, including Australia, Canada, Poland, and the Philippines. 

Officer Matthew Little of the Chicago Police Department's Education and Training Division has been involved in the training of military and police personnel for over a decade. He applies the principles and doctrine of Aiki not only directly as defensive tactics techniques, but also for principle-based firearms and tactical training. This aiki-based principle-driven training methodology allows officers to resolve violent conflict in a calm and appropriate manner, increasing officer safety and lessening the need for use of debilitating or deadly force.

Three years ago, Richard Heckler introduced a Martial Art program into the U. S. Marine Corps using aiki principles. Dojos have been established in every Marine base in the world, and all current personnel and recruits are required to participate in the program. Heckler envisioned this program both as a way to enhance the effectiveness and ethical comportment of marines, and as a kind of character training that would stand them, and their society, in good stead after discharge. Since its inception, reports continuously come in about how incidents of drunkenness, brawling, drug abuse, and domestic violence have gone down and morale has risen in cases where Marines have been engaged in regular practice of the art. Last year, the results of the Marine Corps Martial Art Program were presented to an appreciative audience at a conference of Marine Commandants from all over the world.

\subsection*{Youth Outreach}

An area that is just starting to be developed involves a more proactive approach to extending aiki practice to young people outside conventional settings. For several yeas now, Bill Leicht has headed a Bronx Peace Village/Dojo, where fundamentals of aikido, conflict resolution, meditation and council circle are taught to help inner city children how to live non-violently in high-violence areas. [A slide show on this project was shown after this talk; copies can be ordered for \$10 through Aiki Extensions, via the same method as for payment of dues and donations.] In Chicago, a Greater Chicago Aikido Youth Project coordinated three different projects for youth, with an eye to reaching out into all high schools in the area. In Providence, RI, aikidoka Michael Werth helped organize a kata-a-thon to promote awareness of martial arts training for nonviolent objectives. Dr. Victor la Cerva has transformed his public health work into a campaign for violence prevention. Working for the state of New Mexico, he makes the rounds of high schools with his interactive message of aiki-based alternatives to violence, a message also conveyed in publications, including Pathways to Peace: 40 steps to a less violent America.

\section*{Extending Aiki Principles in Symbolic Work}

\subsection*{Theatre, Dance, Music, and Spirituality}

In a dojo built inside a professional school for dance, music, and theater near Munich, Martin Gruber teaches Aikido for Actors, as a way to enhance their resources for dealing with scenic demands as well as promote physical and mental training. Working with actors, dancers and singers in northern California, Pamela Ricard uses aiki-based techniques to help performers stay `present' and thereby maintain moment-to-moment physical, emotional, and mental awareness in order to create believable characters. Through theatrical practices of creating and developing characters in imaginary scenarios, actors learn to identify with, feel compassion and empathy for another person's point of view---someone for whom they might not otherwise feel any affinity. She accompanies this training with some grounding and centering practices---to help them tolerate the discomforts of conflict so they can stay present more skillfully.

Bill Levine, a jazz pianist and film composer working in Hollywood, experiences aikido as a time-based art, similar to music and dance, which contains improvised phrases of energy. He speaks of playing and composing musical phrases, from beautiful/smooth (spiraling) to dynamic/sharp (entering), more effectively when he applies the discipline, wisdom, and compassion cultivated from the practice of aikido, and of how aikido has enabled him to viscerally feel varying degrees of harmonic tension as sound moves around a tonal center, analogous to the ``hara'' (center) in aikido.

Jack Susman has found considerable connections between the mysticism of aikido and the mysticism of Judaism. Both in the Shinto-based tradition of kototama and in the kabbalah, the fundamental views of the systems are set forth in a form that is often paradoxical, usually unintelligible, and always surprising. One fascinating connection is in their respective theories of creation: both use a symbol of exhalation to explain the origin of the cosmos.

The activities I've just described represent a small fraction of work going on in many countries by aikidoka who are affiliated with Aiki Extensions, not to mention many hundreds more who are not. For a complete list of members and their activities, see the web site link at http://www.aiki-extensions.org/affiliates/, where you may also find links to the various AE members mentioned in these remarks. The network is growing, the work is deepening, and there is no reason not to believe that the aiki spirit may accumulate substantial momentum in the years ahead.

\section*{Connecting the Links of Aiki Extensions}

These areas of application require a good deal of specialized training. Normally, professionals in one domain would have little or nothing to say to those in others. Nevertheless, the fact that all of them are aikidoka, seeking to manifest different dimensions of the Aiki Way, might lead one to think that sooner or later they could develop valuable understandings to exchange with one another. 

The work of Jos\'{e} Roberto Bueno in Brazil begins to suggest some openings of this sort. To begin with, Bueno organized a program to bring young people from the \emph{favela} to an after-school center for regular classes in aikido taught by volunteers. At the same time, he also teaches aikido to members of an upscale business consulting firm, Amana-Key. Thanks to his own personal networking, the employees of Amana-Key who practice aikido in a small dojo there have become interested in the \emph{favela} project, to the extent that some have become sponsors of the children in the \emph{favela} center and a few have even reached a point of aikido training where they can serve as volunteers in the youth outreach center as well. 

And suddenly, the possibilities seem endless. Ask me, I think it is what O'Sensei would have wished. 