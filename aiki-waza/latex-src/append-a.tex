\chapter*{Appendix A: Conflict Theory and Aikido Course Syllabus}
\addcontentsline{toc}{chapter}{\hspace{1.5em}Appendix A: Conflict Theory and Aikido Course Syllabus}

\newcommand{\syllabustitle}[1]{\noindent{\textbf{\large{\uppercase{#1}}}}}
\newcommand{\week}[2]{\textbf{#1 --- \emph{#2}}}

\footnotesize

\begin{center}The University of Chicago\end{center}
Sociology 20115/30115 \hfill{} Donald Levine, instructor\\
Autumn 2010 \hfill{} Dan Kimmel, course assistant

\begin{center}
\large\textbf{\uppercase{Conflict Theory and Aikido}:\\
The Aiki Way to Managing Conflict}
\end{center}

\noindent This course has three aims:

\begin{enumerate}[noitemsep]
\item to expand knowledge about social conflict and ways of dealing with it;
\item to explore bodymind reflexivity as a resource for cultivating self and understanding others;
\item to introduce the practice of aikido, as a means for dealing with conflict and for cultivating selves.
\end{enumerate}

\syllabustitle{I. Somatic Awareness and Aikido} \\
 \\
\week{M, 9-27}{Aikido and Bodymindfulness} \\
\indent Connections among body-feelings-mind-spirit \\
\indent \indent Mind \textgreater{} body (mental framing) \\
\indent \indent Body \textgreater{} feelings (postural affects) \\
\indent \indent Mind \textgreater{} body \textgreater{} energy (force of intention) \\
\indent \indent Body \textgreater{} perceptions \textgreater{} mind \textgreater{} spirit (relaxations) \\
\indent Modes of learning through aikido \\
\indent \indent attending to bodymind experience  \\
\indent \indent collaborative inquiry with training partners \\
\indent Conditions of intense bodymind learning \\
\indent \indent dedicated place (\emph{dojo}), uniform (\emph{dogi}), attitude (\emph{shugyo})  \\
\indent \indent disciplines of respect: \\
\indent \indent \indent 1) for Place; 2) for self; 3) for partners; 4) for teachers; 5) for conversations; 6) for Truth \\
\week{W, 9-29}{The mat-dojo as a place for learning the martial Way (budo)} \\
\indent Respect in the dojo (\emph{li} / \emph{rei}) \\
\indent Elements of martial practice: \\
\indent \indent Stance (\emph{kamae}; \emph{hanmi}). Changing \emph{hanmi}. Stepping and pivoting.  \\
\indent \indent Sitting (\emph{seiza}). Rolling. Falling (backward). \\
\indent Lowry, \emph{Sword and Brush}: ch 1, ``\emph{Do}''; ch. 3, ``\emph{Keiko}''; ch. 35, ``\emph{Rei}''; ch. 40, ``\emph{Shugyo}'' \\
\indent ``The Dojo and its Culture'' (Selected Readings: A) \\
\week{F, 10-1}{The mat-dojo as a place for learning about one's self} \\
\indent Centering experiences. Testing for centering and ki extension. \emph{Katate-dori kokyu-nage}. \\
\indent ``Why Aikido?'' (Selected Readings: A) \\
\indent Lowry, \emph{Sword and Brush}: ch. 11, ``\emph{Ki}''; ch. 26, ``\emph{Hara}''; ch. 27, ``\emph{Uke}'' \\
 \\
\syllabustitle{II. Inquiry into Social Conflict} \\
 \\
\week{M, 10-4}{Broaching the study of social conflict} \\
\indent Broaching the study of anything (stasis theory) \\
\indent Commonplace questions: \\
\indent \indent \indent Is it? \\
\indent \indent \indent Why study it? \\
\indent \indent \indent Defining it: \\
\indent \indent \indent \indent How define it? \\
\indent \indent \indent \indent Why define it that way? (cf. ``essentially contested concepts'') \\
\indent \indent \indent How study it? \\
\indent Simmel, ``The Problem of Sociology'', ``On Conflict'', ``Competition'' (SR:A) \\
\indent Boulding, Conflict and Defense, pp. xv-xvii, 1-6 (e-reserve) \\
\indent Coser, \emph{The Functions of Social Conflict}: preface, introductory, props. 1, 2, 4, 5 \\
\week{W, 10-6}{Aikido practice as collaborative inquiry} \\
\indent Attacking sincerely and falling safely (\emph{ukemi}). \emph{Katate-dori kokyu nage} (wrist-grab, breath-throw).  \\
\indent Lowry, ch. 39, ``\emph{I: Intent}'' \\
\week{F, 10-8}{Investigating conflict on the mat} \\
\indent Types of conjoint training. \emph{Katate kosa-dori ikkyo} (cross-hand grab, first takedown). \\
\indent Lowry, ch. 5, ``\emph{Kata}''; ch. 23, ``\emph{Te}''; ch. 24, ``\emph{Kamae}'' \\
 \\
\syllabustitle{III. Elements of Conflict} \\
 \\
\week{M, 10-11}{Motives, means, and consequences in conflictual interaction} \\
\indent Boulding, \emph{Conflict and Defense}, pp. 7-18 (e-reserve) \\
\indent Coser, \emph{Functions}, prop. 3 \\
\indent Gelles \& Straus, ``Determinants of Violence in the Family,'' Intro, sec. 1-4 (e-reserve) \\
\week{W, 10-13}{Elements of martial engagement} \\
\indent Distance and timing (\emph{ma-ai}). \emph{Katate-kosa-dori ikkyo}. \emph{Yokomen-uchi waza}. \\
\indent Simmel, ``Distance'' (SR:A);  \\
\indent Lowry, ch. 15, ``\emph{Hyoshi}'', ch. 36, ``\emph{Ma}'' \\
\week{F, 10-15}{Types of attack and types of response.} \\
\indent \emph{Tai no henko} (three forms), \emph{Musubi} (joining), \emph{Katate-dori kokyu-nage} \\
\indent Saotome, ``\emph{Musubi}'' (SR:A) \\
\indent Kriesberg, \emph{Constructive Conflicts}, ch. 3 \\
 \\
\syllabustitle{IV. Escalation / De-Escalation: Personal Sources} \\
 \\
\week{M, 10-18}{Paradigms of escalation} \\
\indent Kerr, ``Chronic Anxiety and Defining a Self'' (SR, B)  \\
\indent Boulding, \emph{Conflict and Defense}, ch. 2 (e-reserve) \\
\indent Coser, props. 5 \& 6 (60-72) \\
\week{W, 10-20}{Escalatory effects of different responses to attack} \\
\indent Counter-attack. Acquiescence. Moving off the line. \\
\indent Taking a hit. Offline with connection. \emph{Munetsuki kokyu-nage}. \\
\indent Leonard, ``Taking the Hit as a Gift'' (SR:B) \\
\week{F, 10-22}{Escalatory effects of different forms of aggressive expression} \\
\indent Expressing antagonism in a relationship. \emph{Munetsuki kokyu-nage}. \\
 \\
\syllabustitle{V. Escalation / De-escalation: Social Sources} \\
 \\
\week{M, 10-25}{Social mechanisms for controlling escalation} \\
\indent Coleman, \emph{Community Conflict} (SR: C) \\
\indent Parsons, ``Racial and Religious Differences as Factors in Group Tensions'' \\
\indent Kriesberg, \emph{Constructive Conflicts}, ch. 6, ``Escalating Conflicts'' \\
\week{W, 10-27}{Bodymind mechanisms for controlling dispositions to escalate} \\
\indent Positive receptivity. Reframing. \emph{Munetsuki kote-gaeshi}. \\
\week{F, 10-29}{Embodied responsive techniques for controlling escalation} \\
\indent \emph{Munetsuki waza}. \emph{Irimi nage waza}.  \\
\indent Kriesberg, \emph{Constructive Conflicts}, ch. 7, ``De-escalating Conflicts'' \\
 \\
\syllabustitle{VI. Violence} \\
 \\
\week{M, 11-1}{Dimensions of violent engagement} \\
\indent Biological: Lorenz, \emph{Aggression}, Intro, ch. 13;  \\
\indent Wrangham \& Peterson, \emph{Demonic Males}, chs. 3, 4, 6 (7, 9 optional) \\
\indent Social-Psychological: Scheff, ``Male emotions/relationships and violence: a case study'' (e-res) \\
\indent Social: Coser, ``Some Social Functions of Violence'' (SR:B) \\
\indent Cultural: Sorel and Fanon, selections (SR:B); Fromm, ``Anthropology'' (SR:B) \\
\week{W, 11-3}{Training for courage} \\
\indent Entering the line of attack. \emph{Marubashi} training. Katatedori irimi-nage. \\
\indent Lowry, \emph{Sword and Brush}, ch. 15 ``\emph{Shin}'', ch. 19 ``\emph{Fudo}'' \\
\week{F, 11-5}{Staying centered under stress} \\
\indent Multiple attacks (\emph{randori}). \emph{Irimi waza}. \\
 \\
\syllabustitle{VII. Nonviolence} \\
 \\
\week{M, 11-8}{Conceptions of non-violent engagement} \\
\indent James, ``The Moral Equivalent of War'' (e-reserve) \\
\indent Bondurant, \emph{The Conquest of Violence: The Gandhian Philosophy of Conflict}, 3-41 \\
\indent Rosenberg, \emph{Nonviolent Communication}, selections (SR:B) \\
\week{W, 11-10}{Training for Calm Control} \\
\indent \emph{Mushin and fudoshin}. Reframing. \\
\indent Leggett, ``\emph{Mushin}'' (SR:B) \\
\indent \emph{Shomen-uchi ikkyo}, \emph{omote}. \\
\indent \textbf{\uppercase{Reframing assignment distributed}} \\
\week{F, 11-12}{Leading the mind} \\
\indent \emph{Shomen-uchi ikkyo}, \emph{ura}.  \\
\indent An Aiki reconstruction of the Cain and Abel story: \\
\indent Heckler, \emph{In Search of the Warrior Spirit}: 84, 134-40, 197-203 (e-reserve) \\
\indent \indent  \\
\syllabustitle{VIII. Mediation} \\
 \\
\week{M, 11-15}{Third parties in the management of conflict [with guest Craig Naylor]} \\
\indent Simmel, ``The Nonpartisan and the Mediator'' (e-reserve) \\
\week{W, 11-17}{Mental states of conflict mediators (classroom)} \\
\indent Kriesberg, \emph{Constructive Conflicts}, ch. 8, ``Intermediary Contributions'' \\
\indent Kagan, \emph{Adversarial Legalism: The American Way of Law} 9-17 (SR: B) \\
\indent Hovering awareness (\emph{zanshin}). \emph{Happo undo}. \emph{Yokomen-uchi shihonage}. \\
\indent Lowry, \emph{Sword and Brush}, ch. 32, ``\emph{Zan}'' \\
\indent \textbf{\uppercase{Reframing assignment due}} \\
\week{F, 11-19}{Position and timing in mediating conflict} \\
\indent Conflicts with multiple parties \\
\indent Folberg, \emph{Resolving Disputes: Theory, Practice and Law}: 95-97, 204-207  \\
\indent Saposnek, ``Mediating Child Custody Disputes'' \\
\indent Kerr, ``Chronic Anxiety and Defining a Self'' (SR, B) \emph{Reprise}. \\
 \\
\syllabustitle{IX. Aikido and Kindred Disciplines? Other Aspects of Conflict?} \\
 \\
\week{M, 11-22}{Review of Readings and Discussion of Final Paper} \\
\indent Levine, ``Ki Development and Aiki Training'' (handout) \\
\week{W, 11-24}{Keiko Review} \\
 \\
\syllabustitle{X. The Aiki Way} \\
 \\
\week{M, 11-29}{Classic formulations} \\
\indent Ueshiba, \emph{The Spirit of Aikido} \\
\indent Quotations from Aikido Masters: Ueshiba, Saotome, Doran (SR:B) \\
\week{W, 12-1}{A Paradigm of the Aiki Way (handout)} \\
\week{F, 12-3}{Optional keiko review} \\
 \\
\syllabustitle{XI. Putting It to the Test} \\
 \\
\week{M, 12-6}{Testing waza} \\
\week{W, 12-8}{Final papers due} \\