\chapter[The Masculinity Ethic and the Spirit of Warriorhood (2006)]{The Masculinity Ethic and the Spirit of Warriorhood in Ethiopian and Japanese Cultures\footnote{Revised version of paper presented at the World Congress of Sociology, Brisbane, Australia, July 8, 2002. Research Committee on Armed Forces and Conflict Resolution, Session 4: The Military and Masculinity. Published in \emph{International Journal of Ethiopian Studies} Vol. 2, No. 1 \& 2, 2006]}}

In modern social science, the notion that human behavior has instinctual bases has been downplayed. Over the past century, anthropologists and sociologists have marched under the banners of Sumner's dictum that ``the folkways can make anything right," Dewey's advice that there are ``no separate instincts," and Benedict's formula that cultures pattern behavior. 

In one area, however, some resonance to the notion that genes affect destiny has persisted: the phenomenon of human aggression. To William James's suspicion before World War I that ``our ancestors have bred pugnacity into our bone and marrow, and thousands of years of peace won't breed it out of us" (James 1910, 314) Freud added his theory, in the inter-war years, that humans are animated by an inexorable stream of destructive energy fueled by a Death Instinct. The thesis of innate aggressiveness was advanced, and linked to gender, with the work of ethologists Konrad Lorenz (1966) and Nikolaas Tinbergen (1968), who analyzed the adaptive significance of aggression among human males. Revising Freudian instinct theory from such an ethological perspective, psychoanalyst John Bowlby argued that

\begin{quote}
virtually every species of animal shares its habitat with a number of potentially very dangerous predators and, to survive, needs to be equipped with behavioural systems resulting in protection. \ldots{} When members of the group are threatened, the mature males, whether monkeys or men, combine to drive off the predator whilst the females and immatures retire. (Bowlby 1969)
\end{quote}

More recently, comparative primate studies have marshaled robust evidence to show that the human genome resembles most closely that of the chimpanzee, and the latest research on chimpanzees shows an unmistakable proclivity for violence by males against males of other groups (Wrangham and Peterson 1996). In addition, genetic research has begun to zero in on the chromosome that may account for such aggressivity. For example, on chromosome \#17, there is a coding region which affects the distribution of serotonin throughout the body, and the extent of that distribution in turn affects the disposition to commit violent actions (Ridley 2000, 168).

Be all that as it may, the fact remains that as with any other such genetically-based traits, cultures shape inborn dispositions variously; in the felicitous words of a dictum pronounced, I think, by P. B. Bedawar, ``Instinct proposes \ldots{} culture disposes.'' Even if humans possess a genetically based behavioral system that tends toward physical aggression, cultural systems process that disposition in various ways---by glorifying it, polishing it, or suppressing it. They determine whether or not and how aggressive inclinations get molded into an ideal of what it means to be a ``real man.'' In many cultures, the ideal of virtuous manhood stands to impose strict control over aggressiveness, which thereby becomes subordinated to a more pacific model of what it means to be a mature human being and citizen. In the ancient Hellenic period, for example, the virtue of a man, \emph{ar\^{e}te andros}, was equated with the capacity to manage one's household and the affairs of the city well. When a man's personal obligations conflicted with his civic obligations, it was simply a mark of manliness (\emph{andreios}) to resist the requirements of the law (Adkins 1960, 226-32).\footnote{Reconfigured in an aesthetic mode, this antinomian undertone to masculinity persists in present-day Crete.  According to Michael Herzfeld, in \emph{Poetics of Manhood}, the Cretan village ethos foregrounds a studied skill in playing at being a man, through deeds that strikingly speak for themselves; in any domain such performative excellence ``can gain from judicious rule breaking, since this foregrounds the performer's skill at manipulating the conventions'' (Herzfeld 1985, 25).} Within the Jewish tradition, being a real man was associated with the assumption of full moral responsibility, either in the mode of altruistic generosity symbolized by the Yiddish term \emph{Mensch} or in the mode of manly self-control sometimes described as the mark of modern Jewish manliness at the turn of the last century (Boyarin 1997). Closer to this mode of manly self-control, Alexis de Tocqueville described Americans as tending to esteem ``all those quiet virtues which tend to regularity in the body social and which favor trade'' (Tocqueville 2000, 621). Insofar as the American conception of honor includes the virtue of courage, it does not have to do with martial valor. Rather, the type of manly courage

\begin{quote}
best known and best appreciated is that which makes a man brave the fury of the ocean to reach port more quickly, and face without complaint the privations of life in the wilds and that solitude which is harder to bear than any privations, the courage which makes a man almost insensible to the loss of a fortune laboriously acquired and prompts him instantly to fresh exertions to gain another. (622)
\end{quote}

Not surprisingly, however, Tocqueville contrasts this ethos with that of a feudal aristocracy ``born of war and for war,'' in which ``nothing was more important to it than military courage. It was therefore natural to glorify courage above all other virtues'' (618). Indeed, societies in which warriorhood figures prominently tend to feature combative excellence in their ideal of masculinity and to give it a high place in their scheme of values. This was surely the case in the Archaic Age of Greece, when the most powerful words of commendation used of a man, \emph{agathos and ar\^{e}te}, signified above all military prowess and the skills that promote success and war (Adkins 1960, 31-32). 

\section*{Martial Values in Ethiopia and Japan}

This pattern was also conspicuously evident in two of the oldest continuous national societies, Ethiopia and Japan, where, for most of the past millennium, there existed expectations of continuous readiness for martial combat. In both countries, military prowess offered a royal road to prestige and legitimacy, and the ascendance of powerful warrior-lords and their retainers lifted martial values to a dominant position. It was these two nations alone that successfully defied European imperial ambitions: Ethiopia over Italy in 1896, Japan over Russia in 1904.

In both nations, esteem for warriorhood was not just a matter of according high prestige to military men; it involved the diffusion of martial attitudes, virtues, and ambitions throughout the population. That diffusion came about through very different routes. In Ethiopia, it took the form of promoting widely the inculcation of combative dispositions. This feature so impressed the first European scholar of Ethiopian civilization, Job Ludolphus, that he described Ethiopians as ``a Warlike People and continually exercis'd in War \ldots{} except in Winter, at what time by reason of the Inundation of the Rivers, they are forc'd to be quiet" (Ludolphus 1684, 217). It meant that every able-bodied male who was not a clergyman was assumed to be ready to engage in battle at a moment's notice---armed, skilled, supplied, and transported, all through his own devices. It meant that boys were encouraged to be combative and that as men they were disposed to be fearless in combat. It even meant that, for most of the past millennium, the royal capital took the form of an army camp---``a vast array of tents, arranged in combat-ready formation with the Emperor's tents in the center, flanked and guarded at the front and rear by officers of standard ranks with their entourages'' (Levine 1968, 7). As a result of the prominence of warfare in Ethiopian history, military virtues have ranked among the highest in the Abyssinian value system; military titles have been among the most prestigious in their social hierarchy; military symbolism has provided a medium for important national traditions and a focus for a good deal of national sentiment; and military statuses and procedures have influenced patterns of social organization in many ways (Levine 1968, 6).

In Japan, the hegemony of martial values derived not from universal combat-readiness but from the way in which a military stratum, the \emph{samurai}, came to set the tone of the national culture. This class emerged in the late Heian Period (10-12 C) as a group of military specialists positioned to serve the court nobility. In time they acquired power in their own right by establishing domination over agricultural land and building their own hierarchical political organizations, culminating in a semi-central regime, the shogunate, in the late 12C. The samurai political organization rested on the formation of strong emotional bonds between military masters and vassals upheld by a strict code of honor (Ikegami 1995). In the Tokugawa Period this code was elaborated into a formal code of martial ethics known as Bushido (the Way of the Warrior). The code enjoined such virtues as loyalty, politeness, diligence, frugality, and a constant sense of readiness to die. At this time, the bushi class became more segregated than ever, since membership in it was hereditary and only those within it were entitled to bear arms. On the other hand, the ethos of this class became hegemonic in the society. In contrast to China of the time, the Japanese insisted on retaining a martial spirit as part of the mark of a gentleman (Hall 1970, 82). During the Tokugawa period, it has been said, the samurai ethic came close to being the national ethic, for even the merchant class had become ``Bushido-ized" (Bellah 1957, 98). 

One of the marks of the warrior ethos in both cultures was a disposition to value ascetic hardiness. This is manifest, for example, in the Ethiopian ideal of \emph{gw\"{a}b\"{a}znet}, a symbol for masculine aggressiveness and hardiness. In consequence, Ethiopian soldiers have been noted for great endurance---they climb mountains with ease, march rapidly for distances under heavy pack with light rations, and sleep on a rock. In Japan, similar virtues were the pride of the samurai class, who prided themselves on undergoing great hardships without complaint---for example by undergoing a week of arduous training outside each year in the dead of winter (\emph{kangeiko}). 

Another mark of the warrior ethos has been a pronounced concern about honor and a sensitivity to insult that numerous observers have found in the psychological profile of both peoples. In Ethiopia, insults traditionally formed reason enough for violent retribution, and continued into the modern era as grounds for instigating legal proceedings. In Japan, a cult of honor became the subject of extensive elaboration, leading samurai to cultivate an extreme sensitivity to insult (Ikegami 1995). Countless legends idealize the person who secures revenge against someone who impugns his honor. 

Finally, although Ethiopia and Japan have traditionally held esteemed the just warrior, in both cultures there existed a type of antinomian hero who carried masculine aggressivity to a high pitch. In Ethiopia this took the form of the \emph{shifta} (from \emph{shefete}, to rebel), a retainer who rebelled against his chief and withdrew, often hiding in the hills, to fend as an outlaw (Levine 1965, 243-4). In modern times, this word has in fact acquired the primary meaning of a bandit. Many stories depict the shifta in idealized terms. The first modern nation-building emperor, Tewodros II (1855-68), famously began his climb to power as a \emph{shifta}. 

The Japanese counterpart of the \emph{shifta} was the \emph{ronin}, a samurai who left his lord or never subordinated himself to a lord. Here, too, heroic performances by \emph{ronin} form the stuff of legends. And in modern times, the status of outlaw strong man has been taken by the \emph{yakuza}, the bold gangster. In a playful form of this status, Japanese young males in the 1970 and 1980s took up a semi-delinquent lifestyle called Yankee and, combined with prowess on motorcycle, formed \emph{bosozoku} (``violent driving tribe'') gangs in major cities where their ultramasculinity could be flaunted (Sato 1991).

\section*{The Ethiopian Masculinity Ideal: Aggressivity Unbound}

Although both Ethiopians and Japanese construed the ideal of masculinity in ways that provided a strong impetus to warriorhood, one can also identify characteristic differences in how these play out in Ethiopia and Japan. In presenting this analysis, I shall also comment on distinctive institutions that represent a counterbalance to male aggressivity. 

In describing the Ethiopian pattern, I shall rely initially on what for most of the past millennium has been the politically and culturally dominant group, the Amhara, and the terms of their language Amharic.\footnote{Strictly speaking, although the Amharic language was the national political language of Ethiopia from the thirteenth century at least, the term Amhara denoted a local geographic region, and was not extended to the vast population of Amharic speakers until the second half of the twentieth century.  See Levine 2003.} The Amharic term for male, \emph{w\"{a}nd}, not only indicates gender (e.g., \emph{w\"{a}nd lijj}, ``male child''), but also connotes strong emotional approval. To say of someone, \emph{Essu w\"{a}nd n\"{a}w}, ``He is a male,'' is to state more than biological fact; it is a eulogy of virtue, analogous to the American expression, ``He's a real man.'' However, unlike the American concept, \emph{w\"{a}nd-n\"{a}t} does not connote manly maturity and the assumption of adult moral responsibilities. In Amharic, this notion is signified by the term for middle-aged man (\emph{mulu s\"{a}w}). The term \emph{w\"{a}nd} may refer to any age and has nothing to do with moral maturation. Nor does it connote male prowess in heterosexual affairs, for the Amhara attach no particular value to the expression of heterosexual sentiment or the enjoyment of sexuality. In fact, a puritanical attitude toward sexuality in the public realm has the effect of keeping such matters from becoming the object of spoken concern at all; for an Amhara male to boast of his heterosexual achievements would be considered shameful.

The traditional Amhara ideal of masculinity refers primarily to aggressive capacity. The Amhara male likes to boast over his ferocity, his bravery in killing an enemy or a wild beast. Amhara culture provided genres of oral literature for such impassioned boasting, employed before and after military expeditions as well as for entertainment on festive occasions. In the second place, \emph{w\"{a}nd-n\"{a}t} connotes the ability to make little of physical hardship---to live for a long time in the wild, to walk all day long with no food. In short, for traditional Amhara the virtues of the male are the virtues of the soldier.

The Amharic word which represents the virtues of the soldier is \emph{gw\"{a}b\"{a}z}. \emph{Gw\"{a}b\"{a}z} may be translated as ``brave,'' as ``hardy,'' or simply as ``outstanding.'' One of the goals in the socialization of boys was to teach them to be \emph{gw\"{a}b\"{a}z}. This is done in a variety of ways. Amhara boys are early taught to defend themselves with sticks and stones against any outsider who happens to injure or insult them. Tiny boys are trained in mock battles with members of their family. Temper tantrums are regarded positively by the child's parents as a sign that he is \emph{gw\"{a}b\"{a}z}. The norms of violent revenge when someone has taken one's land, harmed one's relative, had relations with one's wife, or spoken a grievous insult are taught to growing boys. Boys of about twelve were wont to prove their virility by scarring their arms with red-hot embers. The Amhara youth develops skill in improvising \emph{shill\"{a}la}, the strident verse that is declaimed in order to inflame the blood of the warrior; and he commits to memory verses which glorify the \emph{gw\"{a}b\"{a}z} warrior and the act of killing (Levine 1966, 18-19).

Warriorhood takes different forms among the two major cultural traditions in Ethiopia, the Amhara-Tigrean and the Oromo or Galla,\footnote{Similar to the way in which ``Amhara'' was extended to represent a much broader population that its original local referent, the term ``Oromo'' has come to designate the entire population of those who speak dialects of the language called Afan Oromo, formerly known as Gallinya.  Even today, a group believed to represent the purest form of traditional Oromo culture refuse to refer to themselves as Oromo, but as Boran.  It has therefore been difficult to adopt a tern that can be used consistently.} as we shall see below. In both cultures, however, the secular identity associated with being a male is tied closely to a man's capacity for combat. Both Amhara-Tigrean and Oromo cultures extol courage the virtues of aggressive masculinity and martial courage in particular. In both societies, boys are trained to be fearless fighters. Men who slay dangerous animals or human enemies are lavishly honored. Special boasting chants are declaimed to shame cowards and incite the brave. Amhara and Oromo verses of this sort often share a close resemblance.\footnote{
\begin{tabular}[t]{ l l }
\underline{Amhara:} & \\
Shellelew shellelew & War cries, war cries! \\
Mindenew shellelew & Of what use is boasting and challenging \\
Baddisu gorade & Unless you decorate your new sword \\
Demun telamesew & With his blood! \\
 & \\
\underline{Oromo:} & \\
Sala buttan dakkutti sala & The sword’s edge on the [shepherd's] apron is shameful \\
Chirriqun durba sala & To spit on a girl is shameful \\
Sala lama batani & After bringing the two edges [of a spear] \\
Lama bachifatani & After ordering two [edges of a spear] to be brought \\
Dirarra diessun sala & The flight from men [enemies] is shameful
\end{tabular}
\flushright{(Levine 2000, 152-3)}
}

This has the effect of informing warriorhood in both traditions with a spirit of enormous daring, bordering at times on foolhardiness. In the modern period, this meant that Ethiopians with arms inferior to the Italians were able to inflict a crushing defeat on that invading force at the Battle of Adwa in 1896. Their spirit was embodied in the refusal of some Ethiopian soldiers to get down in trenches; they insisted in fighting out in the open, as befits a real \emph{w\"{a}nd}. This meant that Ethiopian men were disposed to fight again in 1935 with spears and limited weapons against an Italian enemy now equipped with planes and poison gas.\footnote{It was due to their ``unreasoning offensive spirit,'' an Italian officer wrote in 1937, that Ethiopian troops were easy to defeat by a disciplined modern army (Perham 1948, 167).} It was later reflected in the extraordinary performance of the battalion of Ethiopian troops sent to Korea to fight with the United Nations forces in 1951, a performance that earned them the reputation of being perhaps the most effective military unit of the entire U.N. contingent. 

\section*{The Japanese Masculinity Ideal: Aggressivity Bound}

Although the ideal of courage figures prominently in the Japanese ethic of masculinity, that ethic has come to depart from a notion of raw aggressivity. The Japanese have traditionally referred to those who behave with untamed violence, not as real men, but as barbarians or wild beasts. The attitude toward a man who manifests physical strength alone is just as negative as that toward an effete courtier. Rather, the fully realized masculine character---\emph{otoko no otoko}, a ``man's man''---modifies raw, self-asserting physical prowess in a number of ways. 

To be sure, the earliest professional warriors, of the 8th and 9th centuries, who may represent a distinctive ethnic group who were originally hunters, appeared extreme in their raw violence. However, by the middle ages and continuously thereafter, samurai violence was progressively domesticated, as Eiko Ikegami's \emph{The Taming of the Samurai} (1995) demonstrates so elegantly. 

The conduct of the samurai and of those who emulate the samurai model came to exemplify a quality called \emph{shibui}. As Lebra describes it, 

\begin{quote}
The concept of \emph{shibui} implies an outlook which is practical, devoid of frills, and unassuming, one which acts as circumstances require, simply and without fuss. In baseball, neither the spectacular homerun batter nor the brilliant infielder can really become valuable players unless they acquire this \emph{shibui} quality. Unless the spectacular and the brilliant include in themselves this element of the \emph{shibui}, the technique can never really be called mature. The ever-available ability to go concisely and simply to the heart of what is required \ldots{} the pursuit of high efficiency, shorn of excessive individual technique, neither flashy nor yet dull (Lebra 1976, 20).
\end{quote}

In addition, Lebra writes, man-like behaviors include suppression of the emotions. It is important to be free from lingering attachments, so that one does not hesitate for a second to kick one's wife out if something is found wrong with her. Real men should also not talk too much. One of the best-known commercial catchphrases in recent years is: ``\emph{Otoko wa damatte Sapporo biru}'' (``Men silently drink Sapporo beer''), uttered by Toshiro Mifune, the John Wayne of Japan'' (80, 18, 78).

Beyond such qualities of personal comportment, certain cultural accomplishments formed part of the repertoire of the Japanese male ideal. Japanese samurai were expected to show proficiency, not only in the arts of war (\emph{bu}), but in a number of non-martial spheres that linked with the neo-Confucian notion of personal culture (\emph{bun}). This linkage was represented by an ideal that joined them by means of a compound phrase, \emph{bu-bun}. Proficient calligraphy was the main one. The embodiment of \emph{bu-bun} involved practice with the pen and brush in a manner that evinced unself-conscious and fearless directness. A secondary art was the composition of highly stylized verse, most notably haiku. 

As samurai culture evolved, it also came to experience the martial code in a context formed by overarching ideals of loyalty and devotion to corporate groups. This progressed from impassioned martial loyalty to the household (\emph{ie}) of one's lord, to a sense of loyalty to the samurai status group and its code of honor, to political loyalty to the head of the state (Ikegami 1995). Such loyalty was no less important than courage in defining the ethic of the full Japanese male. Well-known stories describe Japanese retainers undergoing enormous pain and other deprivations to serve their lords, not to mention the countless episodes of \emph{seppuku} (suicide by disembowelment). This ideal of manly courage pertained to the peasants as well as to the \emph{samurai}. A famous tale of peasant protest concerns a 17C villager named Sakura Sogoro who, at the cost of being crucified, brazenly presented a petition from his neighbors to the \emph{shogun} in the tip of a six-foot-long bamboo pole. The traditional text about this episode concludes, ``Truly if you are a warrior, you ought to leave behind a glorious reputation because your name is written down in the records for all posterity'' (Walthall 1991, 75). 

The sacrifice of personal comfort on behalf of corporate goals and organization fed into the Japanese penchant for collective discipline. Before WWII at least, regimentalized patterns of collective action were instilled in Japanese schools. 

In warfare, these ideals promoted distinctive patterns of conduct. The implications of these ideals for patterns of martial conduct were twofold. On the one hand, the notion of subordination of the individual promoted deeds of suicidal daring, most notably in the kamikaze pilots. On the other hand, the ideal of cultivated warriorhood, \emph{bu-bun}, meant that combativeness was traditionally restrained by norms of exaggerated gentlemanly decorum. Even so, a turbulent self-assertiveness that constituted what Ikegami has called ``honorific individualism'' fueled their dispositions to serve.

\section*{Ethiopian and Japanese Warriorhood in Social Context}

Within Ethiopia, however, how the masculinity ideal played out in warfare was further determined by the context of social structure. This variable led to marked differences between the two major ethnic protagonists of modern Ethiopian history, the Amhara-Tigreans, often known as Habesha or Abyssinians, and the Oromo, formerly known as Galla.\footnote{In present day Ethiopia, the term Oromo has become standard for referring to all of the peoples formerly designated as Galla in the Ethiopian chronicles.  Even so, some ``Oromo'' groups today still do not use that term for themselves.  I shall use both terms loosely, depending on the context.  Interaction between the Oromo and the Amhara-Tigreans from the sixteenth century on, I have argued, formed a central dynamic in the evolution of the modern Ethiopian nation-state (Levine 1974).}

The Abyssinian military ethic took the form of a cult of the hero. Personal bravery---not discipline, training, honor, or self-sacrificing loyalty---was the paramount virtue in Abyssinian warfare. The \emph{gw\"{a}b\"{a}z} warrior was rewarded by his chief, praised by the minstrel, and esteemed by the populace. His bravery was ranked according to the fearfulness of the enemy vanquished. Thus, in Menelik's day the fanciest headdress was given to a noble who killed one of the fierce Danakil, a less fancy headdress being awarded to the killer of the tough Raya Galla. Such actions constituted the one area in which personal boasting was permitted and, in fact, institutionalized in the genre known as \emph{fukara}. 

We are indebted to Arnauld d'Abbadie for a firsthand account of the effect of this cult of the individual hero on the orientation of the Abyssinian warrior, in a passage worth citing at length:

\begin{quote}
The type of combat which [the Abyssinian] prefers over all others---because it gives him the most freedom to expand his personality---is that where, due to insufficiency of terrain or other circumstances, the chiefs can engage only a part of their forces. \ldots{} Joyously he throws off his toga to clad himself in some military ornament. \ldots{} He loves \ldots{} to know, finally, that on the hills, behind their drummers who beat out the charge in place, the two rival chiefs and the two armies are following him with their eyes, and that he may at one moment or another, return to his lord and, hurling before him some trophy, tell him, at the end of his war chant: ``There! This is what I know how to do!'' (Abbadie 1868, 313)
\end{quote}

The military organization of the Amhara was highly individualistic. Unlike traditional Oromo, the Amhara did not provide for the collective training of their warriors. Each man was left to learn how to fight by himself and to provide his own equipment. A man could become a ``career'' soldier when he came of age simply by purchasing a shield; or he might prevail upon an established lord to arm him temporarily, with the promise of returning equipment should he leave that lord's service. Similarly, there were no collective provisions for the supply of troops. Each man was left to fend for himself, drawing upon the supply of grain he brought along and whatever booty he could acquire on the warpath; the preparation of his food was left to the wife or servant who accompanied him to battle. 

The conduct of a military operation exhibited a minimum of external constraint and discipline. Chains of command existed with respect to the general direction of troop movements, and the camping pattern was highly structured. But the marching and fighting unit seems to have been, for all practical purposes, the individual soldier and his retainers. Battles were not fought in a disciplined manner; the outcome depended on the sheer number of troops, their state of morale, and the chance of catching the enemy off guard. Except for the large-scale deployment of troops in accord with the customary tactic of envelopment, there was little expectation of subordinating the impulses of individual soldiers to the needs of a ``team''; the prevailing military ethic stressed rather the heroism of the individual soldier and his drive to bring back a cache of booty and trophies (Levine 1965, 262-3).

This pattern contrasted with the pattern exhibited by Oromo warriors. The Oromo went to war, not as proud and self-sufficient individuals, but as members of named collectivities. Raiding and military expeditions were executed by members of the same age set, or \emph{hariyya}. Formed by boys in their late teens by wandering from camp to camp, the age sets were deployed in organized divisions called \emph{chibra}, which collected supplies for the campaign, elected regimental leaders, recruited scouts, and distributed booty. The \emph{chibra} served as fighting units and followed carefully planned battlefield strategy. Where Amhara males fought as individual soldiers, expected to provide their own supplies and capture personal booty, the Oromo derived support, resources, guidance, and morale from their age-mate comrades. Oromo proverbs celebrate the efficacy of massed collective action in waging war.

Beyond that, Oromo were bound to one another deeply through a number of social classes that went through a system of grades generally lasting eight years, a system known as \emph{g\"{a}da}. Often misconstrued as an age-class system, \emph{g\"{a}da} was actually a system based on generational position, in which sons of whatever age entered the system precisely five grades after their fathers. Each \emph{g\"{a}da} class took a turn at serving as the governing class of a particular Oromo society, during which it made the decisions as to when and where military expeditions should be launched as well as when ritual ceremonies should be performed. Oromo males traditionally felt strong ties not only to the general class which they joined but also to a transgenerational solidarity, the \emph{gogessa}, consisting of the classes of their father, their son, their son's sons, and so on. The decisions of a particular ruling class thus had historic implications. The class in power felt obliged not only to avoid the chief misfortunes that befell its ancestors and to repeat its signal successes, but also to set precedents that would benefit its descendants many generations in the future (Legesse, 1973).  

Oromo traditionally observed an injunction to undertake a ritual killing expedition every eight years. The \emph{g\"{a}da} class that undertook the expedition fought not only for itself but also to live up to the reputation of its ancestral \emph{gogessa} classes. In contrast to the repertoire of Abyssinian martial chants, which exclusively glorify the boasting man's own exploits, Oromo also possessed a distinctive genre of boasting songs known as \emph{farsa}, which celebrate the deeds of famous ancestors. The \emph{farsa} are sung to glorify Oromo solidary groups---clans, lineages, age sets, or \emph{gogessa}.

One other important difference should be mentioned, the religious dimension. Although Abyssinian culture put a premium on associating masculinity with aggressive prowess, it nevertheless placed great emphasis on the curbing of aggression through religious teachings and practices. An extensive regime of fasting in Abyssinian Christianity is held to curb man's natural sinful aggressive inclinations. A substantial proportion of the populace---a 17C visitor estimated as high as one-third (Lobo 1984, 178)---have been monks and clergy, and so ineligible to take up arms. Piety in many forms stood to curb the tendency to violence. Among the Oromo, warfare itself was integrated into their religious system. A religious ritual known as \emph{butta} entailed the execution of raiding and killing expeditions every eight years. 

The structuration of masculinity and warriorhood in Japan represented a kind of middle ground between Abyssinian individualism and Oromo collectivism, and also between their respective forms of religiosity. As with Abyssinian Christianity, Japanese Buddhists promoted an ultimate ethic of nonviolence, and supported monastic roles on its behalf. On the other hand, Buddhist temples were among the staunchest bastions of armed defense during the medieval period. Some forms of Buddhism preached the oneness of death and life, and did not regard death as a source of impurity (as did native Shinto). The samurai drew eagerly on Buddhism as a resource to steel themselves against fear of death. 

Institutionally, Japanese warriordom was centered in a complex of patron-client ties, as was the case in Abyssinia. In contrast to the Amhara pattern, however, Japanese patron-client ties were embedded in a named collectivity to which deep loyalty was expected: the household (\emph{ie}) of a lord. This nexus enmeshed the warrior in a corporate grouping, which reinforced a disposition to self-sacrifice on its behalf. Even so, the striving for aggressive self-assertion continued to permeate the samurai outlook. The result, Ikegami notes, was ``two coexisting modes of aspiration in the Japanese elite \ldots{} competitive individuality on the one hand and orderly conformity on the other'' (1995, 335). 

\section*{Historic Consequences}

Differences in the ways in which the traditional cultures of Japan and Ethiopia construe the masculinity ethos in the service of warriorhood represent instructive exemplifications of how ``culture disposes'' what male gender-linked instincts of aggressivity propose. Beyond that, these phenomena may be seen to have had important historic consequences. 

To begin with, differences in the spirit of warfare between Abyssinian and Oromo had, I have argued, important consequences for the making of the modern Ethiopian state. In the course of the Oromo expansions of the 16th and 17th centuries, their advances were rarely checked by Abyssinian troops. This remarkable fact was noted by our most valuable contemporary source, an Amhara monk named Bahrey who wrote a \emph{History of the Galla} in the 1590s. ``How is it,'' Bahrey wondered, ``that the Galla [Oromo] defeat us, though we are numerous and well supplied with arms?'' (cited Levine 2000, 89) 

In accounting for the Oromo victories, I have relied on a clue provided by a statement attributed to Bahrey's contemporary, Emperor S\"{a}rts\"{a} Dingil, who reportedly ascribed the Oromo conquests to their firm determination on going into battle to either conquer or die, and the routs and defeats of the Amhara to the exact opposite disposition. In explaining this difference, I have argued that although both cultures placed enormous emphasis on fearless masculine combativeness, they differed in the extent to which those motivations were activated. 

The Amhara pattern of hierarchical individualism had the effect of making the motivation of individual soldiers contingent on the particular reward structure of a given campaign. Amhara troops fought for personal gain from booty and to be acknowledged and rewarded by their superiors. The presence of the king or lord on the battlefield typically made a great difference in how bravely Amhara soldiers were inclined to fight. If the relevant lord was killed, or if there was no chance of his learning about a soldier's bravery, the latter was likely to feel that there was not much point in fighting. If their lord was defeated in battle, Amhara soldiers often shifted allegiances and went over to another side. If the gains possible from any battle situation seemed too small, they felt no moral compulsion to continue the fight. 

In the Oromo case, by contrast, several factors made the activation of their military ethic less contingent on the particularities of the battle situation. For one thing, killing a man was intrinsically an important accomplishment for any Oromo male who wanted to live a self-respecting life. It enhanced his chances of securing a wife or wives, and not to be married at the appropriate time was considered quite shameful. It gave him the self-esteem associated with wearing the victorious warrior's hairstyle. Beyond that, the Oromo warriors' engagement drew considerable support, we have seen, from the social structures in which it was organized. Consequently, he was inspired to contribute to the corporate success of his fighting division, and to play his part in the drama of Oromo history, as well as to appear a fully competent male in the eyes of his home community. Since he thereby had a set of motivations for battle that were continuously operative and not contingent on the circumstances of the particular battle, the Oromo warrior needed no lord to inspire and reward his particular exploits in battle.

The upshot was that the Oromo not only overran a vast territory inhabited by Amhara and other ethnies, but made their way to the center of the historic kingdom. Their accommodation with indigenous groups with which they came to mingle, and their integration to the national center by intermarriage and vassalage constituted the central dynamic of the emergence of the modern Ethiopian nation (Levine 2000). In particular, they soon came to provide troops for the Ethiopian Crown. Quick to appreciate their valor, S\"{a}rts\"{a} Dingil, for example, deployed Oromo warriors as early as 1580 in missions to defeat rebels aligning themselves with Turks on the Red Sea Coast, and also in expeditions against the Falasha and other Oromo tribes (Conti Rossini, 1907). This pattern made it possible for Oromo troops in substantial numbers to fight alongside Amhara-Tigreans under Emperor Menilek II, who quadrupled the size of the Ethiopian empire, and led a multiethnic army to defeat the Italians in 1896. 

Likewise, in Japan, the samurai ethos played a double role in creating the modern nation-state. Their ethic of shaping conduct through rigorous discipline and subordinating individuals to collective interests worked wonders when transferred to nation building under the Meiji restoration and economic transformation thereafter. The transference of absolute martial loyalty from one's immediate lord to the imperial head of the Meiji state furthered mightily the establishment of a powerful modern nation, one which at Port Arthur in 1904 became the first Asian country to defeat a European army. 

With that achievement, Japan joined Ethiopia to become the only other non-European country to defeat a European army in the final era of imperial expansion. Recognizing this affinity, a number of Japanese citizens showed enormous sympathy with the Ethiopians when they were invaded in 1935, even to the extent of sending them a shipload of swords. Differences in their social structural and other cultural patterns, however, meant that the application of martial dispositions to economic life enabled the Japanese to modernize far more rapidly in both economic and political domains (Levine 2001).

Contrasts in contemporary expressions of these martial dispositions appear as well. On the one hand, mobilization of traditional warrior values on behalf of a strongly centralized modern nation-state led Japan to embark on a program of ruthless military expansion, invading Manchuria and China in the 1930s and imposing severe cruelties on the peoples of East Asia, including China, Korea, Burma, and the Philippines.  By contrast, Ethiopia in the 1930s was a victim of unprovoked invasion by Fascist Italy, pursued through a war machine that rained poisoned gas upon peasants armed with spears. In the postwar era, Japan tended to abstain from international efforts to stem Communist expansion and maintain world peace, whereas Ethiopia, earlier casualty of a dysfunctional system of collective security, played a gallant role in United Nations military actions in Korea and the Congo and, through actions of both Emperor Haile Selassie and her current Prime Minister Meles Zenawi, performed statesmanlike services in mediating major conflicts in Nigeria, Morocco, Somalia, and the Sudan.

A less conspicuous contrast, albeit one no less consequential, appears in the manifestation of this ethic in the civil political sphere. The process of taming of the samurai has continued well into the twentieth century, as traditional martial arts (\emph{bujutsu}) became transformed into disciplines pursued purely for the cultivation of character (\emph{budo}), and finally underwent a revolutionary charismatic transformation into a practice known as aikido, designated by its founder as a way to promote peace and world harmony (Saotome 1989, Beaulieu 2005).

The civilian manifestation of this ethic presents a far-reaching expression of civil discourse in the political arena, albeit one that offers less room for the individualistic assertiveness that could be displayed even in the samurai universe.  The lack of a comparable taming process in Ethiopia has meant that throughout the twentieth century, the assertive martial habitus never disappeared from the governance system. Like all of his predecessors of the past two centuries, the current Prime Minister has had to shoot his way into power, and has publicly boasted of the significance of his guerrilla days in the bush as the schooling of choice for his political career and vocation.\footnote{In an interview in a Tigrinya-language Eritrean quarterly, the Prime Minister expressed his conviction that ``To me quality of life means to be part of an armed struggle \ldots{} I don't think that there is a better life than the life of a combatant.  If I were not a combatant I don’t think I would have been a happy person.'' (Hwyet 11, May 1997)} Once the taming of her traditional warrior ethic gets under way, Ethiopia may well experience a surge of new productivity and cultural achievement. 

\section*{References}

\begin{list}{}{\refstyle}
\item D'Abbadie, Arnauld. 1868. Douze ans de s\'{e}jour dans la Haute-\'{E}thiopie. Paris.
\item Adkins, Arthur W. H. 1960. Merit and Responsibility: A Study in Greek Values. Oxford: Clarendon Press.
\item Beaulieu, S\o{}ren. 2005. ``After O'Sensei: Dynamics of Succession to a Charismatic Innovator.'' Unpublished Master's Thesis, MAPSS program, The University of Chicago. 
\item Bellah, Robert. 1957. Tokugawa Religion. Boston: Beacon Press.
\item Bowlby, John. 1969. Attachment. NY: Basic Books.
\item Boyarin, Daniel. 1997. Unheroic Conduct: The Rise of Heterosexuality and the Invention of the Jewish Man. Berkeley: University of California Press.
\item Conti Rossini, Carlo, ed. 1907. Historia Regis Sarsa Dengel (Malak Sagad). Paris.
\item Hall, John W. 1970. Japan From Prehistory to Modern Times. New York: Dell.
\item Herzfeld, Michael. 1985. The Poetics of Manhood: Contest and Identity in a Cretan Mountain Village. Princeton: Princeton University Press.
\item Ikegami, Eiko. 1995. The Taming of the Samurai. Cambridge, MA: Harvard University Press. 
\item James, William. 1910. ``The Moral Equivalent of War." In Essays on Faith and Morals, 311-28.
\item Lebra, T.S. 1976. Japanese Patterns of Behavior. Honolulu: University of Hawaii Press.
\item Legesse, Asmarom. 1973. Gada: Three Approaches to the Study of African Society. New York: Free Press, 1973.
\item Levine, Donald N. 1965. Wax and Gold: Tradition and Innovation in Ethiopian Culture. Chicago: University of Chicago Press.
\item \_\_\_\_\_\_\_\_\_\_\_\_\_. 1966. ``The Concept of Masculinity in Ethiopian Culture.'' The International Journal of Social Psychiatry 12:1, 17-23.
\item \_\_\_\_\_\_\_\_\_\_\_\_\_. 1968. ``The Military in Ethiopian Politics: Capabilities and Constraints.'' In Henry Bienen, ed., The Military Intervenes: Case Studies in Political Development, 5-34. New York: Russell Sage Foundation.
\item \_\_\_\_\_\_\_\_\_\_\_\_\_. 2000. Greater Ethiopia: The Evolution of a Multiethnic Society. Second edition. Chicago: University of Chicago Press.
\item \_\_\_\_\_\_\_\_\_\_\_\_\_. 2001. ``Ethiopia and Japan in Comparative Civilizational Perspective.'' Passages 3:1, 1-31. 
\item \_\_\_\_\_\_\_\_\_\_\_\_\_. 2003. ``Amhara.'' Encyclopaedia Aethiopica. v. 1, 230-22. Institute of African and Ethiopian Studies, Hamburg University.
\item Lobo, Jer\'{o}nimo. 1984. The Itinerario of Jer\'{o}nimo Lobo. London: Hakluyt. 
\item Lorenz, Konrad. 1966. On Aggression. New York: Harcourt, Brace and World.
\item Ludolphus, Job. 1684. A New History of Ethiopia. Translated by J.P. Gent. London.
\item Mead, Margaret. 1937. Cooperation and Conflict among Primitive Peoples. New York, NY: McGraw-Hill.
\item Morgenthau, Hans. 1960. Politics Among Nations, 3rd ed. New York: Knopf.
\item Perham, Marjorie. 1948. The Government of Ethiopia. London: Faber and Faber.
\item Portal, Gerald H. 1892. My Mission to Abyssinia. New York: Negro Universities Press.
\item Ridley, Matt. 2000. Genome: The Autobiography of a Species in 23 Chapters. New York: Harper Collins.
\item Saotome, Mitsugi. 1986. Aikido and the Harmony of Nature. Boston: Shambhala. 
\item Sato, Ikuya. 1991. Kamikaze Biker: Parody and Anomy in Affluent Japan. University of Chicago Press. 
\item Simmel, Georg. 1903/4. ``The Sociology of Conflict,'' trans. Albion W. Small. American Journal of Sociology 9, 1903/4, 490-525, 672-89, 798-811.
\item \_\_\_\_\_. (1908) 1955. Conflict, trans. Kurt H. Wolff. Glencoe, Ill.: Free Press.
\item Tinbergen, N. 1968. ``On War and Peace in Animals and Man: An ethologist's approach to the biology of aggression,'' Science 160, 1411-18.
\item Tocqueville, Alexis de. (1835) 2000. Democracy in America. New York: Perennial Classics.
\item Walthall, Anne, ed. 1991. Peasant Uprisings in Japan: A Critical Anthology of Peasant Histories. Chicago: University of Chicago Press. 
\item Wrangham, Richard, and Dale Peterson. 1996. Demonic Males: Apes and the Origins of Human Violence. New York: Houghton Mifflin.
\item Zivkovic, Marko. 2002. ``Noble Criminals, Highlanders and Cryptomatriarchy: Poetics of Masculinity in Serbia (and how to get at it).'' Paper presented at conference on Balkan Masculinities, University College London, 7-8 June 2002.
\end{list}