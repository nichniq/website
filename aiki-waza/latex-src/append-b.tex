\chapter*{Appendix B: Student Reflections on the Aikido Course Experience:
 Update Autumn 2010}
\addcontentsline{toc}{chapter}{\hspace{1.5em}Appendix B: Student Reflections on the Aikido Course Experience}

Each year that the Conflict Theory and Aikido course has been offered, students have submitted notes---in journals, responses to prompts, or spontaneous letters to the instructor---which describe some key learning experiences from the course. Over the past two dozen years, these notes show considerable continuity, particularly regarding the themes of enhanced ways to manage conflict situations; in handling personal stress; in gaining awareness of body-mind connections; and in ways to gain an understanding of an unfamiliar culture. In the later versions of the course, other themes became salient. These include enhanced appreciation of formal structures of etiquette, and awareness of the process and value of improved social connections.

\begin{center}****************\end{center}

1. Expectably, learning new ways to deal with conflict appeared as one fundamental outcome, inasmuch as aikido introduced techniques of deescalation and mediation.

\begin{quote}
One does not seek to block an attack, but to blend with it, one does not push or pull their partner but instead connects with them and guides them. All of these techniques are intended to help both parties understand one another and see the conflict from the other's perspective, thereby allowing the conflict to be resolved in a manner beneficial to both parties.

\begin{center}**********\end{center}

The on-the-mat practice of Aikido has entirely transformed the way I see conflict, relationships, leadership, and life-energy by showing me that the harmonious middle-ground always exists; all it takes is a few breaths and a relaxed mind.
\end{quote}

2. Like their predecessors, students in the Autumn 2010 class applied their aikido training to other aspects of their lives. Aikido opened up new ways to handle the enormous stress that the academic requirements of The College places on its students:

\begin{quote}
Another consequence of centering yourself that has been very useful, both inside and outside of class, is its stress-relieving effects. \ldots{} [T]hrough aikido classes, I have learned to focus on breathing from my center. Not only has this been useful for relaxing myself during mat sessions when I find myself confused in an exercise, but also with dealing with the stress of classes, work, and any other demanding situation. Focusing on breathing from your center grounds you; it calms you down and allows you to see things clearer with an unclouded mind.
\end{quote}

Another student shared a similar experience, writing:

\begin{quote}
Whenever I was feeling particularly stressed this quarter, I made sure to take a second to sit down, calm myself, and re-find my center. Originally, I tried this on a whim when I was feeling very overstressed, and to my surprise, it worked! I felt a lot better. Ever since then, I've been using our technique of centering whenever I feel anxious and once I finish, I find that I am able to tackle whatever problems were bothering me with a much calmer demeanor and a rejuvenated enthusiasm. \ldots{} [T]hese two practices translate very easily to the world outside of the mat. They become applicable, and eventually necessary, to one's everyday life.
\end{quote}

3. As before, the dojo rituals and codes of conduct created an atmosphere that initially confused some students and often provoked resistance. More recently, students talked about coming to find meaning in such prescriptions as the course progressed, and about the value of relating to them with an open mind. One student, who identified himself as a ``free spirit,'' found the dojo structure difficult, but wrote later:

\begin{quote}
Once I realized that the structural confines liberate Qi flow in the dojo, however, I was able to fully accept Aikido. In essence, by accepting Aikido, I turned it into my partner. I regained my center and pivoted into this new world to see things from its perspective. What Don Sensei said, that the physical practice inculcates the theory into the self, is starting to ring true to me.
\end{quote}
 
Some students described growing up in an environment that emphasized advancement by competition and besting other persons. The non-competitive, mutually respectful ethos of aikido challenged them to rethink those earlier norms. From fighting with his partners when they did not comply with his expectations, one student wrote, he began to be influenced by the calm, respectful demeanors of other aikido practitioners:

\begin{quote}
At the dojo, instead of focusing on causing my partner to fall, I started to focus on the precision of my own technique and stance. This awareness of my own center and my personal development has naturally evolved into a sincere attitude towards everything in my daily life. \ldots{} I realized that by following the norm [of the dojo], I actually started to develop real respect towards the other classmates and the dojo tradition. Such attitude of respect has extended beyond the dojo. As I have become more respectful during the Aikido training, I also naturally became more respectful to people I met everyday.
\end{quote}

4. A fundamental tenet of aikido is the creation of a ``connection'' between uke and nage. Without this connection, techniques will not work and partners will stay locked in trying to overpower one another. Like all beginners, students in the class focused exclusively on crude, physical connections. With time, the physical connection became refined, and what became even more important was mental: ``a connection of intentions.'' Aikido thereby became more than just a way of warding off an attack; it opened a new type of understanding. Training led the students to look at one another in a new light:

\begin{quote}
Aikido is showing me that it's a perfectly realistic aspiration for us to turn adversaries into partners.
\end{quote}

This type of awareness extended to enabling students to improve personal relations outside of the class. By learning to take challenges, problems, and negative feelings as opportunities to learn, students began to view the tense relationships in their lives as paths towards growth:

\begin{quote}
Aikido forced me out of my comfort zone. I had no choice but to get closer to others. Indirectly, I guess, this helped me open up a bit more with people. 
\end{quote}