\chapter[Clashes or Dialogue Among Civilizations (2011)]{Clashes or Dialogue Among Civilizations\footnote{Original version was presented in the session, ``Clashes versus Rapprochement,'' at Comparing Modern Civilizations: Pluralism versus Homogeneity. A Conference in Honor of Shmuel Noah Eisenstadt. Jerusalem, November 2-4, 2003. The paper had the benefit of comments from Adam Kissel, McKim Marriott, Nilesh Patel, and Rabbi Arnold Wolf and was published as: ``Civilizational Resources for Dialogic Engagement?'' In \emph{Comparing Modern Civilizations: Pluralism versus Homogeneity}, ed. Eliezer Ben-Rafael. Boston: Brill. This revised version was published in the \emph{Journal of Classical Sociology} 11, No. 3. August 2011: 313-26 as ``The Dialogue of Civilizations: An Eisenstadt Legacy.''}}

\section*{Abstract}

\begin{quote}
The thesis of a ``Clash of Civilizations,'' famously voiced by Samuel Huntington in 1993, draws support from selected social science generalizations and the fact that all historical civilizations organized around core beliefs and values condemned outsiders. This thesis can be challenged by showing that civilization are internally complex, including elements that also develop non-exclusionary themes; and by specifying a human need for ``dialogue'' driven by compresent needs for attachment and differentiation. The historic emergence of those inclusionary subtraditions by looking at the cases of Gandhi in India, Ueshiba in Japan, and a number of historic and contemporary figures in the Abrahamic civilizations of Judaism, Christianity, and Islam.
\end{quote}

\noindent
In 1993 the late Samuel Huntington advanced a claim that the bipolarized world of the latter 20th century would yield inexorably to clashes among civilizations. This alarm caught many social scientists by surprise. In the early 1990s literate opinion lingered under the glow of the Soviet collapse and savored a sense that world consensus behind liberal democracy and capitalism stood to preclude future ideological clashes. The view that the array of culturally diverse historical societies would ``converge'' on a single common constellation of modern society---a principal tenet of the first two centuries of sociology---seemed reconfirmed.

Shmuel N.~Eisenstadt figured prominently among those who had long challenged the convergence thesis. His noted conception of ``multiple modernities'' seemed to point to a world future in which gross cultural differences would perdure and if anything grow more intense. His perspective might thereby have been assumed a priori as fielding an argument consistent with the central claims of the Huntington thesis. This essay will demonstrate, however, that in virtue of Eisenstadt's championing of two other ideas---the complexity of historic civilizations and the potentialities of dialogue---that assumption must be challenged.

Global developments since the early 1990s could be said to have corroborated Huntington's claim. As a rough indicator of that denouement, consider John Mearsheimer's recent summary: in the first years after the Cold War, many Americans evinced profound optimism about the future of international politics, but since 1989 the United States has been at war for a startling two out of every three years, with no end in sight, such that the public mood has shifted to an aching pessimism (Mearsheimer 2011, 17). To be sure, it is a large leap from the frequency of post-Cold War international clashes to an assumption about the clash of civilizations. Warfare among contemporary societies stems from many sources: growing competition over increasingly scarce resources like land, energy, and water; struggles for political control and economic hegemony; and hostile reactions to economic insecurities and rapid social change.The management of such conflicts depends largely on the restraint of statesmen, negotiations among political stakeholders, and the attitudes of their followers. 

Even so, the salience of those polemogenic factors does not rule out the thesis of a deeper-lying clash of civilizations. This sweeping claim deserves to be addressed in its own right. 

\section*{In Support of the Huntington Thesis}

The Huntington thesis holds that \emph{diverse civilizations are marked by core symbolic complexes that ultimately stand in irreducible conflict}. This claim draws support from three truths.

Ever since William Graham Sumner (1906) provided the language to say so, social scientists have affirmed that all human groups manifest \emph{ethnocentrism}. This designates a syndrome marked by an exaggerated view of a group's own virtues; a pejorative view of others; a relation of order, law, and industry among members of the in-group; and a relation of predation against out-groups. Related to these elements is a tendency to exaggerate the differences between in-groups and out-groups. The universality of this pattern can be linked in part to the ways in which it satisfies at once two of the most powerful human needs: the need for attachment and the need for differentiation.\footnote{These needs, as recent social neuroscience has demonstrated, are hard-wired in the human species (Smith and Stevens 2002).}

Second, as systematic studies on the matter have shown, the more complex and technologically advanced a society, the stronger its level of ethnocentrism is likely to be(LeVine and Campbell 1972). 

Third, ethnocentric beliefs become fortified when intertwined with imperatives that stem from strong cultural mandates. Certain of these mandates derive from the work of elites who have produced transcendent ideals for reconstructing worldly relations, ideals that were elaborated in what have been called the Axial civilizations (Eisenstadt 2003, I, chs. 1, 7).

The great civilizations, consequently, have tended to defend and extend their respective domains through glorified ethnocentric processes involving conquest, conversion, and assimilation of those outside the pale. In Greco-Roman civilization, for example, Hellenes came to disparage outsiders who were ignorant of Greek language and civilization, thereby uncivil and rude. Calling them barbarians (\emph{barbaroi}) encouraged the Greeks to conquer, enslave, and colonize others who were deemed culturally inferior. This conceit continued in Roman times, as Roman citizens justified their extensive conquests of alien peoples (\emph{barbari}) in ways that coerced them into adopting the Latin language and their religious beliefs. In the case of European civilization this pattern found its denouement in the ``\emph{missione civilatrice}'' whereby Italian airplanes rained poisoned gas on shoeless Abyssinian peasants armed with spears, and Nazi armies attempted to expand their notion of a superior German culture throughout Europe. The Greek/barbarian paradigm can be found in all other major civilizations. Its omnipresence underlies the plausibility of the clash of civilizations thesis. 

The pejorative distinctions one associates with the great civilizations include, alongside the Hellenic distinction between Greek and barbarian, the dichotomies of Hindu/\emph{mleccha}, Chosen People (\emph{am segulah})/gentiles (\emph{goyyim}), Christian/pagan, \emph{umma}/\emph{fakir} (infidel),and \emph{nihongo}/\emph{gaijin}. Each of those dichotomies derives from certain core values in each civilization, values that implant criteria used to justify disparagement if not aggression against others. If, in fact, those values represent hegemonic notions that subordinate all beliefs and norms in their respective civilizations, then there would indeed be grounds for adducing theoretical support for the Huntington worldview. 

\section*{Challenges to the Huntington Thesis}

Nevertheless, the Huntington thesis appears vulnerable when both of its key assumptions are subjected to question. The first views civilizations as monolithic formations, organized around a coherent core of animating beliefs and values. The second holds that the most likely interactional form in which serious differences tend to get aired is that of combat. These assumptions simply do not hold up under critical examination. Few thinkers have had the erudition and imagination to provide as much substance for those critiques as did Shmuel Eisenstadt. 

The first critique was voiced eloquently by Edward Said, when he discounted the Huntington view of civilizations as ``shut-down, sealed-off entities that have been purged of the myriad currents and counter-currents that animate human history, and that over centuries have made it possible for that history not only to contain wars of religion and imperial conquest but also to be one of exchange, cross-fertilization and sharing'' (Said 2001). Few scholars have gone so far as Eisenstadt in elucidating the enormous complexity of all civilizations, not least in identifying strains within and between institutional structures and cultural complexes. In consequence of this, each civilization has evolved internally contradictory sub-traditions. Although each embraces a core value that separates some category of worthy humans from one that denigrates others, each alsocontains elements that promote a more inclusive orientation. All civilizations possess customs that promote hospitality toward strangers. They contain elements that can be used to encourage the toleration of diversity. They harbor teachings that cultivate understanding and compassion. They thereby offer seeds that can sprout into resources for inter-human dialogue---a form of open communication that could inspire ways of reducing clashes among contemporary civilizations. In fact, in an interview given shortly before his passing, Eisenstadt emphasized his belief that all civilizations contain universalistic elements (Shalva Weil 2010).

The second critique takes aim at implied assumptions about panhuman belligerence. It questions the notion that combat is the most likely interactional form in which differences come to be resolved. To be sure, much research---by biologists such as Konrad Lorenz, Nikolaas Tinbergen, Richard Wrangham, and Dale Peterson---supports the assumption of an inherent human disposition toward aggression; and some ideologists regard the polemical principle as a defensible human ideal. A growing body of research in neurophysiology, however, supports the idea that humans are essentially motivated by needs for community and social harmony---claims that fit a long tradition of philosophical argument about the value of open communication and consensus. In its pure form this yields to the Habermasian frame that stipulates ideal conditions of conversation under which concerned parties will expectably arrive eventually at similar positions. 

In contrast to a notion of open communication as mutual aggression or harmonious consensus, dialogue signifies a type of discourse in which parties take turns listening respectfully, and responding genuinely to one another's expressions. Empirically, the quest for dialogue draws support from the same human tendencies cited earlier---namely, the need both for attachment \emph{and} for differentiation. It implies, in the words of that prophet of dialogue Martin Buber, ``the acceptance of otherness'' (Buber 1992, 65). The simultaneous wish for attachment and differentiation formed a central theme in the social-psychological analyses of Buber's own teacher in Berlin, Georg Simmel. 

Thanks to the anomalous circumstance that Shmuel Eisenstadt imbibed his sociology from books loaned by Buber, his professor at Hebrew University, he early on became acquainted with this notion of dialogue. Indeed, in later autobiographical reflections he acknowledged the deep impact of Buber's teachings, and went on to edit a volume of Buber's writings for The Heritage of Sociology series. What is more, in the course of writing \emph{Visions of the Sociological Tradition}, I came to realize that Eisenstadt's narrative (in \emph{The Form of Sociology: Paradigms and Crises}) was not, as I previously thought, strictly pluralistic, but rather took the form of a dialogical narrative: it saw diverse approaches to sociology as occasionally offering dialogical openings to one another an interpretation that Eisenstadt himself corroborated in a personal communication (Levine 1995, 96).

\section*{From Clashing to Connecting Civilization: The Greco-Roman Case}

If we were to conjoin Eisenstadt's affinity for the principle of dialogue with his passion for the comparative study of civilizations, we might be led to ask: how was it possible for historic civilizations, rooted as each was on a starkly exclusionary principle, to have evolved to a point where some of their elements could be used to support an ethic of dialogue? How, in other words, could each of the major world civilizations give rise to developments in which authentic traditional symbols were invoked in ways that heighten levels of openness and inclusiveness?

To adumbrate the transformational pattern that I have in mind, let me begin with a prototype of the process in Greco-Roman civilization. The concept of \emph{physis} (nature) formed a central notion in the Greco-Roman worldview. This concept defined nature, not in the post-Newtonian sense of an inherent force which directs the world, but as designating the essential quality of something in a universe of substances. Hellenic philosophers moved from questions about the nature of inorganic and organic bodies to a concept of nature that could be taken as a foundation for ethics. The texts of Plato and Aristotle afforded a basis for superseding conventional notions of morality with a search for what is good by nature as distinguished from what is good merely by tradition or convention (Levine 1995). 

At the same time, however, the notion of nature provided a basis for dividing people into superior and inferior categories on the basis of naturally given characteristics. This distinction was used to reinforce the Greek/barbarian dichotomy, in that all barbarians were held to be slaves \emph{physei} (by nature). Aristotle quotes a line of the poets, ``It its fitting for Greeks to rule barbarians,'' commenting that ``the assumption being that barbarian and slave by nature are the same thing'' (\emph{Politics}, Book 1, ch. 2, 36).

In the minds of other Hellenic thinkers, however, the notion of nature was employed to overcome such political oppositions by envisioning a single polis of the entire world. Diogenes the Cynic thus proclaimed the doctrine of a world state (\emph{cosmopolis}) in which all humans would be citizens. This became a central doctrine of the Stoics, based on the assumption that all humans possess by nature an identical divine spark (\emph{apospasma}). Accordingly, Stoicism undermined distinctions based on race, class, and even gender. These ideas were amplified by Romans like Epictetus and Marcus Aurelius, who expanded the doctrine of humanitarian cosmopolitanism. Their doctrines drew on the core Greco-Roman idealization of nature in ways that articulated the notion of a universal human nature, as a means for transcending the pejorative attitude toward outsiders that proponents of the civilized/barbarian dichotomy had fostered.

\section*{India and Japan}

In the civilization of India, the idea of \emph{purity} (Sanskrit: \emph{sattva}) figured as one central symbolic theme. Connoting freedom from alloy, and so from defilement of the spirit by the impurities of matter, purity was tied to the belief that there is no possibility for humans to see and manifest divinity without being cleansed. In accord with this ontology, Hindus divided people into categories (\emph{varna}) that classified groups with respect to their levels of purity/impurity. Historically, the first group to be so classified was the Brahmans. Although Brahmanic status rested on birth, to become a fully accredited Brahman a man had to study the Vedic texts, learn certain ritual practices, and acquire a holy belt. Brahmans were expected to manifest a number of virtuous qualities, grounded on purity in several dimensions, including purity of body, purity of mind, and purity of heart, and the avoidance of contact with impure substances and persons. They were obliged to provide literary instruction, priestly duties, and certain magical services, and to support themselves from gifts, not by earning a salary. 

Commitment to this ideal of purity had well-known consequences of an exclusionary and destructive character, both internally and externally. Within Indian society, one category designated a set of castes that came to be known as the Untouchables. These were considered irredeemably impure and therefore to be excluded from such goods as rights to own land and opportunities to perform certain rituals. In addition, Hindu doctrine considered those outside their religious traditions to be impure as well. Groups who did not respect the Vedic rituals and the ban on killing certain animals were called \emph{Mleccha} or outsider, a term that generally connoted impure. \emph{Mleccha} and Untouchables were often thought of as being in a similar or identical status category. Hostility toward Muslims thus was grounded to some extent ideologically on their being impure.

On the other hand, the enormous heterogeneity of Indian culture, together with absence of political pressures to impose religion and an egalitarian strain in Hindu culture, accounted for the proverbial syncretistic cast of Indian culture as well as the conspicuous absence of wars of religion (Eisenstadt 1996, 410). Evolving from such background a position of radical egalitarianism and inclusiveness, Mohandas Gandhi devoted himself to overcoming those established polarizing animosities. He strove to secure equal rights for the Untouchables, even renaming them as \emph{harijan}, children of God. He also worked continuously for unity between Hindus and Muslims, aspiring to promote the notion of Indian nationals living together in a civic society. He strove valiantly to prevent the creation of a separate Islamic state following India's Independence, but in vain. Identifying with the traditional Indian notions of \emph{mleccha} and impurity, a Muslim League under Muhammed Ali Jinnah established a ``Nation of the Pure,'' Pakistan.\footnote{They were obliged to provide literary instruction, priestly duties, and certain magical services, and to support themselves from gifts, not by earning a salary. Although Brahmanic status rested on birth, to become a fully accredited Brahman a man had to study the Vedic texts, learn certain ritual practices, and acquire a holy belt.}

Although Gandhi failed to prevent the Islamic split-off and the ensuing massacre of millions, he created a Way for Hindus to transcend tenacious animosities stemming from deeply held cultural convictions by drawing on other aspects of Indian tradition. He did so by turning to classical symbols such as \emph{ahimsa} (nonviolence, drawn from the Jain tradition) and the quest (\emph{graha}) for truth (\emph{satya}). Gandhi found purity above all in what he called the search for truth. He categorically ruled out the use of violence on the ground that it inhibited the search for truth, since no one could know more than a portion of what is true. In Gandhi's teachings, to use \emph{satyagraha} to overcome injustice required considerable training and confidence. Training included understanding and controlling one's impure thoughts through regular meditation. To transform the mind of an opponent, a \emph{satyagrahi} needed this mental purity.

Around the time of Gandhi's transfiguration of Indian notions, a comparable breakthrough was taking place in Japan, with efforts to reorient the heirs of the culture of Japanese warriors. For Japanese civilization, the core symbol to be considered here is \emph{makoto}. Usually mistranslated as `sincerity,' \emph{makoto} signifies a disposition to discharge one's social obligations with utter fidelity, suppressing personal utilitarian goals. Considered the highest virtue of the Japanese hero, \emph{makoto} connotes the value of calm action in whatever circumstances.\footnote{Success is not the criterion here. Ivan Morris (1975) suggests that the value of \emph{makoto} action may be enhanced by failure. Other aspects of \emph{makoto} are described in Gleason 1995.} Although the focus of \emph{makoto} has varied in different periods of history, a constant theme has been the disposition to act in a self-effacing manner on behalf of the well-being of others.

As Eisenstadt (1996) made clear, the ultimate ideal of Japanese civilization lies not in some transcendent value to which worldly actions are held accountable, but to the authority figures of this world, on whose behalf \emph{makoto} actions are dedicated. Since the Middle Ages, the samurai were expected to display this conduct most consistently. The seven pleats of their traditional garb, the skirt-like pants known as \emph{hakama}, allude to what are understood as the components of \emph{makoto}: loyalty, honor, respect, affection, and sincerity (shin). The samurai ethos diffused through Japanese society; economic entrepreneurs recast the notion of samurai \emph{makoto} in ways that favored Japan's economic modernization (Bellah 1957). That ethos was further utilized following the Meiji Reformation by political modernizers, who directed it toward passionate allegiance to the emperor as symbol of the Japanese state.

That symbolism, notoriously, turned Japan in externally destructive directions. It fostered frequent violent combats among trained martial artists. It eventuated in imperialistic ambitions that led Japan to embark on brutal conquests under Emperor Hirohito.

Yet those same samurai ideals served to transform Japan's traditional martial arts in an opposite direction. This began with the work of educator Jigoro Kano, who reconfigured the traditional teaching of lethal unarmed combat, \emph{ju-jitsu}, into a practice of judo utilized only to develop character. It eventuated in the teachings of Morihei Ueshiba, who reoriented martial arts training away from competitive struggle of any sort toward practices designed to produce an attitude of respect for all living beings and to serve as ``a bridge to peace and harmony for all humankind'' (Ueshiba 1984, 120). Ueshiba failed to persuade Japanese militarists to desist from launching war against the United States, just as Gandhi failed to prevent the partition of India. Nevertheless, just as Gandhi's teachings in South Africa and India inspired subsequent political leaders like Martin Luther King, Jr., and Nelson Mandela to relate to their political opponents in a respectful, nonviolent manner, Ueshiba's teachings, through the practice he created, aikido, have inspired millions worldwide to embrace a Way that would enhance inter-civilizational dialogue.

\section*{The Abrahamic Civilizations}

Christianity was founded on an ideal of universal love. Funneled through the Greek word \emph{agape}, the teachings of Jesus propounded the virtue of unselfish and benevolent concern for the welfare of others. The universalistic cast of this teaching received classic formulation in the words of the proselytizing convert Paul, himself influenced by Stoic doctrines, who announced: ``There is neither Jew nor Greek, bond or free, male or female; for ye are all one in Christ Jesus'' (Gal. 3:28). In society after society, these teachings have restrained violence and promoted generosity of spirit.

On the other hand, Christianity holds the record for the number of people from other cultures slain on behalf of a religious emblem, including millions of native Americans, Africans, and aboriginal Australians, not to mention, from among its own members, huge numbers of heretics and ``witches.'' Western Christianity created a tenacious pattern of anti-Semitism that, acknowledged in the recent statements of Pope John Paul II, played a nontrivial role in destroying the civilization of Continental European Jewry. Although Christian figures from time to time espoused a turn to the ethos of Jesus and early Christianity, almost none of them grappled conspicuously with the challenge of using the foundational statements of Christianity to oppose the waves of persecution launched against the Jewish people in their midst (Carroll 2001).

None of them, that is, until Pastor Dietrich Bonhoeffer. Inspired by the social activism of the Abyssinian Baptist church in Harlem, which he assisted during a postdoctoral year at the Union Theological Seminary in the early 1930s, Bonhoeffer returned to Nazi Germany to join Martin Niemoeller in his work with the Confessing Church (\emph{Bekennende Kirche}), the center of Protestant resistance to the Nazis. He directed one of the underground seminaries of the Confessing Church in 1935. After the Nazis closed down the seminaries, he went on to engage in underground activity to help Jews escape and was associated with the conspiracy to assassinate Hitler. The theological and ethical statements that he worked out in the course of this resistance became a benchmark for a new brand of Christians. In justifying courageous pastoral intervention against Nazi oppression, he worked out a justification of political activism in an immoral world, based on a notion of ``venture of responsibility'': ``It is better to do evil than to be evil,'' he decided. His theological creativity has been described as forging a kind of ``religionless interpretation of biblical concepts in a world come of age'' (Bonhoeffer 1963, 5). Bonhoeffer thereby paved the way for the more inclusive kind of rapprochement that many German Christians have displayed since the War, and has been described as a key theologian for leading future generations of Christians.

For Islam, the core symbolic notion is, evidently, \emph{islam}, i.e., submission. This signifies a posture of humble acceptance of and outward conformity with the law of God. The term is derived from Arabic \emph{`aslama}, to surrender or resign oneself, in turn derived from Syriac \emph{`aslem}, to make peace. Islamic tradition focuses on a complex of laws found in the Koran and promulgated by Muslim clergy, laws which cover everything from family relations and civil accords to criminal codes.

Among the notions to which Muslims owe submission, nothing is more motivating than the injunction to pursue \emph{jihad}. And nothing illustrates the capacity of civilization to promote different directions better than the different meanings this term has acquired in Islamic civilization. On the one hand, \emph{jihad} refers to aggression against Unbelievers through the legal, compulsory, collective effort to expand territories ruled by Muslims. Most scholars argue that despite ambiguities about the term in the Koran, this has been the principal line of interpretation of the doctrine in Islamic tradition. Thus, \emph{jihad} was invoked to instigate the conquest, beyond the Arabian Peninsula, of the region from Afghanistan to Spain within a century of Mohammed's death, and later to spur Muslim invasions of such territories as India, Anatolia, Balkans, Ethiopia, Sudan, and West Africa. More recently, it has been dramatically revived in modern Islamic fundamentalism by influential figures such as Sayyid Outb, who argues that the only way for Muslims to achieve religious purity is to establish an Islamic state through \emph{jihad}.

On the other hand, \emph{jihad} has been interpreted as a struggle for personal moral improvement, in the sense of living more closely in accord with Islamic Law. Thus, in language that parallels Ueshiba's formulation that in his form of martial art, there are no enemies and that the greatest victory is the victory over oneself, the 11th-century theologian Abu Hamid al-Ghazali maintained that the soul is an enemy which struggles with one and which must be fought, and that this \emph{jihad} against the soul constitutes the ``greater \emph{jihad}'' (al-Ghazali 1995, 56). In this sense of the term, it extends beyond overcoming baser instincts to a struggle for social justice. So understood, it could be viewed as an injunction to live peaceably with everyone, and to cooperate with people of all faiths in a quest for social reform. This position has been embraced by virtually all Sufi theologians. This accords with the absence in Islam of any particularistic ethnic emphasis, apart from the status of Arabic as a sacred language (Eisenstadt 1992, 41). In fact, in many contemporary societies until recently, including Ethiopia and India, the norm was for public displays of solidarity between Muslims and other religious groups.

Although some progressive Muslims wish seriously to promote and extend the latter definition of \emph{jihad}, no charismatic figure, such as a Gandhi or a Bonhoeffer, has arisen to challenge authoritatively the contemporary drift toward an escalation of the other view.\footnote{This view was propounded with particular virulence by heirs to the 13C \emph{jihad} revivalist Ibn Taymiyya and his 18C disciple, Mohammed Ibn Abdul WahhabNajdi, from whom the fundamentalist Wahabi sect derives.} In the past dozen years, Muslims appealing to the symbol of \emph{jihad} have launched a worldwide campaign involving assassinations, vandalism, and terrorist acts---against Christians in Indonesia and Yemen, Jews in Israel, Hindus in Kashmir, and traditional religionists in Sudan; and against Buddhists through demolition of their world-prized mountain sculptures in Afghanistan. This trend has been exacerbated by another tenet of Islamic faith, the notion that the requirement to act in accordance with God's decrees as a condition of salvation---possible but difficult to fulfill---may be short-circuited when fulfilling the religious obligation of \emph{jihad}, thereby enhancing one's chances of being sent to heaven at the Last Judgment or, if one dies a martyr, going directly to heaven. 

For Jewish civilization, a core symbolic notion is \emph{berith}, or covenant. This refers to biblical accounts of the covenants made between God and the Jewish people, whereby God would provide certain benefits for the people of Israel in exchange for their loyalty to Him and obedience to his moral directives. Accordingly, a central distinguishing feature of Jewish civilization, in Eisenstadt's insightful account, consists of the semicontractual relationship with the Higher Power, in contrast to the absolute status of the transcendental symbols in the other Axial Age civilizations. 

Over time, as related in the Bible, the content of God's promissory note changed. With Abraham, it had to do with the Eretz, the Land, of Israel. With David, it had to do with legitimizing the political authority of a lineage. But the heart of the divine covenant for Jewish civilization lies in the central chapters of the Book of Exodus, where God's promises to consider the Jews a Chosen People, in exchange for their adherence to the numerous commandments enumerated therein. 

The quality of being Chosen set up a constant invidious comparison with other peoples, referred to in what later became a pejorative Yiddish term, the \emph{goyyim}. This dichotomy never led to conquest or aggression, although when a 6C South Arabian king DhuNuwaas converted to Judaism, he began to persecute Christians (thereby provoking the Ethiopian Christian emperor at Aksum to send troops across the Red Sea to overthrow him). However, the conceit of chosenness produced at times an arrogant attitude toward outsiders that belittled their worth. (One account relates that Mohammed's turn against Jews was based on their rejection of his appeal for support at the beginning of his mission.)

On the other hand, the evident meaning of chosenness, as the covenant is spelled out in Exodus 19-24, signifies the adherence of Jews to a system of maxims that enjoin ethical behavior toward a wide range of people. Prominent among those maxims is the commandment to take care of strangers. Whatever narrow, cultic or particularistic grounds for the Covenant are entailed in the covenant with Abraham, or later with King David, are far overshadowed in the history of Judaism by moral imperatives. And this history of Judaism is itself an essential part of the core symbolism. The central text of Jewish Civilization takes the form of a historical narrative, not a straight listing of absolute commands or mythic portrayals. The course of its history moves steadily away from the primordial cultic observance and toward a universalistic ethical dimension. This shift is itself a subject of attention in the sacred text itself, as when God rebukes those who simply following old ritual prescriptions for fasting, just bowing their heads, and spreading sackcloth and ashes under them: ``Is not this the fast that I have chosen? To loose the bands of wickedness, to undo the heavy burdens, and to let the oppressed go free?'' (Isaiah 58:6).

Even so, the particularistic aspects were never completely transcended; People and Land were perpetually celebrated. And when the time of the great return arrived, there were those who sacralized it in the terms of the earliest covenant. For them, the reappropriationof ancient soil amounted to a return of the earliest covenant. For some, that motivated a commitment to reclaim territory by building settlements on a vulnerable, contested area that became a constant provocation to the people with whom they were sharing this piece of the earth's surface. This appeal to the earliest covenant has been defended in some fundamentalist Christian groups more avidly than by most Jews.

\begin{figure}
\caption{Exclusionary and Inclusionary Concepts of Selected Civilizations}
\scriptsize
\begin{adjustwidth}{-4em}{-4em}
\begin{center}
\begin{tabular}{ | C{.16\textwidth} | C{.16\textwidth} | C{.16\textwidth} | C{.16\textwidth} | C{.16\textwidth} | C{.16\textwidth} | }
\hline
\textbf{Civilization} & \textbf{Core Idea} & \textbf{Benign \mbox{Consequences}} & \textbf{Exclusionary Framework} & \textbf{Expanded Inclusionary Concept} & \textbf{Creative Agent} \\
\hline
Greco-Roman & nature & rational ethics & civilized/ barbarian & cosmo\-politanism & Stoics \\
\hline
Indian & purity & Brahmanic moral leadership & pure/impure & satyagraha & Gandhi \\
\hline
Japanese & \emph{makoto} & social order, rapid \mbox{modernization} & \emph{nihon}/\emph{gaijin} & aikido & Ueshiba \\
\hline
Western Christian & \emph{agape} & domestic pacification & believer/pagan & Confessing Church & Niemoeller \& Bonhoeffer \\
\hline
Islamic & submission & domestic pacification & \emph{umma}/infidel & peaceful jihad? & Badshah Khan \\
\hline
Jewish & covenant & \mbox{Promulgation} of moral law & chosen/gentile & cohabitants on sacred land? & Buber \\
\hline
\end{tabular}
\end{center}
\end{adjustwidth}
\end{figure}

\section*{A Challenge for the Future}

The major source of civilizational clashes in the coming generation lies in the actions of the minority of Abrahamic religionists who are extreme fundamentalists. Most visible, of course, are those Muslims who insist on the aggressive side of \emph{jihad}. There could be a kind civilizational clash in the coming generation if those Muslims who insist on the aggressive side of \emph{jihad} continue to grow in strength---if the politicized elements of Islamism continue to make headway in their recurrent assaults on the other world religious groups including Hindus and Buddhists as well as Christians as well as Jews. 

Jews also play a part in perpetuating the clash of civilizational exclusivists. Those who do so include those settlers who occupy the West Bank, not as a tactical move, but out of deepest conviction. Just as militant jihadists draw on early Islamic beliefs and practices to inspire their terrorist attacks, so ardent Jewish West Bank settlers draw on archaic biblical symbols to justify this occupation.

One way these symbols can be recast is through the emergence of a charismatic leader or group who, steeped in traditional symbolism, will connect Islam with its deepest roots in ways that point to inclusionary imperatives. Within the Islamic tradition, the potential for turning \emph{jihad} in a nonviolent, inclusionary direction was demonstrated by Khan Abdal Ghaffar Khan (1890-1988)---known as Badshah Khan---a Pathan (Pushtun) Muslim from Afghanistan. Khan defined Islam as a faith in the ability of every human being to respond to spiritual laws and the power of \emph{muhabat} (love) to transform human affairs. So oriented, Khan raised a `nonviolent' army of some 100,000 Pathan warriors and worked closely with Gandhi to use nonviolent techniques to promote social justice and independence (Easwaran 1999). In this vein strong statements against Islamic terrorism have been issued by contemporary Islamic spokesmen such as Abdal-Hakim Murad, who finds the taking of innocent civilian lives unimaginable in Sunni Islam, and Hamza Yusuf, a popular American Muslim speaker, who has declared that the ``real \emph{jihad}'' for Muslims is to rid Islam of the terrorist element.

And as in Islam, potential for overriding such exclusionary claims lies near to hand in Judaism. The Talmudic tradition has recently been drawn on by Aaron Lichtenstein, in \emph{The Seven Laws of Noah} (1981), to argue that observance of the Noahide laws sufficed to include non-Jews in the divinely approved community. Figures such as Joseph Abilea have eloquently endorsed a nonviolent, universalist position, as have participants in such groups as Oz ve-Shalom, the Jewish Peace movement. A substantial portion of the world Jewish community considers the moral covenant of Exodus to supersede the territorial part of the covenant with Abraham.

To make these new openings does not require a purist \emph{ex nihilo}. The charismatic innovators needed could come from perfectly conventional backgrounds, as did the exemplars whom I described above. Gandhi began as an elitist who shared the white South Africans' disdain for blacks. Ueshiba served proudly in the Japanese army in 1904 and trained officers of the Japanese military academy until 1941. Niemoeller, a submarine commander in World War I, supported the National Socialists until they came to power in 1933. Bonhoeffer began as a conventional German who refused to perform the marriage ceremony of his brother to a Jewish woman in 1930. What all of them shared was a deep grounding in their respective traditions, which earned them credibility, and then a powerful impulse to break out of their elitist/ethnocentric molds in response to the ethical demands of the current world situation. 

In a brief essay composed just after World War I, ``What Is To Be Done?'' Eisenstadt's mentor Martin Buber confronted the dilemma of our time in the voice of unknown comrades:

\begin{quote}
Some say civilization must be preserved through ``subduing.'' There is no civilization to preserve. And there is no longer a subduing! But what may ascend out of the flood will be decided by whether you throw yourselves into it as seeds of true community. No longer through exclusion but only inclusion can the kingdom be established. \ldots{} Silently the world waits for the spirit. (1957, 111)
\end{quote}

\section*{References}

\begin{list}{}{\refstyle}
\item Al-Ghazali. 1995. Al-Ghazali, On Disciplining the Soul [KitabRiyadat al-nafs] and on Breaking of the Two Desires [KitabKasr al-Shahwatayn]: Books XXII and XXIII of the Revival of the Religious Sciences [IhyaUlum al-Din]. Trans. T.J. Winter. Cambridge: Islamic Texts Society.
\item Aristotle. 1984. The Politics. Trans. Carnes Lord. Chicago: University of Chicago Press.
\item Bellah, Robert N. 1957. Tokugawa Religion. Glencoe, IL: Free Press.
\item Bonhoeffer, Dietrich. 1963. The Communion of Saints: A Dogmatic Inquiry into the Sociology of the Church. Trans. R. Gregor Smith. New York: Harper and Row.
\item Buber, Martin. 1957. ``What Is To Be Done?'' In Maurice Friedman, ed. and trans., Pointing the Way: Collected Essays . Atlantic Highlands, NJ: Humanities Press International.
\item \_\_\_\_\_\_. 1992. On Intersubjectivity and Cultural Creativity. Ed. S.N. Eisenstadt. Chicago: University of Chicago Press.
\item Carroll, James. 2001. Constantine's Sword: The Church and the Jews - A History. New York: Houghton Mifflin. 
\item Easwaran, Eknath. 1999. Nonviolent Soldier of Islam: Badshah Khan, a Man to Match His Mountains. Tomales, CA: Nilgiri Press.
\item Eisenstadt, S.N. 1992. Jewish Civilization: The Jewish Historical Experience in a Comparative Perspective. Albany, NY: State University of New York Press.
\item \_\_\_\_\_\_. 1996. Japanese Civilization: A Comparative View. Chicago: University of Chicago Press.
\item \_\_\_\_\_\_. 2003. Comparative Civilizations and Multiple Modernities. Leiden: Brill.
\item Gleason, William. 1995. The Spiritual Foundations of Aikido. Rochester, VT: Destiny Books. 
\item Huntington, Samuel P. 1993. ``The Clash of Civilizations.'' Foreign Affairs, Summer. 
\item Levine, Donald. 1995. Visions of the Sociological Tradition. Chicago: University of Chicago Press. 
\item LeVine, Robert A., and Donald T. Campbell. 1972. Ethnocentrism: Theories of Conflict, Ethnic Attitudes, and Group Behavior. New York: John Wiley and Sons.
\item Lichtenstein, Aaron. 1981. The Seven Laws of Noah. New York: The Rabbi Jacob Joseph School Press.
\item Marriott, McKim. 2003. ``Varna and Jati.'' In Gene R. Thursby and Sushil Mittal, eds.,The Hindu World. London: Routledge (forthcoming).
\item John J. Mearsheimer, ``Imperial by Design,'' The National Interest, No. 111 (January/February 2011)
\item Morris, Ivan. 1975. The Nobility of Failure: Tragic Heroes in the History of Japan. New York: Holt, Rinehart, and Winston.
\item Said, Edward W. 2001. ``The Clash of Ignorance.'' The Nation, October 22.
\item Outb, Sayyid. 1980. Milestones. Beirut: The Holy Koran Publishing House.
\item Smith, Thomas, and G. Stevens. 2002. ``Hyperstructures and the Biology of Interpersonal Dependence.'' Sociological Theory 20:1, 106-30.
\item Sumner, William Graham. 1906. Folkways. New York: Ginn.
\item Ueshiba, Kisshomahu. 1984. The Spirit of Aikido. Trans. TaietsuUnno. New York: Kodansha International.
\item Weil, Shalva. 2010. ``On Multiple Modernities, Civilizations, and Ancient Judaism: An Interview with Professor S.N. Eisenstadt.'' European Societies 12 (4): 451-65.
\end{list}