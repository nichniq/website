\chapter[Aikido and the Art of Mediation (2013)]{Aikido and the Art of Mediation}

How can an adversarial relationship be replaced by harmonious transactions that benefit both parties? Independently, portions of the traditions both of Japanese martial arts and of American legal practice have developed ways to accomplish such a change. Both have replaced notions of defeat and victory with the idea of enhancing the wellbeing and autonomy of both parties. 

What follows is a modest effort to open up a conversation about the remarkable confluence of those two developments. The paper sketches the historical \emph{evolution} of their key ideas---for the martial arts, in the development of \emph{aikido}; for legal practice, through the development of \emph{mediation}. It proceeds to outline some \emph{key features} of the two practices. A concluding section offers suggestions regarding ways the two practices stand to reinforce and \emph{learn from each other}.

\section*{I. The Martial Arts in Japanese Culture}

The practice of aikido emerged in 20th-century Japan following an evolution of martial arts there over two millennia. Those arts stem from customs of the \emph{samurai}, a stratum of military specialists that came to the fore in the late Heian Period (10-12C CE). The \emph{samurai} came to replace the stratum of professional warriors of preceding centuries---men from a different ethnic group it seems, who originally were hunters and manifested an extreme sort of raw violence; other Japanese often viewed them as barbarians or wild beasts. However, seeds of the tutored \emph{samurai} culture can be found in the 8C Japanese classic, the Kojiki. Before that, esoteric lore regarding sword work was cultivated at the imperial court.

Initially, the \emph{samurai} (``retainers'') were positioned to serve the court nobility. In time, they acquired power in their own right, establishing domination over agricultural land, and building their own hierarchical political organizations. This culminated in a semi-centralized military regime, the shogunate, in the late 12C. The \emph{samurai} political organization rested on the formation of strong emotional bonds between military masters and vassals upheld by a strict code of honor (Ikegami 1995). By the 16C the \emph{samurai} code was elaborated into a code known as \emph{bushido} (the Way of the Warrior), consisting of seven bushi virtues: integrity, rectitude, courage, benevolence, honor, loyalty, and respect.\footnote{The seven \emph{bushi} virtues came to be symbolized by the seven pleats of the \emph{hakama}, a skirt worn by \emph{samurai} during the Tokugawa period ((1603-1868).}

Beyond qualities of comportment, \emph{samurai} were expected to show proficiency in a number of non-martial spheres that linked with the neo-Confucian notion of personal culture (\emph{bun}). This linkage was represented by an ideal that conjoined them by means of a compound phrase, \emph{bu-bun}. One such art was the composition of highly stylized verse, most notably haiku. Another was calligraphy: the embodiment of \emph{bu-bun} involved practice with pen and brush in a manner that evinced unself-conscious, fearless directness. Shogun Tokugawa Ieyasu proclaimed that the brush and the sword are one.

Nevertheless, the core \emph{bushido} virtue consisted of fearless combativeness in battle and readiness to kill or be killed by a perceived enemy. In the words of samurai Kato Kiyomasa (1562-1611), ``[By] reading Chinese poetry \ldots{} one will surely become womanized if he gives his heart knowledge of such elegant and delicate refinements. Having been born into the house of a warrior, one's intentions should be to grasp the long and the short swords and to die'' (Wilson 1982, 131).\footnote{Kato sama further prescribes: ``One should rise at four in the morning, practice sword technique, eat one's meal, and train with the bow, the gun, and the horse. \ldots{} When one unsheathes his sword, he has cutting a person down in mind'' (\emph{Ibid}., 130).} But grasping the swords was far from spontaneous; it required years of training in one of the specialized schools (\emph{ryu}) that flourished toward the end of the medieval period. This involved mastery of one or more of the martial techniques for which complex curricula of instruction had become codified.\footnote{Mastery of the dagger (\emph{tanto}), glaive (\emph{naginata}), bow and arrow (\emph{kyujutsu}), empty hands combat (\emph{jujutsu}) and, above all, the long sword (\emph{katana}) and short sword (\emph{wakizashi}).} During the long period of peace under the Tokugawa Shogunate, the martial skills could rarely be exercised on the battlefield. Even so, their cultivation remained no less sharp. The status of lords often depended on the number and quality of expert martial artists under their authority. The spirit of contests, even for matters of honor, dictated the ambition of seeking victory of an opponent, which often meant his death. Even as the arts of combat became ``domesticated'' during the long Pax Tokugawa, competition among different courts and \emph{ryu} was no less fierce. During the Tokugawa period, it has been said, \emph{samurai} ideals became close to a national ethic, for even the merchant class had become ``\emph{bushido}-ized'' (Bellah 1957, 98). 

With the overthrow of rule by the feudal lords (\emph{shogun}), the system of Japanese martial arts faced major challenges. The advent of Western culture and the spirit of commerce dislodged the hegemony of samurai notions of victory and defeat in combat. Not many years after the Meiji Restoration of 1868, a prominent Japanese educator, Jigoro Kano, began to reconfigure the ethos of martial arts training. Kano Sensei started a \emph{dojo} (training hall) in a Buddhist temple in Tokyo, the \emph{kotokan}, which became the matrix for developing a discipline he called judo. In this effort, he sought to reconfigure the goal of training from defeating enemies into something purely educational: promoting the development of personal character and social engagement. He renamed the educational goal \emph{shushin-ho}, ``the cultivation of wisdom and virtue as well as the study and application of the principles of Judo in our daily lives'' (Kano, in \emph{AikiNews} 1990, 4). As he later came to formulate it, ``the ultimate objective of Judo discipline is to be utilized as a means to self-perfection, and thenceforth to make a positive contribution to society'' (Murata 2005, 147-8).

The view of budo training that Kano articulated became increasingly prominent in Japan in the 20th century. This was especially true following World War II---the most disastrous outcome of the resurgence of the bushidoized nation imaginable, a denouement that Kano opposed. By the 1980s the Japanese Budo Association (Nippon Budokan) took the question of defining their goals so seriously that they spent years deliberating the matter, proclaiming in their 1987 Charter: 

\begin{quote}
\emph{Budo}, the Japanese martial ways have their origins in the age-old martial spirit of Japan. Through centuries of historical and social change, these forms of traditional culture evolved from combat techniques (\emph{jutsu}) into ways of self-development. \ldots{} Practitioners study the skills while striving to unify mind, technique and body; develop [their] character; enhance their sense of morality; and to cultivate a respectful and courteous demeanour. \ldots{} This elevation of the human spirit will contribute to social prosperity and harmony. (Nippon Budokan 1987) 
\end{quote}

Even so, tensions remained between the age-old martial spirit of Japan and the pacific goals of moral development and social harmony. However much Kano Sensei espoused the ideals of ego-transcendence and societal betterment, judo retained something of the traditional martial goals of victory in combat. This spirit was rekindled by the incorporation of judo into Olympic competition. A Budokan was built to house the judo Olympics in 1964, and continues to house national competitions among different martial arts, including karate, kendo, shorinji kempo, kyudo, and naginata as well as judo. In addition to the egoistic competitive spirit promoted by such matches, judo's goal of \emph{victory} enabled practitioners to use such means as ``throwing, choking \ldots{} bending or twisting the opponent's arms or legs. The combatants may use whatever methods they like'' (Kano 1932, 58). Recognizing this tension, the Japan Budo Association saw fit to express concerns over ``a recent trend towards infatuation just with technical ability compounded by an excessive concern with winning'' (Nippon Budokan 1987).

It was given to Morihei Ueshiba to complete the evolution of \emph{budo} and resolve that tension. This involved configuring a curriculum of training that \emph{embodies} in its foundational principles the elimination of competition and movements designed to avoid inflicting pain and promoting peace. Drawing both on superb training in traditional martial ways and on immersion in a universalistic new Japanese religion, Ueshiba's aikido journey began with an epiphanic experience in 1925, through which he says he came to understand that the way of the warrior is to spread divine love. He continued forging new martial techniques throughout the 1930s. In vain he tried to forestall Japan's attacks against the United States. During the war, he withdrew in inner exile to Iwama, where in 1942 he renamed his practice \emph{aikido}. In the postwar years, the catastrophes of Hiroshima and Nagasaki together with revelations from a Japanese soldier present at the liberation of Hitler's concentration camps spurred him into another turn. In 1948 he invited an old disciple, Hikitsutchi Sensei, to join him in promoting a ``new kind of \emph{budo},'' one devoted explicitly to promoting world peace. Ueshiba Sensei continued to refine this practice for the rest of his life, which ended in 1969.

As Ueshiba came to formulate the end of his \emph{budo}, the goal was not victory over the other, but \emph{masagatsu agatsu}: ``the great victory is victory over oneself.'' The practice he created relied not on pain or physical force in any form, but a welcoming of the energy of an attack, neutralizing its aggressive direction, and caring for the attacker. The structure of combat had transformed into a harmonious exchange of gestures. This was an idea whose time had come. In the early 1950s aikido dojos were established first in France and the United States, then in the United Kingdom, Germany, and Australia; at present, more than a million practitioners pursue aikido training in all six continents. 

\section*{II. Litigation in Euro-American Culture}

The transformation from combat to nonviolence in Japanese martial arts appears to have been prompted by educational, civic, and spiritual concerns. In contrast, the move from adversarial legalism to professional mediation in the legal profession was motivated largely by economic and political concerns.

As with the martial arts, arts of litigation evolved over millennia, from resolving disputes through violence, to civil litigation, to socially mediated opposition, to a process of seeking agreements that both parties freely assent to. The initial evolution was from spontaneous fighting between aggrieved parties to formal dueling with rules and witnesses. Among Germanic peoples, trial by combat---sometimes known as judicial dueling---appeared in the early Middle Ages. An 8C document prescribes a trial by combat for two families who dispute the boundary between their lands: the contestants were required to touch a piece of that land with their swords and swear that their claim is lawful; the loser would forfeit claims to the land and pay a fine also. Other issues settled through trial by combat concerned dynastic power. ``Wager of battle'' entered the common law of England following the Norman Conquest. In Renaissance Italy and France codes for formal dueling emerged, conflicts in which honor rather than material interests was at stake. Similar codes emerged elsewhere in Europe, especially in Scandinavia (\emph{Holmgang}) and Ireland (\emph{code duello}). All these were forms in which Might makes Right, under conditions in which social and then judicial norms were in place to regulate the antagonistic encounter. 

In the course of the 16th and 17th centuries, trial by combat began to disappear, initially due to ecclesiastical opposition and then through legislative banning.\footnote{Because Britain did not abolish wager by battle until Parliament's 1819 response to \emph{Ashford v Thornton} (1818), and because no court in post-independence United States has addressed the issue, the question of whether trial by combat remains a valid American alternative to civil action remains open, at least in theory.  \emph{Wikipedia}, ``Trial by Combat.''} Instead, civil disputes came to be settled almost exclusively in courts through the arguments of lawyers and the testimony of witnesses. Modern European civil procedure begins with the Napoleonic Era and the passage of the French Civil Code of 1806. That widely influential code sought to standardize civil procedure. It promoted a court system that featured oral arguments between equal parties that were open to the public. This heightened the dramaturgical presentation of legal conflict in court trials. In the United States, litigiousness grew as the expanding young country's litigation scene evolved alongside new societal and economic conflicts of the Industrial Age and the consequent creation of an ever-denser network of courts. It engendered a system that Robert Kagan aptly describes as ```adversarial legalism'---a method of policymaking and dispute resolution with two salient characteristics: \emph{formal legal contestation} [and] \emph{litigant activism}'' (Kagan 2001, 9). 

Over time, critics began to target the socially dysfunctional aspects of this system. President Lincoln advised Americans to ``discourage litigation'' and instead encouraged them to consider ``how the nominal winner is often the loser in fees, expenses and costs of time'' (Steiner 1995, 2). Edward Bellamy called for the ``abolition of law as a special science,'' seeing ``no use for the hair-splitting experts who presided and argued in [the] courts'' (Hensler 2003, 169). Toward the end of the century, Austrian legalist Franz Klein broached ideas that would gain traction only half a century later, arguing that ``parties to a lawsuit should cooperate in order to facilitate a judgment'' instead of stretching facts and the law in a zero-sum showdown (Rhee, 12). Opposition to litigious practices grew in the 20th century as conflicts between families, contractual parties, and businesses grew more complicated, populations swelled, legal codes thickened, and court costs rose. 

By the middle of the 20th century, litigation had reached a saturation point in American life, as civil case filings reached all-time highs and courts carried overloaded case schedules. One step toward relieving this situation was to give judges assistance from professional court administrators to set their calendars and manage the flow of cases (Hensler 2003, 174). Beyond that, communities and disputants came increasingly to favor alternative forms of dispute resolution. The community justice movement of the late 1960s and early 1970s supported ADR because participants felt that that the litigation system in the United States disproportionately protected elite interests and neglected the need of the socioeconomically disadvantaged. Child custody disputants and divorcees came to see the bloated civil litigation system as too sclerotic and adversarial to produce nuanced outcomes tailored to the specifics of familial and individual disputes. Businesses found that ADR was better equipped to handle industry specific disputes in a manner more in line with the ever-faster world of commerce. This evolved attitude towards ADR is one significant factor in the 84 per cent drop in federal civil cases that went to trial between 1962 and 2002 (Stipanowich 2010, 4). ADR's newfound prominence in American legal life was ratified by the passage of the Alternative Dispute Resolution Act. As a result of the 1998 law, federal courts are required to offer ``some form of ADR'', and many state courts began to standardize such options voluntarily (Hensler 2003, 167). Other countries followed suit. In 2001, for example, the Government of Colombia mandated that all civil and commercial disputes undergo a conciliation process before being filed in court.

The first step away from standard litigation process took the form of arbitration. The process of resolving disputes by submitting them to a third party adjudicator is probably as old as organized human societies. The process became formalized with the expansion of international trade in the 16th century. In France, the 1566 Decree of the Moulins made arbitration the only mean to resolve commercial disputes; in Germany and England, too, arbitration was practiced early and recognized as an effective form of dispute resolution. In the USA, arbitration among merchants was common already in the colonial period, since it proved more efficient than the courts; George Washington himself served as an arbiter prior to the Revolution. Arbitration achieved permanent international status in the wake of the Hague Conference of 1899. In 1923, the League of Nations issued a Protocol on Arbitration Clauses to cover non---domestic arbitration agreements. Two years later, the USA Congress passed a Federal Arbitration Act drafted initially by the American Bar Association.

By the 1960s, massive cultural shifts were starting to provide a type of support for ADR that specifically favored mediation as preferable to arbitration. To the improvements over litigation offered by arbitration---speed and efficiency, reduced cost, and confidentiality---mediation added the benefits of autonomy for the disputants and increased consensuality. The latter values were championed by changes in the social milieu. The growth of family therapies came to provide an alternative to dealing with antagonisms in marriage other than the cold calculations of the divorce lawyer industry. The Civil Rights Movement found in Dr. Martin Luther King Jr. a charismatic proponent of Gandhi's methods of voluntaristic nonviolent political reform. Relatedly, heightened attention to the ideal of universal human rights encouraged tendencies toward non-combative solutions. In this spirit, an industry of Family, Marital, and Business Mediation Services sprang up at national and state levels, as did academies that provided training for professional mediators.\footnote{In the case of at least one prominent professional mediator, the parallels between mediation work and aikido have been explicitly discussed and even diagrammed (Saposnek 1998).}

Indeed, this very cultural jump that produced a market for less adversarial forms of dispute resolution paralleled the shift that created an enthusiastic market for aikido teaching in the martial arts.  Americans and Europeans came to experience a hunger for methods of conflict resolution that favor autonomy and consensus.

\begin{figure}
\caption{Evolution from raw combat to consensual conflict resolution}
\centering
\small
\begin{tabular}{ | C{.3\textwidth} | C{.3\textwidth} | C{.3\textwidth} | }
\hline
Evolutionary stage & \uppercase{Japanese Martial Arts} & \uppercase{Euro-American Litigation} \\
\hline
1. Raw physical combat & Violent struggle & Violent struggle \\
\hline
2. Disciplined physical combat & Samurai martial engagement: \emph{bujutsu} & Trial by combat \\
\hline
3. Regulated verbal combat & --- & Civil litigation \\
\hline
4. Conflict subordinated to societal object & Martial forms subordinated to societal betterment: judo & Arbitration \\
\hline
5. Consensually achieved resolution & Conflict resolution through non-combative interaction: aikido & Mediation \\
\hline
\end{tabular}
\end{figure}

\section*{III. The Methodology of Aikido}

To schematize the methodology of aikido as a resource to manage social conflict, I list below a set of factors known to promote the onset and escalation of conflict, and explore how aikido deals with each of them. This is by no means an exhaustive list; complex tomes and thousands of papers have investigated the universe of internal and systemic variables related to conflict, escalation, and violence.\footnote{I find \emph{Constructive Conflicts: From Escalation to Resolution}, by Louis Kriesberg (2007) a particularly valuable overview the field–not least for its useful distinction between destructive and constructive conflicts.} Those I have selected delineate factors which, in decades of teaching a course on Conflict Theory and Aikido, have seemed particularly plausible to me and relevant to engagement with aikido practice.\footnote{The syllabus of that course has been made public as an Appendix to my \emph{Powers of the Mind: The Reinvention of Liberal Learning in America} (2005).}

Classic theories of conflict identify a number of factors internal to the parties: 1) bio-psycho-sociocultural dispositions toward aggression; 2) emotional reactivity; 3) hostile sentiments of the parties; 4) low self-esteem; and 5) memories of prior conflicts between the parties involved. Social science also has identified kindred factors located in the social and cultural environments, including 6) cultural beliefs about conflict and violence; 7) social controls that dampen conflict; and 8) availability of allies to help protagonists pursue the conflict. 

Dispositions to aggressiveness in human personalities stem from a wide range of biochemical, psychological, social, and cultural factors (Levine 2006a, 2006b). Aikido theory assumes that humans will be subject to aggressive inputs from others as a matter of course. As a practice that seeks to promote harmony in action, accordingly, aikido seeks methods whereby attacks do not elicit counterattacks, but instead teaches ways to neutralize incoming aggression. Indeed, neutralizing aggressive attacks by others forms the core of aikido training. This involves both cognitive and kinesthetic responses. A major cognitive shift involves reframing the attacker as a training partner, not as an enemy; and reframing the attack itself not as a threat but simply as a charge, even a ``gift,'' of energy. In words that noted Sensei Mitsugi Saotome has expressed in seminars, ``when someone grabs your wrist, it does not mean the start of a fight; it is the beginning of a conversation.'' This reconfiguring can be extended to cognitive operations that critique distorted perceptions one has of others (Eidelson and Eidelson 2008). 

Kinesthetically, neutralizing the aggression of an attacker involves a number of moves. It means moving in such a way that the attack is not permitted to impinge on the body or the feelings of the person attacked, which is known as ``getting off the line.'' At the same time, it means allowing the energy of the attack to express itself fully---not ``cutting the ki'' of the attacker. Instead, it means conjoining with the attacker's energy and directing it in such a way that neither party is harmed. It does so, moreover, not in a spirit of directing attackers to change their ways, but by listening to them, concurring with them and, indeed, even caring for them. 

In whatever manner the attacker is defined, there remains the psychological issue of how ready the person attacked is to experience a Fight-Flight response. In his classic work on conflict, economist Kenneth Boulding coined the expression, ``coefficient of reactivity,'' to represent the degree to which parties react to a negative gesture by the other, which Boulding describes as the ``touchiness'' of the parties (1962, 25-7). Aikido puts a premium on learning to ``respond, not react,'' to attacks. Training for this includes learning how to remain calm by continued breathing, relaxed musculature, and staying ``centered''---a state of being in which attention is directed to the lower abdomen. 

Another factor that Boulding identified as inducing the escalation of conflict is the variable of what initial levels of hostility were evinced by one or both parties. Evidently, persons with an initial proneness toward hostile feelings and gestures are likely to instigate attacks and to perform counterattacks. Aikido trains persons to control their hostile impulses in a number of ways. They habituate themselves to express gratitude frequently. They learn to be continuously mindful of their bodily states, and to examine their own motives so as to subdue egoistic strivings that motivate aggressive gestures. 

In a classic paper on community conflict, James S. Coleman (1957) begins his inventory of causes of social conflict by considering whether the parties had a prior history of conflict. Memories of previous conflicts can be recalled quickly and thereby reactivate the neurons that carry traumatic memories. One way in which aikido minimizes this factor is by training people to be present in the moment, to work to avoid carrying the baggage of prior injuries or hurt feelings into current transactions. 

Beyond these factors intrinsic to the parties in interaction, other elements in aikido practice work to substitute harmony for conflicts that are promoted by external conditions. As comparative cultural studies have demonstrated, cultures vary widely with respect to the positive or negative values they place on conflict and violence (Fromm 1973). The ideology of aikido implants strong dispositions to avert or counteract cultural dispositions to aggression. The very word aikido contains elements that signify harmony and love.\footnote{`Aiki' translates as joining of energies, or harmony. `Ai' also has a homonym that signifies love.} In the words of its Founder, ``I'm not teaching you how to move your feet; I'm teaching you how to move your mind toward nonviolence.'' 

The customary ways in which aikido is practiced include elements that theorists have shown to have a dampening effect on conflict. Coleman showed that social conflicts were likely to be contained when the antagonists shared allegiance to some sort of supervening authorities and/or symbolism that enabled them to transcend their local conflict, and to third-party controls over their interaction (Coleman 1957). Aikido practice always begins and concludes with a ritual bow to the Founder of the practice and to the Japanese kanjis that signify harmonious interaction. On the mat, instructors intervene tirelessly to check students when their movements become the least bit aggressive. Other theorists point to the tendency of combatants to escalate conflict through the recruitment of allies among others in the system (Kerr 1988). Again, dojo etiquette requires partners to solve their own problems, and to seek assistance only when they cannot reach a solution in any other than a combative manner.

On all counts, then, aikido works to reduce if not eliminate factors understood to produce conflictual interactions, such that its practitioners do successfully replace notions of defeat and victory with the idea of enhancing the wellbeing and autonomy of both parties.

\begin{figure}
\caption{Elements of aikido that reduce conflict and promote mutual respect}
\centering
\begin{tabular}{ | C{.5\textwidth} | C{.5\textwidth} | }
\hline
\bfseries\uppercase{Factors That Promote Conflict and Escalation} & \bfseries\uppercase{Aikido Responses That Counteract Those Factors} \\
\hline
Aggression invites counterattack  & Neutralizing the aggression: get off the line of attack; reframe the attack; permit energy of attacker to spend itself \\
\hline
Reactivity heightens Fight-Flight response & Relaxation and centering \\
\hline
Hostile sentiments feed the fight & Generalized gratitude; understanding and connecting with the other \\
\hline
Insecure egos cannot stand humiliation of defeat & Using setbacks or ``failures'' as occasions for growth \\
\hline
Memories of prior conflicts feed reactions & Focusing awareness on the present \\
\hline
Symbols glorifying war, macho aggressiveness & Symbols of peace and humanity \\
\hline
Ineffectual moral authorities & Instituting respected authorities \\
\hline
Recruiting allies heightens escalation & Search within to eliminate discord \\
\hline
\end{tabular}
\end{figure}

\section*{IV. The methodology of mediation}

While the aiki approach to managing conflict emerged from a continuous historic process of domesticating martial ways, from the most brutish combat to cultivated weaponry to a benign exchange of non-injurious gestures, the history of judicial litigation shows a substantial upturn before economic and political crises forced the turn to alternative forms of dispute resolution. Sociologist Georg Simmel was among the first to note that when interpersonal disputes get transferred to the jurisdiction of courts, they become uncompromising in content and vicious in execution. In a passage worth citing at length, he writes:

\begin{quote}
In judicial conflict \ldots{} claims on both sides are pursued with pure objectivity and by employing all permissible means, without being deflected or in any way attenuated by personal or any other extraneous considerations. \ldots{} Elsewhere, even in the fiercest battles, something subjective, some mere turn of fortune, or some interference from a third party is at least possible. In legal conflict, however, everything of that sort is excluded by the matter-of-factness with which the just fight and absolutely nothing else proceeds. \ldots{} The prosecution of legal battles in more evolved societies serves the pure disentanglement of the controversy from all extraneous personal associations. When Otto the Great orders that a particular legal controversy be settled through trial by combat (\emph{gottesgerichtlichen Zweikampf}) to be decided through professional swordsmen, only the bare form---the process of fighting and winning---is what remains out of the whole conflict of interests. (Simmel [1908] 1992, 305-6; translation mine)
\end{quote}

In this spirit, from Law School on the contemporary legal system trains lawyers to deal with conflict by out-strategizing and out-maneuvering their opponents through an arsenal of techniques that aim at convincing a jury or a judge to produce a decision favorable to their interests---without regard to the best interest of both parties, and surely without regard to the best interests of third parties and society more generally. In the words of Daniel Weinstein, a former litigator and judge who became a professional mediator:

\begin{quote}
The goal of convincing juridical authorities is achieved through employing a blitzkrieg of maneuvers that includes interrogatories, depositions, and advocacy aimed at influencing decision makers rather than the ``opponent.'' The results are measured by how much you ``win'' \ldots{} like Rocky standing on the steps with his arms raised in victory. Unlearning this warrior-like behavior for any litigator who enters the world of mediation advocate is difficult and not at all natural. Winning by a verdict imposed on the other side is so much a part of our system that in order to inveigle lawyers to take mediation training, I once had to rename a course I taught on the subject from ``Effective Mediation Advocacy'' to ``How to Win at Mediation,'' an oxymoron of sorts. (Weinstein 2004).
\end{quote}

Accordingly, just as aikido practitioners have to unlearn so much that is associated with the samurai ambition to defeat an enemy, so do lawyers who wish seriously to pursue a career in mediation have to learn a whole new set of techniques, techniques which are rarely available in the curricula of law schools. As Weinstein phrases it:

\begin{quote}
Effective mediation skills for the lawyer representing a client are very different from those of the litigators, whose skills do not translate from the courtroom to the mediation table. Stating your claims in terms that do not inflame the other side, and yet still integrate your clients' important interests, is a learned rather than a spontaneously manifested skill. Turning your opponents' fears, weaknesses, and anxieties into advantages, giving them a share of the outcome, and creating win/win solutions are new territory for the warrior litigator. (Ibid.)
\end{quote}

The skills and norms of mediation were codified initially by practitioners in the areas of family counseling and conflict resolution education. The mediation movement was boosted substantially by the publication \emph{Getting To Yes} (1981), the outcome of a Negotiation Project at Harvard University (2nd ed., 1991). The authors offer prescriptions for conduct that run precisely opposite the paradigm of lawyerly practice that Simmel had articulated when writing about legal conflict. They advocate moving from a win-lose mentality in which personal feelings and biases are rigorously excluded to a process in which perceptions are clarified; emotions are recognized and legitimated; listening to one another is prioritized; what the participants really need and want is assessed honestly; finding solutions in which both parties gain is encouraged; and fair standards and fair procedures are agreed to. 

During the 1980s, a growing number of lawyers and judges developed an increasingly sophisticated repertoire of ideas and techniques for resolving disputes through mediation. In \emph{Mediation: A Comprehensive Guide to Resolving Conflicts Without Litigation}, Folberg and Taylor provided a useful overview of the field. They provided a useful, succinct definition of the process of mediation: ``[A]n alternative to violence, self-help, or litigation that differs from the processes of counseling, negotiation, and arbitration. I can be defined as the process by which the participants, together with the assistance of a neutral person or persons, systematically isolate disputed issues in order to develop options, consider alternatives and reach a consensual settlement that will accommodate their needs. Mediation is a process that emphasizes the participants' own responsibility for making decisions that affect their lives. It is therefore a self-empowering process'' (7-8). The volume offered materials on stages of the mediation process; relevant skills; diverse styles of mediating conflict; the educational, ethical and practical dimensions of mediation as a profession and an extensive bibliography.\footnote{The literature on mediation techniques has grown enormously in recent decades.  Prominent treatments include such titles as \emph{Mediation: The Roles of Advocate and Neutral} (Folberg and Golann 2011) and ``The Secrets of Successful Mediators'' (Goldberg 2006).}

Although law schools were relatively slow to embrace this approach, since 2000 they have hastened to catch up. At present, many introduced courses and even programs about mediation. Now almost every American law school offers a course in mediation; many in fact offer programs with a constellation of mediation courses, clinics, and certificates. In the process, numerous traditional law course texts have come to include some material on mediation in the domains of contracts, torts, and trial practice.

If one were to draw up a set of training points for mediators that bears some resemblance to the list presented for aikidoka, it might look something like the following.

\begin{figure}
\caption{Elements of mediation promoting agreement based on mutual respect}
\centering
\begin{tabular}{ | C{.5\textwidth} | C{.5\textwidth} | }
\hline
\bfseries\uppercase{Factors That Maintain a Litigious Ethos} & \bfseries\uppercase{Mediator Responses That Counteract Those Factors} \\
\hline
Aggression as a stimulus & Lawyers and clients must not attack one another \\
\hline
Reactivity & Maintain a calm and friendly atmosphere \\
\hline
Hostile sentiments & Spot and build on points of agreement \\
\hline
Insecure egos & Praise willingness to be open and creative \\
\hline
Prior history of conflicts & Focus on present aspirations and future goals \\
\hline
Ideology favoring conflict & Appeal to general values of harmony \\
\hline
Nonexistent supra-local controls & Mediator stands to control escalation, and to adduce authority of shared values \\
\hline
Available allies & Identify allies as others who have successfully completed a mediation process \\
\hline
\end{tabular}
\end{figure}

\section*{V. Mutual relevance}

For a society and a time dominated by an ethos of competitive individualism---where the business world dominates public imagination and feeds upon the imagery and motivations of competitive sports---where the American Dream is configured in terms of individuals' ``getting ahead'' and where heroes are celebrated by how they achieve Victory and handle Defeat---aikido and mediation represent cutting edge, counter-cultural engagements in which the dominant motifs include Win-Win, subdue the ego, communicate openly, learn to trust, and build consensus. This is so, we have seen, even though both of them derive from traditions informed by centuries of mortal combat but which have been transformed at their core. 

Insofar as these practices have contemporary value, it may be useful to see in what ways they can be seen to reinforce one another and, even more, how each can enrich and contribute to the other. Both join a number of other contemporary modalities in which combative procedures are explicitly replaced by practices that eschew adversarial postures. These include Couples Therapy, Nonviolent Communication (Rosenberg 2005), a wide range of Alternative Dispute Resolution strategies, as well as Principled Negotiation (Fisher, Ury, and Paton ([1981] 1991).

\section*{Aikido's Gifts to Mediation}

Aikido practice seems pertinent to all three of the domains in which mediators act: 

\begin{enumerate}
\item the mediator's effect on the conduct of the disputing parties and their lawyers;
\item the mediator's effect on the interactional context of the mediation efforts; and
\item the personalities of the mediators themselves.
\end{enumerate}

\subsection*{Affecting the litigators}

The mediation process requires exactly the opposite of what conventionally trained lawyers and their clients are disposed to do. In the words of experienced mediator Antonio Piazza, ``Litigators tend to think of themselves as warriors. Frequently they come into mediation and forcefully communicate to the other party that the other party is: (a) simply wrong, and (b) perhaps too stupid to know it, and (c) quite probably too venal to care, and (d) if they don't settle they will be beaten to a pulp in court'' (2004). Even though the actors in question understand that the goal of the process is a settlement agreement signed voluntarily by both sides, such counterproductive dispositions are a ``natural'' response based on aggressive instincts and a culture that values aggressive macho attitudes. 

An aikido approach here would be not to change the behavior of others, but to change oneself. This begins with the self of the mediator. That is, discarding the usual method of importing techniques into a situation designed to instruct or coach someone how to communicate less aggressively or less defensively, the mediator opens up him-/herself to non-directive, non-manipulative communication. So Piazza:

\begin{quote}
For the mediator, the process is not one of standing outside a dispute and applying skillful techniques to it, but entering fully and wholeheartedly, and without importing yet another agenda (and its concomitant fears and desires) into an already changed situation. By way of example: Mediation theory may tell you that it is critical to allow a disputant with an emotional charge to vent their feelings, and experience being heard. But if ``active listening'' is practiced as a technique to remove an obstacle, the felt experience of the disputant is as likely to be ``I am being manipulated'' as ``I am being heard.'' Paradoxically, aikido might well move you to fill the space between you and the disputant who is winding up for a tirade instantly, and so completely, that he never gets going at all. While that may sound brutal, the felt experience can be one of compassion. The difference is whether you are ``doing to'' or opening up to the person with whom you are interacting. (Ibid.)
\end{quote}

\subsection*{Affecting the interactional context}

People who train aikido walk into a dojo carrying whatever stresses, frustrations, peeves, and gripes the day has brought them. They are expected to leave these at the door, much as Ethiopians traditionally left their weapons at the door of the church or mosque before they entered. They bow into the dojo, begin and close their training with a communal ritual. Expectations for deportment while practicing in then dojo are made clear.

It might be of value for mediators to direct some attention to the ritual setting of their deliberations. Another idea would be to distribute beforehand a list of point about etiquette in the mediation setting, much as many aikido organizations distribute to newcomers information about dojo etiquette. The psychosomatic power of gratitude can be rehearsed at unusual times. No less important would be words that remind the participants to reframe continuously the setting of their work: from a situation of combat to an opportunity to become more free and creative partners in a problem-solving conversation. One experienced meditator has suggested recently that the mediation process would be enhanced by attending more consciously to preliminary groundwork for mediation and concluding mediation with words of grace that acknowledge the work that has been accomplished consensually. Aikido promotes such moments somatically both through a moment of getting centered before each exercise and by bowing appreciatively to one another at the beginning and conclusion of every practice. 

\subsection*{Personalities of the mediators}

Practiced aikidoka may understand the situation of mediators better than they do themselves, in the sense of being trained in mindfulness about inner somatic and emotional responses to a complex of aggressive actors swirling about them. On this point, experienced aikidoka-mediator Stephen Kotev maintains that there is a serious gap in the training of ADR practitioners:

\begin{quote}
As mediators and conflict resolvers our somatic education has been neglected. Mediators are starting to realize their body language is often communicating more than they know. A clenched jaw, an exasperated look can say more than you ever intended. Your stress may cause you to say or do something that you later will regret. Wouldn't it be nice to be able to notice where in your body you were feeling stressed and be able to release it? Wouldn't it be nice to be able to show our neutrality in our posture as well as in our words? Knowledge of your physical process will help you be a more effective conflict resolver. (2007)
\end{quote}

The mediator's need to be neutral requires a level of emotional development that is not easily come by. Aikido offers a variety of techniques and exercises that promote the state of being ``centered,'' a state wherein the charged pushes and pulls of a subliminally litigious context can be finessed. Indeed, learning to be \emph{centered under stress} forms a central part of aikido training. The state of being centered enhances abilities to perceive tense situations with more clarity and understanding, and to become aware of openings and options in stuck situations. Beyond that, the mediator works best when manifesting a positive state of openness and love that litigants can be exposed to and mirror. One particular relevant training is that of \emph{randori} practice, where one is being attacked simultaneously by a surround of aggressive bodies and moving in an aware and flowing manner to manage them effectively. 

\section*{How Mediation Might Enrich Aikido Practice}

This gets us into truly uncharted territory. The most I can do here is throw out a few suggestions. One is that the work of mediators provides greater awareness of the interpersonal dynamics involved in neutralizing aggression and harmonizing energies. This would evidently be particularly true of those, like family or couples therapists, whose primary focus is on the emotional landscape of the parties they work with.

Another contribution could be to turn the attention of aikidoka to the whole area of three-party interactions. Virtually all of aikido training concerns what to do when one party is being attacked by another. Aikido as hitherto practiced has little to show about how to stop fights, how to turn combat among others into conversation, and how to attain peace other than working one each individual's potential response to negativity. In today's world, that cannot be sufficient.

We remain beginners in these new modes of communication. It remains to be seen---most certainly, a worthy initiative to consider---what insights and fresh understandings of their own practices might emerge from occasions in which small numbers of mediators and aikidoka were brought together to share with one another reports of what they already do. I hope that these thoughts might stimulate others to carry the conversation forward.

\section*{References}

\begin{list}{}{\refstyle}
\item Bellah, Robert. 1957. Tokugawa Religion. Boston: Beacon Press.
\item Boulding, Kenneth. 1962. Conflict and Defense: A General Theory. Lanham, MD: University Press of America. 
\item Coleman, James S. 1957. Community Conflict. Glencoe, IL: The Free Press.
\item Eidelson, Roy J. and Judy I. Eidelson. ``Dangerous Ideas: Five Beliefs that Propel Groups Toward Conflict.'' American Psychologist 58 (3), March 2008, 182-92.
\item Fisher, Robert and Ury, William (and William Paton in the 2nd Edition). 1991. Getting to Yes. Boston, MA: Houghton Mifflin.
\item Folberg, Jay and Taylor, Allison. 1984. Mediation: A Comprehensive Guide to Resolving Conflicts Without Litigation. San Francisco, CA: Jossey-Bass.
\item Folberg, Jay and Golann, Dwight. 2011. Mediation: The Roles of Advocate and Neutral. New York, NY: Aspen Publishers
\item Fromm, Erich. 1973. The Anatomy of Human Destructiveness. New York: Holt, Rinehart, and Winston.
\item Goldberg, Steven. 2006. ``The Secrets of Successful Mediators.'' Negotiation Journal, July 2005.
\item Hensler, Deborah R. 2003. ``Our Courts, Ourselves: How the Alternative Dispute Resolution 
\item Movement Is Re-Shaping Our Legal System,'' 108 Penn St. L. Rev. 165.
\item Ikegami, Eiko. 1995. The Taming of the Samurai. Cambridge, MA: Harvard University Press.
\item Kagan, Robert A. 2001. Adversarial Legalism: The American Way of Law. Cambridge, MA: 
\item Harvard University Press.
\item Kano, Jigoro. 1932. ``The Contribution of Judo to Education,'' Journal of Health and Physical Education 3, 37--40, 58.
\item \_\_\_\_\_\_\_\_\_\_\_\_\_\_. 1990. ``The Life of Jigoro Kano.'' AikiNews, 85.
\item Kerr, Michael. 1988. ``Chronic Anxiety and Defining a Self,'' The Atlantic Monthly, Sept., 35-58.
\item Kotev, Stephen. 2001. ``Aikido and Conflict Resolution.'' Unpublished.
\item Kriesberg, Louis. 2007. Constructive Conflicts: From Escalation to Resolution. Rowman and Littlefield. 
\item Levine, Donald N. 2005. Powers of the Mind: The Reinvention of Liberal Learning in America. Chicago: University of Chicago Press.
\item  \_\_\_\_\_\_\_\_\_\_\_\_\_\_\_\_\_. 2006a. ``The Masculinity Ethic and the Spirit of Warriorhood in Ethiopian and Japanese Cultures.'' International Journal of Ethiopian Studies 2, Nos.1\&2.
\item \_\_\_\_\_\_\_\_\_\_\_\_\_\_\_\_\_\_. 2006b. ``Somatic Elements in Social Conflict.'' In Embodying Sociology: Retrospect, Progress and Prospects. Ed. Chris Shilling. London: Wiley. 
\item Melnick, Jed. 2013. ``Lost Opportunities in Mediation.'' Westlaw Journal, Securities Litigation and Regulation 19, no. 4 ( June), 1-4.
\item  Murata, Naoki. 2005. ``From `Jutsu to D\={o}: The Birth of K\={o}d\={o}kan Judo.''' In Alexander Bennett, ed., Budo Perspectives. Auckland: Kendo World.
\item Piazza, Antonio. 2004. ``The Physics of Aikido and the Art of Mediation.'' Unpublished.
\item Rhee, C. H. Van. 2005. European Traditions in Civil Procedure. Antwerpen: Intersentia.
\item Rosenberg, Marshall B. 2005. Nonviolent Communication: A Language of Life. Encinitas, CA: 
\item Puddle Dancer Press.
\item Saposnek, Donald. 1998. Mediating Child Custody Disputes: A Strategic Approach. San Francisco, CA: Jossey-Bass Inc. 
\item Simmel, Georg. ([1908] 1992). Soziologie: Untersuchungen \"{u}ber die Formen der Vergellschaftung. Ed. Otthein Rammstedt. Suhrkamp.
\item Steiner, Mark E. 1995. ``The Lawyer as Peacemaker: Law and Community in Abraham Lincoln's Slander Cases,'' Journal of the Abraham Lincoln Association, Vol 16, No. 2. Summer 1995.
\item Stipanowich, Thomas J. 2010. ``Arbitration: The New Litigation,'' 2010 U. Ill. L. Rev. 1.
\item Weinstein, Daniel. 2004. Talk to J.D. students at Northwestern School of Law. Unpublished.
\item Wilson, William S. 1982. Ideals of the Samurai: Writings of Japanese Warriors. Burbank, CA: Ohara. 
\end{list}