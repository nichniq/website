\chapter[Martial Arts as a Resource for Liberal Education (1991)]{Martial Arts as a Resource for Liberal Education: The Case of Aikido\footnote{1991 ``Martial Arts as a Resource for Liberal Education: The Case of Aikido,'' in \emph{The Body: Social Process and Cultural Theory}, eds. M. Featherstone, M. Hepworth, \& B.S. Turner (London: Sage), 209-24. [Reprint of 1990.] This paper was originally presented at the U.S.-Japan Conference on Japanese Martial Arts and American Sports: Cross-Cultural Perspectives on Means to Personal Growth, University of Wisconsin-Madison, August 7-10, 1989. I am grateful to David Waterhouse and Clifford Winnig for suggestions which helped me improve the final version.}}

In the Autumn 1984 issue of Liberal Education I published ``The Liberal Arts and the Martial Arts,'' an essay which explored how efforts to rethink the rationales of liberal education might benefit from comparing the liberal arts as developed in the West to certain educational programs, commonly known as the martial arts, developed in the cultures of East Asia. The paper made three main points.

To begin with, I suggested that the distinction embodied in the Japanese contrast between bujutsu and budo parallels an age-old Western distinction between strictly utilitarian arts and arts that possess a liberal character. The Japanese distinction contrasts techniques used for practical, combative purposes (bujutsu) with disciplines that employ training in combative forms as a means to cultivate the students' physical, mental, and spiritual powers (budo). The Western distinction derives from Aristotle's discrimination of knowledge which is tied to necessities and so of a servile sort from the kind of knowledge that is worthy of free men (eleutheron)\footnote{\emph{Politics}, 1255b, 1258b.}---a notion embodied in later formulations about the liberal arts (Greek: eleutheriai technai; Latin: artes liberales), arts whose study was intended to cultivate a person's ``humanity.'' In both cases, techniques\footnote{Sino-Japanese \emph{jutsu} corresponds exactly to Greek \emph{techne}.} learned for mundane instrumental purposes stand in contrast with arts which are studied in order to enhance their learner's capacities as a free and virtuous human being. 

Second, I suggested that affinities between the traditions from which both budo and Western liberal arts emerged could be found by noticing parallels in their patterns of historical evolution. In the West, we find in ancient Greece the ideal of paideia, the notion of using culture as a means to create a higher type of human being. Classic Greek thought celebrated the way to ar\^{e}te, or virtue, through cultivating powers of the body, like strength and vigor, as well as powers of the mind, like sharpness and insight. In later centuries cultivation of the body disappeared as a component of liberal training, so that only intellectual arts, organized eventually as the trivium and quadrivium in the Middle Ages, emerged as suitable subjects for liberal learning. Transmitted by monastics for centuries, this curriculum entered secular universities during the Renaissance. American educators of the late 19th century hearkened back to this Renaissance tradition while devising a program of liberal education oriented to the ``formation of character'' and the goal of self-realization. This formed the intellectual background for the experiments in the liberal curriculum which flourished in the United States after World War I.

I traced a comparable development in East Asia, beginning with the movement in China during the Chou dynasty to form an educational program aimed to produce a broadly cultivated person. This curriculum, often referred to as the ``liberal arts'' of classical Chinese education, included training both in literary and martial subjects. Confucius articulated the conception of the ideal person to be produced by this Chinese version of paideia.\footnote{Cf. Max Weber: ``For the Confucian \ldots{} the decisive factor was that \ldots{} in his self-perfection [the ``cultured man''] was an end unto himself, not a means for any functional end.'' \emph{The Religion of China: Confucianism and Taoism}, trans. and ed. H. Gerth (New York: Free Press, 1951), 246.} The eventual decline of that curriculum was followed by the institution of new kinds of martial arts training in Chinese monasteries, which cultivated Shaolin Temple boxing, derived from exercises introduced by the Indian Buddhist monk Boddhidharma and, subsequently, the Taoist-inspired forms of tai chi chuan. In Japan during the Tokugawa Shogunate, a number of samurai adapted the martial techniques into vehicles of spiritual training and, beginning with the efforts by Jigoro Kano in the 1880's, a number of Japanese arts evolved to constitute the resources of modern budo. 

The main part of my paper, finally, drew on the experience of martial arts training programs to suggest ideas relevant to a number of central issues in the modern philosophy of liberal education. These issues included the question of what is ``liberal'' about liberal education; the kinds of cultural forms most suitable for a liberal curriculum; the kinds of capacities liberal training should foster; the characteristics of training programs designed to cultivate those capacities; the relationship between liberal and utilitarian learning; and the ethical justification of liberal learning. 

In that earlier paper, then, I used training programs in the martial arts as a source of ideas to enrich our thinking about the liberal curriculum. I did not explore the possible role which actual training in the martial arts might play in contemporary programs of liberal education, nor did I explore the ways in which the philosophy of the liberal arts might provide ideas for enriching instructional programs in the martial arts. These two questions form the agenda of the present paper. In addressing them I shall first discuss some general issues raised by the aspiration to incorporate budo training into programs of liberal education. I shall then report on an experiment in which I have incorporated martial arts training in an academic course and conclude by reflecting on some implications of that experiment for those who might like to attempt similar efforts in other institutions.

\section*{Is There a Role For Budo in the Liberal Curriculum?}

In my earlier paper I proceeded on the assumption that there are no inherent differences between the educational approaches of budo and the liberal intellectual arts. At this point I wish to question that assumption and suggest that in certain respects budo training appears incompatible with the objectives of the kind of liberal education suited for modern democratic societies.

Although there are clear lines of continuity between the ideals of paideia and humanitas which informed the liberal curricula of ancient Greece and Rome and subsequent developments in the history of Western Civilization, what constituted liberation and the development of humanity underwent changes. In each epoch new curricula and rationales had to be devised to accommodate changes in the state of knowledge, in the circumstances of life, and in the meaning of a free and fully realized human being. In the course of the 20th century, a number of Western educators have worked to articulate the aims and rationales of a liberal education appropriate to life in advanced industrial world society. If, now, we wish to find a place for budo within this emerging educational culture, we must consider whether or not the properties of budo as it emerged from Japanese feudal martial traditions are in all respects consistent with the ethos of a modern liberal education.

Suppose we identify the central features of the state of knowledge in our time as those of accelerated rationalization and fragmentation; and the central features of our historical situation as those of one small world and cultural diversity. Then what notions should guide the construction of educational programs which cultivate the arts of freedom appropriate to the conditions of life in the late 20th century? Two notions would command a great deal of consensus among modern exponents of liberal education, I believe: autonomy and generality. We want students to become autonomous as persons, able to critically understand rationalized courses of thought and action, to formulate rational grounds in support of their positions and present their thoughts clearly and persuasively, and to recover relevant traditions and adapt them creatively to changing circumstances. We want students to attain general breadth, in the senses of possessing ideas and skills which can apply to broad domains of experience, of being able to find connections among dispersed branches of knowledge, and having the capacity to understand and communicate with persons oriented by radically diverse cultures.\footnote{In a powerful elaboration of many of these points which Richard McKeon set forth a quarter-century ago, the liberating arts were described as `general' in four senses. ``They are general in the sense of applying to all subject matters and therefore in the sense of providing an approach to any particular subject matter placed in a context of other parts of information or knowledge. They are general in the sense of embracing all fundamental skills that can be acquired in education and therefore in the sense of providing a basis for any particular skill. \ldots{} They are general in the sense of bearing on the formation of the whole man and therefore in the sense of providing a model or ruling principle for any particular excellence fitted into achievements of a good life. \ldots{} [T]hey are general in the sense of being the arts of all men and therefore in the sense of providing guidance for each particular man and each particular association of men responsive to the cultures and objectives of other men and of mankind.'' ``The Liberating Arts and the Humanizing Arts in Education,'' in Arthur H. Cohen, ed., \emph{Humanistic Education and Western Civilization} (NY: Holt, Rinehart, and Winston, 1964), 171-72.}

If we take some formulation such as this as a standard for the kind of liberal curriculum that is suited for our times, then we may question whether contemporary forms of budo training are in fact conducive to the educational goals of autonomy amidst complexity and rapid change, and generality amidst fragmentation and diversity. A good deal of contemporary budo practice exhibits characteristics one could describe as authoritarianism, anti-intellectualism, particularism, doctrinaire rigidity, narrowness of focus, and excessive competitiveness.

\begin{itemize}
\item Authoritarianism. It is common to attribute absolute authority to the instructor in a dojo. The sensei must not only be accorded complete respect, but no aspect of his teaching is to be questioned. In describing the pedagogy of the dojo, Richard Schmidt among others has observed: ``The sensei serves as the model for the trainee to emulate. Long and difficult hours of intense, repetitive training and prescribed movements punctuated at times by physical and verbal abuse by the sensei is the mode of instruction.''\footnote{Richard J. Schmidt, ``Japanese Martial Arts as Spiritual Education'' Somatics, Aut./Win., 1983, 47, citing H. Befu, \emph{Japan: An Anthropological Introduction} (1971).}
\item Anti-intellectualism. Budo teaching places a great premium on nonverbal training and often exhibits a studied hostility toward discursive presentations of any sort. As Richard Schmidt further observes: ``Reflective of the Zen method of training, the emphasis is on a nonverbalized, intuitive approach rather than rational intellection. The trainee is encouraged to `think with his body' and not with his mind.''\footnote{\emph{Ibid}., 48.} It is generally considered poor form to discuss issues regarding principles or techniques while training.
\item Particularism. Many martial arts senseis expect absolute loyalty to their persons and their organizations. Some senseis even forbid their students to train with any other instructor while they are under his tutelage. This trait accounts for the pronounced sectarianism which afflicts a number of budo organizations.
\item Doctrinaire rigidity. The combination of authoritarianism, anti-intellectualism, and particularism supports the belief that the teachings of a particular sensei represent the one right way of doing things. His approach is presented as one which all students must reproduce faithfully in every detail.
\item Narrowness of focus. Virtually all the training in most dojos is confined to the mastery of a circumscribed set of techniques. Although these may be taught on the assumption that that kind of training develops the student in accord with certain more general principles, those principles are rarely articulated. It is even more rare to find explicit consideration given to ways in which those principles might be applied in other domains. 
\item Excessive competitiveness. Some schools of budo place considerable emphasis on competition, both within the dojo and with other, rival, dojos. It becomes a primary goal to defeat the ``enemy,'' which can be another student, members of another school, or another martial art.
\end{itemize}

Insofar as these characteristics are inherent in budo, it would seem that they operate in an illiberal direction. However appropriate they may have been in earlier times, they seem inconsistent with the objectives of a liberalizing and humanizing approach to education suitable for the late 20th century. Authoritarianism and anti-intellectualism run counter to efforts to cultivate personal autonomy; particularism, rigidity, and narrowness of focus run counter to the spirit of generality; and an exclusively competitive ethic runs counter to the capacities for mutual understanding and synergistic collaboration which arguably are essential to the advancement of the life of the human species at this point in history.

This raises the question whether one can modify these features of traditional martial arts pedagogy in a liberalizing direction without losing the heart and soul of authentic budo. I believe it is possible. My belief is inspired by the fact that a number of exemplary aikido teachers have shown ways of doing so.

On the matter of authoritarianism I have witnessed a number of prominent aikido teachers question this as an absolute value, by example as well as by precept. Although they naturally expect proper respect, they do not appreciate slavish compliance or obsequious attention. While following the sensei's directives remains an important condition for proper training, if only for reasons of safety, this is fully compatible with an active and questioning spirit on the part of students. Some of the most highly ranked aikido instructors with whom I have trained often conclude their demonstration of a certain technique with the remark: ``Try this out and see if it works for you.'' In my own course, to be described presently, I give students an opportunity to raise questions from time to time on the mat, and encourage them to reflect on our practices critically when they are off the mat.

Again, one can affirm the importance of nondiscursive teaching and nonverbal learning in the dojo without supposing that committed training in a martial art entails the sacrifice of the intellect. Nonverbal learning is good for the mind as well as the body, but one can also benefit from reflection and discourse about what one has learned thereby. 

Although it is natural and helpful to develop sentiments of attachment to one's sensei, this need not take the form of fanatic or highly partisan loyalty. As Mitsugi Saotome Shihan has written wisely on this point, ``Blind loyalty is most dangerous for it is all too easy to twist the ideas of loyalty and righteousness with the lever of human greed and selfish ego.''\footnote{\emph{University of Chicago Aikido Club Handbook} (1989), 24. See also Mitsugi Saotome, \emph{The Principles of Aikido} (Boston \& Shaftesbury: Shambhala, 1989), 198: ``If you accept the idea that \emph{budo} is a study that can encompass all aspects of your life, there is another fallacy which you must avoid. This is the temptation to turn the teachings of your art into doctrines, or your teacher into an idol. \ldots{} Your teacher is a guide, not a guru. There is a great difference between respect and idolization.''} Some aikido senseis make a point of encouraging their students to visit other dojos and to train with different kinds of instructors. The Founder of aikido, Morihei Ueshiba, encouraged aikido students to learn from as many teachers as possible. 

On the issue of doctrinaire rigidity, two points can be made which draw on the most reputable of budo authorities. At the highest level of practice, one can cite the ideal which many budo masters subscribe to, that of the ``technique of no-technique'' or the ``form of no-form.'' Indeed, one interpretation of that formula could serve as a standard for the highest ideal of liberal education, in which particular forms are viewed merely as resources to be employed variably as the occasion indicates. A magnificent formulation of this ideal appears in the dictum by Matsuo Basho, ``Only by entering into the principles and then taking leave of them can one attain autonomy.''\footnote{Cited in Uzawa Yoshiuki, ``The Relation of Ethics to Budo and Bushido in Japan,'' paper presented at U.S.-Japan Conference on Japanese Martial Arts and American Sports, 10.} In addition, one can cite the importance which great budo masters have accorded to continuous growth and change. Recall the dictum attributed to the 17th-century master, Miyamoto Musashi---``the purpose of today's training is to defeat yesterday's understanding''--- not to mention the experience of Founder Morihei Ueshiba, who continuously changed ideas as his practice evolved. 

A certain amount of rote training is indispensable for any art. One must drill basic movements in any martial art just as one must practice scales and arpeggios in learning to play musical instruments. Yet to master techniques without learning the principles which underlie them is patently illiberal, and it is also illiberal to learn principles but to confine their application to a narrow domain. Budo faces the challenge of finding ways to apply its principles to domains outside the martial art in question. A number of aikido masters have met this challenge with enormous creativity. Koichi Tohei Shihan has written books on the application of aikido principles in daily life. Robert Nadeau Sensei has devised a repertoire of ways to show the applicability of aikido moves to interpersonal situations off the mat. Frank Doran Sensei regularly articulates the more general human meanings of various aikido principles and gestures.\footnote{On the connection between \emph{budo} applications and general knowledge, see also Mitsugi Saotome's statement: ``\emph{Budo} means organizing society. It is management. \ldots{} Unfortunately, many managers come from very narrow, categorizing educations. How many business schools are teaching universal knowledge? They give specialized knowledge but never make a `general mind.' Modern universities seem to pursue the opposite of the original meaning [a place to study universal knowledge]. Some professors do not study biology or the ecology of systems, not even human psychology. They don't understand what it means to be human. Many of the problems are caused by very narrow professional people controlling the world. \ldots{} Top executives must study philosophy, religion, nature, art, science; otherwise they do not have the knowledge to create a vision for themselves and their workers.'' ``Budo and Management," \emph{Aikidoka}, Vol. 1, No. 3 (Washington DC Aikikai, 1987), 7-11.}

Finally, one must question the extent to which a competitive spirit is needed to achieve the developmental goals of budo training. This question is complicated by the surface similarity of competitive and combative ethics. While too much competitiveness is degrading, most forms of budo which are entirely ``liberal'' in orientation focus mainly on combat. At issue here is a distinction between becoming proficient at combat as a way to advance at the expense of others and becoming proficient for the sake of defending oneself and others, and improving one's own character.

Master Morihei Ueshiba understood this distinction and how easy it is to confuse the two notions. He wanted to guard against the competitive spirit in aikido, so he removed the aspect of competitive combat from the art. He proclaimed that the only victory worth going for was the victory over one's self, and that the only kind of character worth cultivating in our time is one devoted to the task of bringing peace to mankind around the world. His words eloquently depict the transformed budo this entails:

\begin{quote}
In Ueshiba's budo there are no enemies. The mistake is to begin to think that budo means to have an opponent or enemy; someone you want to be stronger than, someone you want to throw down. In true budo there is no enemy or opponent. \ldots{} True budo is the loving protection of all beings with a spirit of reconciliation. Reconciliation means to allow the completion of everyone's mission.\footnote{``Memoir of the Master,'' in Kisshomaru Ueshiba, \emph{Aikido} (Tokyo: Hozansha, 1974), 179-80. Translation altered.}
\end{quote}

\section*{Employing Martial Arts Training in a Liberal Arts Program}

I turn now to report on an experiment in which I have incorporated martial arts training in an academic course and present some reflections on what that experience suggests for colleagues who might like to attempt similar efforts in other institutions.

Over the past few years I have twice taught a course at The University of Chicago which includes martial arts training as an integral component. Offered as a regular credit course under my Department of Sociology, it is called ``Conflict Theory and Aikido.'' Half of the time this course proceeds like any other academic offering. Twice a week I meet with the students to discuss a series of texts, chiefly writings by sociologists and philosophers which deal with the sources, dynamics, and consequences of different forms of human conflict. 

In addition, twice a week the class meets at the mat, for a systematic introduction to the art of aikido. (I also ask the students to participate in at least half a dozen of the regular training sessions of the campus Aikido Club.) I define the mat training sessions as ``lab'' sessions and ask the students to keep a lab notebook in which they write down after each session some lessons learned and questions raised by the mat training. The grade for the course is based on six components: frequency of training, performance in a modified 6th-kyu test taken during exam week, quality of the lab notebook, participation in class discussions, short assigned papers, and a final paper in which the students are asked to integrate the major things they have learned in the course as a whole. 

In organizing the sequence of sessions on the mat, I attempt not only to provide a graduated introduction into the art of aikido, but also to time certain mat experiences so that they will be relevant to issues raised by the reading. For example, I introduce the notion of ma-ai, the proper distance between training partners, in connection with the sociologist Georg Simmel's discussion of the proper distance between individuals in social interaction; or I focus on the alternation of attack and defense in aikido training with the notion of ``reciprocal priority'' discussed by the philosopher Walter Watson. 

In presenting this course, I have four chief educational objectives.

\begin{enumerate}
\item By having the students experience regular physical activity as an integral part of the class work, I attempt to overcome the mind-body split which so pervades Western education. Besides reading about issues involving human conflict, on the mat we have an opportunity to experience actual feelings which accompany the expression of physical aggression and the different responses, conflictual and non-conflictual, which one can make to that aggression. As a sociologist, I find this particularly valuable since my academic discipline tends to operate at a high level of abstraction and often represents human relations as though they took place outside of human bodies.\footnote{In recent years some sociologists have in fact rediscovered the body. For a seminal contribution, see Bryan S. Turner, \emph{The Body and Society} (Oxford: Blackwell, 1984). Arthur W. Frank has composed an exceptionally helpful overview of this newer literature, in an essay entitled ``Bringing Bodies Back In: A Decade in Review,'' \emph{Theory Culture \& Society}, forthcoming.}
\item By acquainting students with traditional dojo etiquette and basic aikido ideas, I provide an experiential basis for some cross-cultural learning. Aikido is particularly suitable for affording entree into a number of Asian traditions, including Hinduism, Buddhism, Confucianism, Taoism, Shinto, and Bushido, as well as elements of the Japanese language.
\item The major theoretical point of the course is to refine the student's abilities to think critically about human conflict, both descriptively and normatively. I try not to sell a particular point of view on the subject but require that students articulate and reflect on the assumptions regarding conflict which they bring to the class. At the first session, I asked them to write a short paper indicating what they understand by conflict, whether they think that conflict is good or bad, and what questions about conflict they would most like to have answered. At the end of the course, I asked them to return to their initial formulations and write a long essay which incorporates ideas and insights provided by the texts and the training experience. 
\item Throughout the course, I attempt to cultivate their ability to follow the aiki\footnote{The central concept of aikido, \emph{aiki} refers to the process by which energies from different sources are brought into harmonious integration rather than opposition.} way, in everything they do related to the course and not just on the mat. In reading, I encourage them to respect the ki of the author and to blend with it in a centered way. In learning, I encourage them to treat mistakes as useful features of the learning process. When they communicate with one another in class discussions, I encourage them to use aiki principles of communication, instead of ignoring or combatting responses from their fellow students. I encourage them to think of ways to adapt aiki principles to their life outside the classroom. In my own teaching, I attempt to model the aiki approach, respecting the ki of the students and blending with it to make the points I wish to get across. More generally, I encourage them to think of ways to extend aiki modes of response into all aspects of their living.
\end{enumerate}

\section*{Outcomes of the Course}

In discussing the outcomes of this course, I shall incorporate statements made by the students in their lab notebooks and their final papers. 

\begin{enumerate}
\item One outcome of the course related to the goal of integrating experiences of the body with experiences of the mind. Many students appreciated the challenge presented by an opportunity to experience non-verbal learning. Some expressed appreciation for the special kind of learning that only bodily practice provides---
\begin{quote}
I am sore in a real and profound way that only a good night's sleep will cure. I had one worthwhile thought during the club session this evening. Conflict is only one possible outcome of one person's violence. The point of aikido is to prevent this violence from resulting in conflict. On paper, this hardly seems a profound comment, but my body is beginning to understand the concept.
\end{quote}
or the access physical practice provides to truths which are not accessible through verbal means---
\begin{quote}
If, in fact, thinking and speaking and reasoning are all mere imitations or descriptions of some greater truth, it seems hopeless indeed that we could ever know such a truth. \ldots{} Aikido is one way of learning the nameless truth---while I cannot explain what ki is, I can certainly experience it as it flows through me or when it throws me to the ground.
\end{quote}
For some students the challenge of experiencing pain in a protected space provided a stimulus to reflection:
\begin{quote}
One thing that impressed me during our first meeting today was the obvious fact of physical stress. I am accustomed to exertion, but not self-imposed, arbitrary pain, i.e., the self-torture of sitting seiza. It is very interesting to experience, but only endurable if one assumes the view that it is good. One must adopt the ethos of nobility in self-denial, the importance of the ritual, and grim, unhesitating determination with the immediate task, in order to persist. I did so, though it is really contrary to my normal way. 
\end{quote}
Some students were able after a relatively short period to experience a different state of consciousness attendant on the experience of bodily relaxation:
\begin{quote}
I have discovered a state in myself which I call the simple mind. I discovered the simple mind by accident when I actually joined with uke's\footnote{In aikido practice, \emph{uke} signifies the person who initiates the attack and takes the fall.} ki and successfully defended myself against katate-dori. When uke attacked, I was day dreaming and relaxed; I was not thinking of the impending attack. When uke attacked, I simply reacted without thinking. My response was hardly fluid or graceful but it was more powerful than anything I have ever done. The simple mind, I deduce, is a state of readiness that can only be reached, permanently, through years of practice. It is a state, not of thinking or reasoning, but of knowing. The simple mind reflects an understanding that is so deep and innate that it operates without conscious thought or effort. I doubt that I have the discipline to achieve what I term the simple mind but I feel privileged to know that it exists. In other words, I feel as if I was afforded a rare glimpse of what I can possibly achieve.
\end{quote}
Many students came to understand the importance of patience in learning worthwhile skills. Thus:
\begin{quote}
Frustration again wins the day. I can never seem to do anything in the way it is supposed to be done. I am beginning to think that I will have to conquer tremendous obstacles just to become coordinated. I wish that there was some short-cut to grace, but I know that effort is the only answer. \ldots{} The first rule of aikido should really be patience!
\end{quote}
Finally, many students came to an awareness of the possibility of new forms of body-mind integration. Thus one student wrote:
\begin{quote}
Strangely, I have always been cognizant of a ki force but I located its center in my skull, not my body. However, I like aikido's hara location better because it could forge a link between my mind and body that I have always lacked. In the past, I tended to view my body as nothing more than a vehicle for my brain. I am hoping to forge real mind-body connections so that I can break out of this mold.
\end{quote}
\item The course did appear to provide a relatively efficient way to give students entree into exotic features of a different culture. This was particularly visible with regard to respect rituals which are emphasized in the aikido dojo. Following the first day of training, one student wrote:
\begin{quote}
Today, I overcame a taboo; I accepted bowing. In addition to the foreignness of the custom, bowing to another human is considered unacceptable to Judaism. However, I tried to think like a visitor in another culture. I know that bowing in Japan is a sign of respect, not worship, and thus I should view it only as a courtesy. If I were in Japan I would bow and thus I should accept it here. If nothing else, today I accepted bowing. 
\end{quote}
Following the second day of training, this student wrote:
\begin{quote}
Today I felt a little less intimidated with the rituals that accompany the training. I accepted bowing as a foreign but valid method expressing courtesy and respect.
\end{quote}
For other students, the course provided experiences which facilitated their understanding of notions from East Asian traditions which previously they had only grasped intellectually. So, one student wrote that he had previously had some understanding of the concept of ki from a Japanese civilization course, but previously it was hard for him not to intellectualize the idea and just feel it. Others made similar comments regarding the concept of hara. Finally, some students responded to my invitation to regard the whole practice of aikido as a text and to consider it critically in comparison with other kinds of texts. One student, for example, wrote an extended comment on the question of whether philosophical conceptions embodied in Asian notions of ki and chi are compatible with concepts generated by Western positive science.
\item The practice of aikido facilitated the students' inquiry into the nature of human conflict in a number of ways. It not only gave them a concrete physical anchoring of some of the phenomena we were talking about; it gave them resources for raising new kinds of questions about the meaning of conflict. This was true with respect to the status of conflict in aikido itself. As one student wrote:
\begin{quote}
It appears that on the mat that we are turning another's aggression toward ourselves to work for our benefit, but why all this talk of ``avoiding conflict?'' The phrase, ``getting off the line'' sounds like ``avoiding the conflict.'' In the same movement we will use the force an opponent applies to us in order to engage in contact/conflict to overpower him or make him weak. Is that not engaging in conflict? Is that not using our forces to surmount another? So is the significance of aikido to avoid conflict---to reduce conflict---to resolve conflict---or to stimulate conflict?
\end{quote}
It is precisely that kind of probing, that encounter with the ambiguities of conflict within and outside of aikido, that enables the students to reach a much more sophisticated level of thought when considering the subject of conflict.
\item In learning the aiki way, a number of students felt that they had acquired a resource that would be helpful in many other learning contexts. The students who habitually rebelled against authors found that they could learn to respect the ki of the authors without sacrificing their own individuality, their ability to remain centered. Students learned how to integrate mistakes as part of the learning process, rather than waste energy blaming themselves and expressing remorse for making mistakes. They learned to listen to and communicate with one another in a more empathic and constructive way. Thus, about halfway through the course, one student wrote in her lab notebook:
\begin{quote}
I sense a different feeling among the members of our class in and out of the dojo. We all appear to communicate better and more freely among ourselves. Smiling and praising are so much more present than they were at the beginning of the quarter. 
\end{quote}
More generally, most of the students found some ways in which the training experiences on the mat carried over into benefits for their everyday living. One student summed up his experience:
\begin{quote}
The most important thing I learned from the mat sessions is the concept of relaxing, ``joining with the surrounding ki.'' \ldots{} When relaxed, one feels more confident about working or studying; there exists no mental resistance or tension in writing or thinking or just talking with people. When stress or conflict arises, I relax and accept the ki of the offender or attacker, which in return calms him/her also. On one occasion, someone pointed out that I ``radiate an aura of calm,'' which caught me off guard, seeing I feel no different from when I began this course.

Not only did I learn to ``relax,'' I also learned the concept of being ``centered.'' When one is centered, one is in control. In Coleman's diagram of the stages of conflict, conflict escalates because there does not exist a controlling element in its progression. Coleman presents barriers to control the progress of the escalating conflict but provides no control for conflict itself. In the way of dealing with conflict, there exists a center, a calm, relaxed center, containing the range of conflict.

The notion of being centered also transcends aikido and the dojo; [it can] establish a sense of control or stability in your environment. Being centered allows one to be in control of the effect of external forces rather than being controlled by these same forces. These external forces will generally create unnecessary confusion and anxiety, causing one's ki to be ``off.''
\end{quote}
\end{enumerate}

\section*{Concluding Reflections}

Courses on the dynamics of conflict or on conflict resolution provide logical contexts in which to introduce aikido practice. Yet I could imagine other kinds of thematic foci with which aikido practice might be coupled beneficially. One could readily organize a course around any of the other themes I mentioned at the beginning, such as an introduction to East Asian civilization or a course on body-mind connections. 

Topics like the body-mind nexus, the East Asian connection, and the dynamics of conflict represent academic themes which could be linked with a wide range of martial arts, not just aikido. Other kinds of thematic foci might be specific to aikido. For example, I could imagine a course dealing with the aiki process---synergy---as it manifests itself in a wide range of human activities, from the domains of business enterprise or international diplomacy to those of family counseling and the organization of research projects. Training in other martial arts might imaginably be coupled with other, specific kinds of themes. But my sense is that there is a great range of possibilities relevant to both aikido and other arts which I have not yet begun to contemplate. One thinks of courses on religion; on anatomy and physiology; on approaches to healing; on the aesthetics of movement; and so on.

In concluding, I wish to reaffirm my sense that the search for linkages between martial arts training and the liberal arts holds promise for educators. The flow of influence can and should go in both directions. At a time when the pressures of a technicalized society, accelerated now on a worldwide scale, have weakened the traditional case for liberal education, the arts of budo, taught as they were originally intended---as vehicles for personal growth and spiritual enlightenment---provide a formidable exemplar of education for human excellence at its purest. Incorporated judiciously into high school and college curricula, they can add new dimensions to education by focusing on the richness of mind-body learning, new roads for intercultural understanding, new kinds of experience to illustrate general principles, and new ways of being centered in a de-centering universe. On the other hand, martial arts pedagogy stands to be reinvigorated as a force pertinent to the needs of a truly liberating and humanizing culture in our time if it abandons older features of authoritarianism and provincialism in favor of a more open, inclusive, and harmonizing ethos.