\chapter[The Aiki Way to Therapeutic Creative Human Interaction (2007)]{The Aiki Way to Therapeutic and Creative Human Interaction\footnote{Presented at the conference ``Living Aikido: Bewegungs- und Lebenskunst,'' AIKI-Institut f\"{u}r Gesundheitsf\"{o}rderung und Selbstentwicklung, Schweinfurt, Germany, May 19, 2007.}}

\renewcommand{\longleftarrow}{\xleftarrow{\hspace*{8em}}}
\renewcommand{\longrightarrow}{\xrightarrow{\hspace*{8em}}}

\emph{Aiki Waza Michi Shirube}, ``Aiki Training is a Signpost to the Way,'' serves as a motto for the organization co-sponsoring this felicitous event. The saying is ascribed to the Founder of aikido, Morihei Ueshiba O'Sensei. If \emph{Michi}, the Way, is to be understood as the goal of our practice then might we not do well to think about its inner meaning? I think the classical saying that ``the Tao that is told is not the essential Tao'' advises us not to avoid talking about the Way, but only not to assume that whatever words we use possess absolute validity.

Conceptual understandings about the martial arts lag behind what we do in practice. In spite of the historic shift from viewing martial arts training from forms of \emph{jutsu} to approaches to \emph{do}---from techniques of accomplishing something to ways of being (Levine 1991)---available concepts fail to do justice to what we know from the experience of training and teaching \emph{budo}.\footnote{See Shibata 2004 on problems associated with the term \emph{nage}.} We know, for example, that we do not practice aikido as separate individuals but almost always in connection with others. And yet, when we think about the essence of the aiki experience we typically do so with an eye to the improvement of personal character through becoming more accomplished \emph{nages}. Although that perspective is of course valid, exclusive reliance on an individual-centered perspective overlooks the special properties of the \emph{interactions} involved in this joint practice. 

If that is so, we might take a moment to consider the uke-nage transaction as an instance and a metaphor for interhuman relations generally. To examine that transaction fully requires shifting from perspectives centered on individuals to an interactional perspective---to viewing aiki transactions as processes of mutual communication rather than as something that one person does to another. 

An interactional model of the aiki transaction can take different forms. I propose to sketch two of them. For one thing, aiki transactions offer a paradigm of therapeutic relationships of all kinds. In this paradigm, uke is seen as sick, as a patient. In developing this interpretation I draw in particular on the insights and models of Talcott Parsons regarding the ``doctor---patient'' relationship. In a different vein, I conceive uke rather as example of the role of a dynamic creator. Pursuing this notion will take us toward a paradigm that seeks to combine elements from Lao-Tse, Friedrich Nietzsche, and Martin Buber.

My remarks, then, fall into three sections: 1) shifting from focus on single individuals to discourse about social interaction; 2) interpreting aiki transactions as parallel to patient-doctor relationships; and 3) viewing aiki work as modeling the interactions between creators and receptors generally. 

\section*{Paradigm Shift: From Individuals as Such to the Interaction of Parties}

To ground my advocacy of a shift from an individual-centered to an interactional perspective on aikido I need to call on a different sort of \emph{waza}, the history of social theory. This history directs us to observe, first off, that the greatest part of human thought assumes that the proper subject of philosophical, spiritual, and scientific investigations about humans should be the concrete individual. That assumption appears in three major venues. 

\begin{enumerate}
\item We find, in all cultures, a program of human improvement directed to the individual person taken as a moral agent. In this perspective we find, for example, doctrines that regard the person as an entity to be shaped by right discipline; or ennobled by purifying practices; or edified by proper enlightenment; and the like. 
\item In Western moral philosophy, we find a tradition of thought, originating with Thomas Hobbes, that bases its analyses of social phenomena on a concept of the individual as an organism moved by desires, pursuing utilities, and guided by interests. Sometimes referred to as `utilitarianism,' this perspective has gained renewed currency with the ascendance of ``economism'' in the past few decades.\footnote{Ciepley 2006 offers a searching account of social and ideological forces behind the resurgence of economistic worldviews in the United States over the past half-century.}
\item Third, we find a view of the human individual that derives from philosophers like Rousseau, Goethe, Emerson, and Nietzsche---the individual as a subject whose nature is to be expressed, whose personal growth is to be cultivated, and whose creative urges are to be satisfied. This view is sometimes formulated as an effort to promote the cultivation of individuality, a form of modern individualism that has been contrasted with the libertarian individualism championed by thinkers of the Enlightenment (Simmel [n.d.] 1957).
\end{enumerate}

In reaction to these formulations centered in individuals voiced above all by thinkers of the British and German traditions, a number of French thinkers counterposed the notion of `society' as a phenomenon whose natural properties and moral value could not be reduced to those of individual actors. Foremost among these were thinkers such as Montesquieu, Rousseau, Comte, and Durkheim. These thinkers of the French tradition espoused what has been referred to as a notion of ``societal essentialism'' (Levine 1995). (Modern debates between proponents of societal essentialism and those of what has been called ``atomic naturalism'' recapitulated older metaphysical debates between nominalists and realists.)

This opposition between the individual and society dominated nearly all of Western social thought.  There have, however, been two striking exceptions, which emerged toward the end of the nineteenth century. In Germany, philosopher Georg Simmel interposed between those polar terms the notion of ``interaction,'' a domain that had properties, he insisted, that were distinctive and \emph{sui generis}. In the United States, John Dewey and G.H. Mead collapsed the distinction in favor of a notion of socially constituted and societally constituting selves. For Mead, the crucial ingredient of this process was the acquisition and use of language. Both the ability to participate in social interaction and to construct a self-conception, Mead argued, depended crucially on the ability to grasp and internalize the meaning of external objects as symbols. This central process suggests a formulation that works better in German than in English: the birth of dialogue (\emph{Gespr\"{a}ch}) out of the spirit of language (\emph{Sprache}). 

Reaching back to Mead's seminal work, J\"{u}rgen Habermas retrieved the notion of a form of rationality that he called \emph{dialogical}, which he contrasted with the monological rationality that had formed the subject of philosophic discourse previously (Habermas 1984). Well before Habermas, however, the notion of dialogue had been thematized and made central by Martin Buber, whom I regard as one of the philosophers most closely attuned to the Aiki Way. Buber's intellectual development traverses the shift in paradigms of which I have been speaking. He began as a devotee of Nietzsche, from whom he acquired the ideal of intense transcendent experience. Buber became, as his biographer Paul Mendes-Flohr aptly put it, an ``\emph{Erlebnis}-mystic.'' At the University of Berlin he joined the Neue Gemeinschaft, a fraternity dedicated to pursuing the ``Dionysian worldview'' which Nietzsche celebrated. 

At the same time, Buber's studies with Simmel at the University of Berlin planted seeds for a transition away from an exclusive focus on the individual self. Simmel's insistence that psychologistic explanations of interaction are inadequate converted Buber to a perspective in which the interhuman (\emph{das Zwischenmenschliche}) figures centrally. The first step of this transition appears in Buber's introduction to Simmel's essay \emph{Die Religion} (1906) published in \emph{Die Gesellschaft}, a series which Buber edited. In this introduction, Buber endorses Simmel's view of the discipline of sociology, employing Simmelian terms like \emph{Formen der Beziehung}, \emph{Wechselwirkung}, \emph{Vergesellscghaftung} (forms of relation, interaction, association), and affirming Simmel's ontological point:

\begin{quote}
\emph{Das Zwischenmenschliche} is that which occurs between (zwischen) men; in \emph{some ways it is not unlike an impersonal, objective process}. The individual may very well experience \emph{das Zwischenmenschliche} as his `action and passion,' but somehow it cannot be fully ascribed or reduced to individual experience. For \emph{das Zwischenmenschliche} can only be properly comprehended and analyzed as the synthesis of the `action and passion' of two or more men. (cited Mendes-Flohr 1989, 38-9).
\end{quote}

For Simmel, the concept of \emph{forms of association} served to carve out a distinctive domain for the new academic discipline of sociology. Reproducing Simmel's argument in 1906, Buber affirms: ``Sociology is the science of the forms of \emph{das Zwischenmenschliche} \ldots{} [forms such as] super- and subordination, cooperation and noncooperation, groupings, social rank, class, organizations and all types of economic and cultural associations, both natural and normative'' (39).

In spite of this new ontological vision, this awareness of the interaction domain \emph{sui generis}, Buber did not endow social interaction processes with any particular moral or spiritual qualities. He continued to locate transcendence in the sphere of \emph{Erlebnis}, of personal life experienced with the utmost intensity and integrity. Indeed, it was his enthusiastic engagement in the War spirit that brought to Buber, as to so many other German intellectuals of the time, an unprecedented intensity of transcending experience.

What turned Buber away from his War enthusiasm in particular and his idealization of intense personal experience more generally was a traumatic exchange with his close friend Gustav Landauer in May 1916 (in his new family home at Heppenheim, not so far from Schweinfurt). Landauer was one of the few German intellectuals who opposed the War strenuously. After his visit with Buber, Landauer wrote a letter in which he excoriated Buber for the moral lapse of indulging in militaristic sentiments. Mendes-Flohr argues that Landauer's critical letter occasioned a \emph{volte-face} in Buber and writes: ``In Buber's writings subsequent to the spring of 1916, we notice three new elements: an explicit opposition to the war and chauvinistic nationalism; a reevaluation of the function and meaning of \emph{Erlebnis}; and a shift in the axis of \emph{Gemeinschaft} from consciousness (i.e., from subjective-cosmic \emph{Erlebnis}) to the realm of interpersonal relations'' (102).

From that time on, Buber expanded his conception of interpersonal relations in ways that connected it with the wish for transcendence. He came to sacralize what Simmel's lectures had identified simply as a sociological form.\footnote{In the \emph{Die Religion} essay, however, Simmel points the way to Buber's sacralized dialogue by tracing in certain types and moments of interhuman experience the seeds for what becomes objectified as religion.} He came to find in the relation between ``I'' and ``Thou'' an instantiation of ultimate values. In 1914, according to Mendes-Flohr: 

\begin{quote}
Buber, the \emph{Erlebnis}-mystic, spoke of religiosity as a tendency in man that seeks to actuate God's realization; by securing the creative integrity of one's personality one acts to renew the cosmic harmony. In 1919, Buber defined religiosity as the human disposition that affects the realization of God through the establishment of authentic relations: ``Whenever one man joins hands with another, we feel [God's] presence dawning (\emph{aufkeimen})'' (115).
\end{quote}

In sum, Buber had come to find in \emph{das Zwischenmenschliche} the venue for self-transcendence that he had previously sought in Nietzsche's appeal for a peak experience. In this, he later recalled, he was harking back to Ludwig Feuerbach. For Feuerbach, he noted, man

\begin{quote}
does not mean man as an individual, but man with man---the connexion of \emph{I} and \emph{Thou}. ``The individual man for himself,'' runs his manifesto, ``does not have man's being in himself, either as a moral being or a thinking being. Man's being is contained only in community, in the unity of man with man---a unity, however, which depends only on the reality of the difference between I and Thou'' (Buber [1938] 1965, 147-8).\footnote{Buber took this quote from Feuerbach's \emph{Principles of the Philosophy of the Future} (\emph{Grunds\"{a}tze der Philosophie der Zukunft}). This was published in 1843, two years after his most famous publication, \emph{The Essence of Christianity}. The earlier work provided fodder for Marx's famous attack in Thesis VI, where he excoriates Feuerbach by asserting: 

\begin{quote}
Feuerbach resolves the religious essence into the human essence. But the human essence is no abstraction inherent in each single individual. In its reality it is the ensemble of the social relations. Feuerbach, who does not enter upon a criticism of this real essence is consequently compelled to abstract from the historical process and to fix the religious sentiment as something by itself and to presuppose an abstract---\emph{isolated}---human individual. \ldots{} (Tucker ed. 1972, 145).
\end{quote}

It is ironic to compare Marx's words of 1845 to those of Feuerbach in 1843 just cited. The original Feuerbachian text cited by Buber follows. 

\begin{quote}
Der einzelne Mensch f\"{u}r sich hat das Wesen des Menschen nicht in sich, weder in sich als moralischem, noch in sich als denkendem Wesen. Das Wesen des Menschen ist nur in der 

Gemeinschaft, in der Einheit des Menschen enthalten---eine Einheit, die sich aber nur auf die Realit\"{a}t des Unterscheids von Ich und Du st\"{u}tzt. \ldots{} Selbst der Denkakt kann nur aus dieser Einheit begriffen und abgeleitet werden. (Feuerbach [1843] 1903, 318)
\end{quote}

In retrieving these words, Buber goes on to observe: ``Feuerbach did not elaborate these words in his later writings'' (Buber 1965, 148).}
\end{quote}

Buber's journey thereby brought him to a point of fusing the interactionist model of Simmelian sociology with the self-transcending ecstasies projected by Nietzsche. The way to such heights was to be obtained by a concentrated, open, and genuine kind of communication between two subjects. \emph{Buber's notion of genuine dialogue between two committed subjects offers precisely the kind of model of open communication that we strive to attain in the practice of aikido.} 

The possibilities of such interhuman encounters are endless, just as possibilities of uke-nage communication are endless. I turn now to examine two sets of possibilities that are manifest in aiki interactions, forms resonant with our experiences in everyday life. One of those possibilities gets evoked when the person who initiates the interaction presents himself or is perceived to be sick. 

\section*{Uke as a Patient, Nage as Healer: Aiki Interactions as Therapeutic Work}

I attempt now to delineate what I consider rather precise parallels between the therapeutic transaction and the aiki transaction. This effort draws inspiration from three sources. First off, I was struck by how many of those who were initially drawn to the work of Aiki Extensions were themselves psychotherapists or bodyworkers with therapeutic consequence. A number of practitioners claimed to be securing therapeutic results by using aikido techniques or at least aikido-inspired ideas. Indeed, some of them reported accomplishing more by doing aikido with their patients than through any standard therapeutic techniques in which they had been trained. 

Within the non-aikido community of therapists, moreover, I took note of the growing import of those who construe the psychotherapeutic situation in terms of interpersonal process. An earlier proponent of this approach, Jacob Moreno, inventor of sociometry and psychodrama, had in fact acknowledged an explicit indebtedness to Georg Simmel. A number of psychologists were inspired by the pioneering work of Harry Stack Sullivan who defined the therapeutic experience as essentially constituted by interpersonal relationships. 

In pursuing these leads I was struck, as I explored the subject further, by parallels between the founders of these two practices, Sigmund Freud and Morihei Ueshiba. Figure 1 schematizes a few of these parallels. Both men successfully completed rigorous training in conventional disciplines in young adulthood and then, in their early 40s, had breakthroughs associated with intense emotional experiences that led them to found new disciplines and to renounce early martial ambition fantasies (Levine 1984). They were also charismatic figures whose new disciplines---and prophetic postures---inspired international movements which they headed. Moreover, Freud and Ueshiba continued to evolve beyond their mature breakthroughs, remaining active and productive well into their eighties. Both had disciples who trained with them along the way and then went on to transmit the teachings of that phase as \emph{the} orthodox teaching, and they were survived by a number of disciplines whose competitive strivings introduced dissent in what they each hoped would survive them as unitary movements (Beaulieu 2005).

\begin{figure}
\caption{Parallels between Psychoanalysis and Aikido}
\footnotesize
\centering
\begin{tabular}{ | >{\bfseries}l | l | l | }
\hline
Charismatic Founder & Sigmund Freud (1856--1939) & Morihei Ueshiba (1883--1969) \\
\hline
Cultural Context & biologism & martialism \\
\hline
\multirow{2}{*}{Disciple} & head of school & director of institute \\
 & head of ryu & leader of organization \\
\hline
Local Head & supervisor & sensei \\
\hline
Role of Teacher & analyst & sempai/nage \\
\hline
Role of Student & patient, client & kohai/uke \\
\hline
Secessionists & Jung Adler & Tomiki, Tohei \\
\hline
\end{tabular}
\end{figure}

Parallels in their substantive teachings are no less striking. Freud and Ueshiba both propounded an ethic based on nature and respect for the natural propensities of humans rather than on some transcendental conception. Conceptions of natural energetic forces grounded their teachings. Jonathan Lear's words about psychoanalysis apply to aikido: ``Psychoanalysis works both against a devaluation of empirical life and for a reintegration into the flow of life of patients who have been thrown off their middle'' (Lear 2000). Both Freud and Ueshiba identified the sources of human aggression and martial combat in the psychic disposition of humans rather than in culture and social structure. Both illuminated ways in which inner discord gives rise to external discord. Both devised training programs to alleviate inner discord, programs that focused on a slow process of becoming more integrated (inner harmony) as a way to promote external harmony as well as personal freedom. 

Above all, I suggest, both of them invented practices whose meaning they did not fully comprehend, practices which evolved nontrivially through efforts of later practitioners. Others have wondered about this: psychoanalytic theorist Edgar Levenson confessed that ``analysts of all persuasions continue to treat all of their patients with a considerable degree of success \ldots{} and yet are hard put to know exactly how to talk about what it is they do when they do what they know how to do. This ineffable competence can be defined as the \emph{praxis} of psychoanalysis'' (Levenson 1983, 6); and one of Ueshiba's students, Anno Sensei, wondered if what the Master created had not evolved beyond \emph{budo}, or martial arts, altogether (Anno 1999).

Levenson himself attempt to identify the obscure secret of good therapeutic praxis. He describes it as a ``deep structure of cognition \ldots{} [whose] efficacy, no different from that of other forms of propagandizing influence, depends on its resonance to deep structures of thought'' (89). In contrast, I want to suggest that there is \emph{an unconscious structure built into the interactional structure of the therapist-client relationship}, one that is cognate with what Talcott Parsons identified half a century ago as the unconscious structure built into the doctor-patient and many other kinds of socially reintegrative relationships. I believe that both Freud and Ueshiba, through their intuitive genius, created structures whose true significance has only begun to be visible through generations of work since their mature formulations.

During the 1950s, Talcott Parsons came to theorize in different ways the logic of what he termed double-interchange paradigms. The template for this schema came from the depiction of interactional flows of the economic system. Figure 2 shows the familiar schema of this flow in economic exchange, where one party offers labor or its equivalent for goods or their equivalent. 

\renewcommand{\arraystretch}{1}
\begin{figure}
\caption{Double Interchange in the Economy}
\begin{tabular}{l c r}
\hline
 & & \\
\multicolumn{3}{ c }{(Parsons and Smelser, 1956)} \\
\underline{Household} & & \underline{Firm} \\
Has needs & & Has goods \\
 & Labor Services & \\
Accepts employment & $\longrightarrow$ & Offers employment \\
 & Wages & \\
 & $\longleftarrow$ & \\
Purchases & Consumer Goods and Services & Produces \\
 & $\longleftarrow$ & \\
 & Consumer Spending & \\
 & $\longrightarrow$ & \\
 & & \\
\hline
\end{tabular}
\end{figure}

For Parsons, this schema of double interchange offered a template for exchanges among subsystems of action at all levels. He did so unaware that Simmel himself had posited the advantage of doing this when he suggested that ``most relationships among men can be considered under the category of exchange'' ([1907] 1971, 43). 

Prior to presenting this general model of systemic interchanges, Parsons had offered a cognate schema of interchanges in his analysis of the system of medical practice in \emph{The Social System} (1951). In that work and related writings of the period, Parsons analyzed the virtually subliminal structuring of responses of doctors and patients. He did so along lines he would employ later when discussing comparable dynamics in the socialization of children. The net effect of all this was to highlight the unwitting structuring of processes by which the motivations of persons with needs for social integration could be mediated by occupants of roles with resources suited for that task.

\begin{figure}
\caption{Double Interchange in the Medical System}
\small
\centering
\begin{tabular}{l c r}
\hline
 & & \\
\multicolumn{3}{ c }{(Parsons, 1951)} \\
\underline{Patient Role} & & \underline{Doctor Role} \\
Has needs & & Has resources \\
 & Expresses pain & \\
 & $\longrightarrow$ & \\
 & Listens compassionately, does not reciprocate & \\
 & $\longleftarrow$ & \\
 & Offers directions for healing & \\
 & $\longleftarrow$ & \\
 & Agrees to follow doctor's lead, get well & \\
 & $\longrightarrow$ & \\
 & & \\
\hline
\end{tabular}
\end{figure}

With just a little reflection, one can see how closely the elements of the paradigm of medical practice resemble the elements of the uke-nage interaction system. Figure 4 brings out the main aspects of these parallels.

\begin{figure}
\caption{Double Interchange in the Aikido System}
\small
\centering
\begin{tabular}{l c r}
\hline
 & & \\
\underline{Uke Role} & & \underline{Nage Role} \\
Has needs & & Resourceful \\
 & Lashes out & \\
 & $\longrightarrow$ & \\
 & Receives attack, does not reciprocate & \\
 & $\longleftarrow$ & \\
 & Offers better way & \\
 & $\longleftarrow$ & \\
 & Follows nage's lead & \\
 & $\longrightarrow$ & \\
 & & \\
\hline
\end{tabular}
\end{figure}

What this represents is that the script for uke, like that of the patient, is to express his feelings openly. In aiki practice, this is manifest in the advice to ``attack sincerely.'' That is the ``basic rule'' of the psychoanalytic interview, just as it is a basic rule of aiki practice. In response, the task of the therapist/nage is to accept that expression, without getting upset, letting himself be hurt, or reciprocating. The therapist/nage then moves to resolve the situation by guiding the client/uke in a tonic direction. In response to that, the client/uke takes responsibility for changing his patterns by moving in that new tonic direction. This basic schema has been refined in many ways by experienced therapists just as experienced senseis have a repertoire of increasingly subtle ideas. 

Before discussing them, let us step back a moment and note that in order to adapt all of these double interchange paradigms to real situations, one thing more must be added: a starting point or a presenting situation. For the therapeutic situation, two conditions have been identified. One is the setting of the therapeutic interview. It must be defined by ritually demarcated boundaries in time and space, a condition that affords a safe and secure therapeutic ``playground'' for the client, as Freud himself called it. In aikido, the ``playground'' in which the uke and nage carry on is similarly constituted, through the ceremonial marking of boundaries in time (bowing in and bowing out of class) and space (bowing on and bowing off the mat).

The other condition concerns the state of being of the therapist, who is expected to embody a higher degree of integration and whose mind is to be marked by ``evenly hovering attention.'' Similarly, the nage in aikido is expected to strive for a state of being ``centered'' and to maintain a mental attitude marked by ``soft vision.'' In that frame of mind, both therapist and nage can actually initiate the interaction with a ``leading'' move. The therapist can ``lead'' the client to open up with a remark such as ``you seem upset today'' or simply ``how are you feeling?'' The alert nage can sense a coming attack and extend an arm to draw out the imminent energy that uke itches to deliver.

Once the interaction proper begins, a number of subtle responses are likely to be involved. It is hard to imagine the sense of freedom, self-acceptance, self-confidence, and growth that may come in the wake of uke's feeling free to express anything she wishes, or uke's freedom to attack with full sincerity.  There is also an added boost for the client/uke on those rare occasions when they get through to one of the therapist/nage's vulnerable spots. In addition, that the client can be listened to compassionately, that the uke's attack can be graciously received, comprise elements of anticipatory gratification and of actual relief and self-enhancement that may do much to restore confidence in the possibility of genuine I-Thou connecting. It can also be a matter of satisfaction and growth for the therapist and the nage to realize that they in fact possess the capacity not to reciprocate their antagonist's deviant bid and that they have the power to refrain from treating him the way that everyone else normally does. 

That much accomplished, it remains for therapist/nage to resolve what was potentially a difficult problem in a tonic manner. The challenge to them is to avoid making responses that are either exploitative or that involve an improper degree of familiarity. That done in turn, it remains for client/uke to follow their lead in a positive manner, albeit remaining on the lookout for openings and weaknesses in the therapist/nage to make use of as they see fit. It is not productive if they simply wimp along when therapist/nage manifests weaknesses of leadership and shows openings. If client/uke should resist this lead, however, therapist/nage will be challenged not to oppose their resistance but to blend with those any resistance and to soften them. 

Each transaction takes place in a broader context of ongoing interactions. It behooves the therapist/nage to restore attention to the larger context, to mark the boundaries of successive engagements, and to set the terms of continuous work. It is up to the client/uke to integrate what has been learned from each transaction and to get ready for proceeding to the next step.

\section*{Uke as Dynamic Creator, Nage as Creative Receptor: A Six-Stage Paradigm Drawing on Lao Tse, Nietzsche, and Buber}

Instead of viewing uke as a patient, as a pathological actor in need of healing, suppose we reframe the role of uke in a more positive manner. Suppose we carry out the reframing process radically---that we view uke's ostensible aggression as an expression of energy that is to be welcomed for the good it can bring. Such a shift can lead to a reframing of the entire aiki transaction that might unleash a great deal of human potential. The paradigm that I visualize for this interpretation has six components, as in Figure 5. 

This paradigm stays closer to the aikido experience as we know and seek to cultivate it. The paradigm amounts to little more than an effort to take the basic moves that we practice and to extend them directly into everyday responses. It stands at one and the same time as a guide to training and as a guide to life generally.

\renewcommand{\arraystretch}{1.5}
\begin{figure}
\caption{New Uke Paradigm}
\scriptsize
\centering
\begin{tabular}{ | C{.17\textwidth} | C{.16\textwidth} | C{.16\textwidth} | C{.16\textwidth} | C{.16\textwidth} | }
\hline
\textbf{Handeln} & \textbf{Action} & \textbf{Manner} & \textbf{Role} & \textbf{Breathing} \\
\hline
1.Sein & Being & Centered and open & Expecting nothing, ready for anything & Continuous deep breathing \\
\hline
2. Sch\"{o}pfen & Initiating & Energetically & Uke 1 & Exhale 1 \\
\hline
3. Engagieren & Engaging & Harmoniously & Nage 1 & Inhale \\
\hline
4. L\"{o}sen & Resolving & Appropriately & Nage 2 & Exhale \\
\hline
5. Anpassen & Adapting & Creatively & Uke 2 & Exhale 2 \\
\hline
6u. Zuruckprallen & Rebounding & Easily & Uke 3 & Inhale-exhale \\
\hline
6n. Beherrschen & Controlling & Zanshin & Nage 3 & Inhale-exhale \\
\hline
\end{tabular}
\end{figure}

It commits us, to begin with, to find the center of our being in ways that keep us open to the worlds within us and around us. 

It reminds us that, since we are prone perpetually to lose our center, to study more effective ways to regaining center.

It encourages us to align with the yang energy entailed in every creative process, albeit in a way that flavors that extension with the yin of subtlety and control.

It alerts us to be receptors of creative inputs, treating them neither as threats, nor as annoyances, nor as demons.

It bids us offer honest and insightful responses to creative initiatives, such that any destructive or misleading elements they may contain can be redirected into more benign channels. 

It coaches us to be flexible and to learn from obstacles or things that do not work, viewing them not as ``mistakes'' but as a normal part of the creative process.

It tells us to regain our balance after every exchange, returning to a state of readiness to learn, to create, to enjoy, and to be. 

\section*{Conclusion}

The aiki schemas of uke as patient/nage as therapist and uke as creator/nage as receptor are two among many.  I invite you on your own to extend this mode of analysis to other forms in which you may be engaged: parent-child, husband-wife; leader-followers; mediator-client; enemy combatants; whatever. I suggest that it is valuable for us to execute comparisons of this sort with a double aim in mind: to show how aikido practice can deepen our capacities for such experiences off the mat, and no less to suggest how awareness of those applications can enrich our training experiences on the mat.

In setting forth these ideas I hope to have responded to the question with which I began: what can we say about the nature of the Aiki Way, which we try to pursue?

Insofar as we are patients---and we are all patients---it disposes us to reach out when we are in need, to ask for help, and to do so in a sincere and direct manner; and then to respond respectfully and in good faith, yet not blindly, to solutions to our problems offered by those who listen to us. 

Insofar as we are healers---and we are all healers---it inclines us to listen with compassion to requests for help without giving in to illegitimate responses that may be proffered, to learn how to make contact with another while staying attuned to the center of our being, and to develop resources that can be useful in resolving issues that others present from time to time. 

Insofar as we are creators---and we are all creators---it inspires us to express our deepest feelings with courage, honor, and awareness, and to regard obstacles along the way as important components of the entire creative process. ``In the hands of a master,'' one of my music teachers once observed, ``the limitations of a medium become its virtues.''

Insofar as we are receptors---and we are all receptors---we learn to savor the various responses of our partners in ways that show we take them seriously but will not be taken in by gestures that seem misleading or harmful to themselves or us or anyone else. 

The Way of Dialogue, which Martin Buber elucidates on from his devotion to the inspirations of Nietzsche and the profound teachings of Lao Tse, can be enhanced through the somatic practices fashioned by Morihei Ueshiba O'Sensei. I find this point restated with exemplary economy by one of the newer members of Aiki Extensions, David Rubens of London, who wrote in a personal communication: ``One of the blessings of aikido, at least as I have found it in my life and as you have shown in your work with Aiki Extensions, is that it creates a completely effective short-cut to creating connections between people.'' If \emph{aiki waza} is indeed a \emph{michi shirube}, that is not such a bad \emph{michi} to be heading toward.

\section*{References}

\begin{list}{}{\refstyle}
\item Anno, Motomichi. 1999. Interview with Motomichi Anno Sensei, July 11. Conducted by Susan Perry, translated by Mary Heiny and Linda Holiday. Aikido Today Magazine.
\item Beaulieu, S\o{}ren. 2005. ``After O'Sensei: On the Dynamics of Succession to a Charismatic Innovator.'' Unpublished Master's thesis. University of Chicago, Master of Arts Program in Social Science..
\item Buber, Martin. 1965. Between Man and Man. Trans. Ronald Smith. New York: Macmillan.
\item Ciepley, David. 2006. Liberalism in the Shadow of Totalitarianism. Cambridge, MA: Harvard University Press. 
\item Feuerbach, Ludwig. [1843]1903. ``Grunds\"{a}tze der Philosophie der Zukunft.'' S\"{a}mmtliche Werke. Ed. Wilhelm Bolin and Friedrich Jodl. Vol. 2. Stuttgart: Fr. Fromanns Verlag.
\item Habermas, J\"{u}rgen. 1984. The Theory of Communicative Action Vol. 1. Trans. T. McCarthy. Boston: Beacon Press. 
\item Lear, Jonathan. 2000. Happiness, Death, and the Remainder of Life. Cambridge, MA: Harvard University Press.
\item Levenson, Edgar. 1983. The Ambiguity of Change. NY: Basic Books.
\item Levine, Donald N. 1984. The Flight from Ambiguity. Chicago: University of Chicago Press.
\item \_\_\_\_\_\_\_\_. 1991. ``Martial Arts as a Resource for Liberal Education: The Case of Aikido,'' in The Body: Social Process and Cultural Theory, eds. M. Featherstone, M. Hepworth, \& B.S. Turner (London: Sage), 209-24.
\item \_\_\_\_\_\_\_\_. 1995. Visions of the Sociological Tradition. Chicago: University of Chicago Press.
\item Mendes-Flohr, Paul. 1989. From Mysticism to Dialogue: Martin Buber's Transformation of German Social Thought. Detroit: Wayne State University Press.
\item Parsons, Talcott. 1951. The Social System. Glencoe, IL: Free Press.
\item Parsons, Talcott \& Neil Smelser. 1956. Economy and Society. Glencoe, IL: Free Press.
\item Shibata, Beth. 2004. ``Throw versus Release: The Effect of Language and Intention on Aikido Practice.'' www. aiki-extensions.org
\item Simmel, Georg. 1906. Die Religion. Frankfurt am Main: R\"{u}tten \& Loening. Reprinted in Georg Simmel Gesamtausgabe, ed. Otthein Rammstedt, v. 10. 
\item \_\_\_\_\_\_\_. (n.d.) 1971. ``Freedom and the Individual.'' Ch.15 in Georg Simmel on Individuality and Social Forms, ed. Donald N. Levine. Chicago: University of Chicago Press.
\item \_\_\_\_\_\_\_. 1907 (1971). ``Exchange.'' Ch.5 in Georg Simmel on Individuality and Social Forms, ed. Donald N. Levine. Chicago: University of Chicago Press.
\end{list}